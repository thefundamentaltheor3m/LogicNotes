\section{Understanding Roots}

Throughout this section, let $H$ be a Cartan subalgebra of the semi-simple Lie algebra $L$. Let
\begin{align*}
    L = H \+ \bigoplus_{\alpha \in \Phi} L_{\alpha}
\end{align*}
denote its Cartan decomposition. We will denote by $\kappa$ the Killing Form on $L$.

\subsection{Subalgebras of $L$ Isomorphic to $\sl{2}$}

We begin by deducing properties of $L$ (and its Cartan decomposition) by finding copies of $\sl{2}$ inside of it. The point is that we understand $\sl{2}$ very well through its representations, as seen in \Cref{Ch2:Sec:sl2}.

% At this point I've basically given up on Live TeXing...

We begin with a fundamental lemma.

\begin{boxlemma}
    For all $\alpha \in \Phi$, $-\alpha \in \Phi$ as well. In other words, the negative of a root is also a root.
\end{boxlemma}
\begin{proof}
    This turns out to be a direct application of the second point of \Cref{Ch3:Lemma:BehaviourOfWeights}.
    \sorry
\end{proof}

This suggests that there might be certain linear relationships between roots. We will soon discover exactly what those relationships are. In the meantime, we can show the following useful result.

\begin{boxproposition}\label{Ch3:Prop:Exists_Subalg_Isom_to_sl2}
    Fix $\alpha \in \Phi$ and some non-zero $x \in L_{\alpha}$. Then, there exists some non-zero $y \in L_{-\alpha}$ such that $\Span{x, y, \brac{x, y}}$ is a Lie subalgebra of $L$. In particular, this subalgebra is isomorphic to $\sl{2}$.
\end{boxproposition}
\begin{proof}
    
\end{proof}

\subsection{Root Strings and Eigenvalues}

Fix $\alpha, \beta \in \Phi$. Assume that $\alpha \neq \pm \beta$.

\begin{boxdefinition}[Strings]
    The \textbf{$\alpha$-string through $\beta$} is the subspace
    \begin{align*}
        M := \bigoplus_{n \in \Z} L_{\beta + n\alpha}
    \end{align*}
    of $L$.
\end{boxdefinition}

\begin{lemma}
    $M$ is a $\sl{\alpha}$-submodule of $L$, with respect to the adjoint representation.
\end{lemma}
\begin{proof}
    \sorry % WTF does the lemma statement even mean?!
\end{proof}

It does look a bit strange to call something a `string'. The reason for this terminology comes from the following proposition.

\begin{boxproposition}
    \sorry
\end{boxproposition}

\subsection{Consequences of the Cartan Decomposition}

In this subsection, we give some useful facts about $L$ that follow from its Cartan decomposition. It can be seen as a `bag of tricks' that we can reach into whenever we need to prove something about roots.

\begin{boxlemma}
    For all $h_1, h_2 \in H$, we have
    \begin{align*}
        \kof{h_1, h_2} = \sum_{\alpha \in \Phi} \alpha\!\parenth{h_1} \cdot \alpha\!\parenth{h_2}
    \end{align*}
\end{boxlemma}
\begin{proof}
    \sorry
\end{proof}

This allows us to show that $\Phi$ contains a basis of $H^*$, a fact reminiscent of the fact that any root system contains a basis of its ambient space.

\begin{boxcorollary}
    $\Span{\Phi} = H^*$.
\end{boxcorollary}
\begin{proof}
    \sorry
\end{proof}

Seeing as $\Phi \subset H^*$, the Dual Space (with respect to $\kappa$) of the Cartan Subalgebra, we know that every element $\alpha \in \Phi$ is expressible as some map $\kof{t_{\alpha}, \cdot}$ for some \textbf{unique} $t_{\alpha} \in H$.
\begin{boxconvention}
    For any $\alpha \in \Phi$, we denote by
    \begin{itemize}[noitemsep]
        \item $t_{\alpha}$ the unique element of $H$ such that 
        \item $h_{\alpha}$ the unique element of $H$ such that \sorry
        \item $e_{\alpha}$ some nonzero element of $L_{\alpha}$ such that 
    \end{itemize}
\end{boxconvention}

The fact mentioned above will prove useful for the following proposition.

\begin{boxproposition}
    For all $\alpha \in \Phi$, $x \in L_{\alpha}$ and $y \in L_{-\alpha}$, we have that
    \begin{align*}
        \brac{x, y} = \kof{x, y} \cdot t_{\alpha}
    \end{align*}
\end{boxproposition}
\begin{proof}
    \sorry
\end{proof}

Indeed, we can compute an explicit formula for $t_{\cdot}$.

\begin{boxproposition}
    For all $\alpha \in \Phi$, we have that
    \begin{align*}
        t_{\alpha} = \frac{h_{\alpha}}{\kof{e_{\alpha}, e_{\alpha}}}
        \qquad \text{and} \qquad 
        h_{\alpha} = \frac{2t_{\alpha}}{\kof{t_{\alpha}, t_{\alpha}}}
    \end{align*}
\end{boxproposition}
