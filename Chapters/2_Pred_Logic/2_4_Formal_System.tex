\section{A Formal System for First-Order Logic}

The purpose of this section is to define, for any first-order language $\L$, a \textbf{formal deduction system $\KL$} in which we can do \textbf{first-order logic}.

Throughout this section, fix a first-order language $\L$.

\subsection{The Formal Deduction System $\KL$}

Recall \Cref{Ch1:Def:Formal_Deduction_System}, in which we define a \textbf{formal deduction system}. Informally, this is meant to be a system in which we can perform \textbf{deductions}. The purpose of this subsection is to define a \textbf{formal deduction system for $\L$}.

\begin{boxdefinition}[A Formal Deduction System for First-Order Logic]
    Define $\KL$ to be the \textbf{formal deduction system} consisting of the following alphabet, formulae, axioms and deduction rules.
    \begin{enumerate}
        \item \textbf{\underline{Alphabet.}} The \textbf{alphabet} of $\KL$ is the alphabet of $\L$ (cf. \Cref{Ch2:Def:First-Order_Language}).
        
        \item \textbf{\underline{Formulae.}} The \textbf{formulae} of $\KL$ are the formulae of $\L$ (cf. \Cref{Ch2:Def:Formula_FO_Logic}).
        
        \item \textbf{\underline{Axioms.}} For any $\L$-formulae $\phi, \psi, \chi$ and $i \in \N$, the \textbf{axioms} of $\KL$ in $\phi, \psi, \chi$ and $x_i$ are the following distinguished $\L$-formulae:
        \begin{enumerate}[label = \normalfont (A\arabic*)]
            \item\label{FO:A1}
            $\displaystyle \parenth{\phi \to \parenth{\psi \to \phi}}$
            \item\label{FO:A2}
            $\displaystyle \parenth{\parenth{\phi \to \parenth{\psi \to \chi}} \to \parenth{\parenth{\phi \to \psi} \to \parenth{\phi \to \chi}}}$
            \item\label{FO:A3}
            $\displaystyle \parenth{\parenth{\parenth{\neg \phi} \to \parenth{\neg \psi}} \to \parenth{\psi \to \chi}}$
        \end{enumerate}
        \begin{enumerate}[label = \normalfont (K\arabic*)]
            \item\label{FO:K1}
            $\displaystyle \parenth{\parenth{\forall x_i}\phiof{x_i} \to \phiof{t}}$
            where $t$ is a term free for $x_i$ in $\phi$ (cf. \sorry) and $\phi$ can have other free variables.
            \item\label{FO:K2}
            $\displaystyle \parenth{\parenth{\forall x_i}\parenth{\phi \to \psi} \to \parenth{\phi \to \parenth{\forall x_i}\psi}}$
            if $x_i$ is not free in $\phi$.
        \end{enumerate}

        \item \textbf{\underline{Deduction Rules.}} For any $\L$-formulae $\phi, \psi$ and $i \in \N$, the \textbf{deduction rules} for $\KL$ in $\phi, \psi, x_i$ are
        \begin{enumerate}[label = \normalfont (MP)]
            \item\label{FO:MP}
            \underline{Modus Ponens:} From $\phi$ and $\parenth{\phi \to \psi}$, deduce $\psi$.
        \end{enumerate}
        \begin{enumerate}[label = \normalfont (Gen)]
            \item\label{FO:Gen}
            \underline{Generalisation:} From $\phi$, deduce $\parenth{\forall x_i}\phi$.
        \end{enumerate}
    \end{enumerate}
\end{boxdefinition}

We define deductions identically to how we did for arbitrary formal deduction systems.

\begin{boxdefinition}[Deductions in $\KL$]\label{Ch2:Def:Deductions_in_KL}
    A \textbf{deduction} in $\KL$ is a finite sequence of $\KL$-formulae such that each either is one of the following:
    \begin{itemize}
        \item an axiom of $\KL$.
        \item a formula in $\Sigma$.
        \item deduced from a previous formula in the sequence via the deduction rule~\ref{FO:MP}.
        \item deduced from a previous formula in the sequence via the deduction rule~\ref{FO:Gen} with the restriction that when \ref{FO:Gen} is applied to deduce $\parenth{\forall x_i}\phi$ from $\phi$, $x_i$ does not appear as a free variable in any formula of $\Sigma$ that is used in the deduction of $\phi$.
    \end{itemize}
     We write
    \begin{align*}
        \Sigma \vkl \psi
    \end{align*}
    to mean that a $\KL$-formula $\psi$ occurs at the end of a deduction from $\Sigma$.
\end{boxdefinition}

We can similarly define proofs.

\begin{boxdefinition}[Proofs in $\KL$]
    A \textbf{proof} in $\KL$ is a finite sequence of deductions made from the empty set.
\end{boxdefinition}

Theorems are the results deduced in proofs.

\begin{boxdefinition}[Theorems in $\KL$]
    A \textbf{theorem} of $\KL$ is a $\KL$-formula that occurs at the end of a proof. We write
    \begin{align*}
        \vkl \psi
    \end{align*}
    to mean that $\psi$ is a theorem of $\KL$.
\end{boxdefinition}

There is a very deep reason why we restrict the use of \ref{FO:Gen} to deduce $\parenth{\forall x_i}\phi$ from $\phi$ in \Cref{Ch2:Def:Deductions_in_KL} and the subsequent definitions of proofs and theorems in $\KL$. Unfortunately, we are not yet in a position to discuss it. But we will in due course: see \Cref{Ch2:CEg:Gen_Restriction}.

\subsection{Tools for Deduction}

\subsection{Soundness}

We are now in a position to comment on the restriction we impose on the use of the deduction rule~\ref{FO:Gen} in \Cref{Ch2:Def:Deductions_in_KL} and the subsequent definitions of proofs and theorems in $\KL$. We want our formal system to be \textbf{sound}, but it turns out that if we did not have this restriction, we would be able to construct a counterexample disproving soundness.
\begin{boxcexample}[Disproving soundness when we allow unrestricted use of \ref{FO:Gen}]\label{Ch2:CEg:Gen_Restriction}
    If we allowed \ref{FO:Gen} to deduce $\parenth{\forall x_i}\phi$ from formulae in which $x_i$ occurs freely, in a language $\L$ with a unary relation $R$, we would have
    \begin{align*}
        \set{\Rof{x_1}} \vkl \parenth{\forall x_1}\Rof{x_1}
    \end{align*}
    In structures, such a statement makes no sense, as it says that we can deduce a fact about all constants from a fact about a single constant. Indeed, by the deduction theorem, if $\set{\Rof{x_i=1}} \vkl \parenth{\forall x_1}\Rof{x_1}$, then
    \begin{align*}
        \vkl \parenth{\Rof{x_1} \to \parenth{\forall x_1}\Rof{x_1}}
    \end{align*}
    However, in an $\L$-structure like $\A = \cycl{\N; \parenth{= 0}}$, with domain $\N$ and unary relation being the condition that an element be equal to $0$, we have
    \begin{align*}
        \A \not\models \parenth{\Rof{x_1} \to \parenth{\forall x_1}\Rof{x_1}}
    \end{align*}
    because if you take the valuation $\v$ that maps $x_1$ to $0 \in \N$, we have that
    \begin{align*}
        \vof{\parenth{\Rof{x_1} \to \parenth{\forall x_1}\Rof{x_1}}} = \F
    \end{align*}
    because indeed, \sorry
\end{boxcexample}

\sorry
\subsection{Consistency}

% Up to Prop. 2.5.2 in David Ang's notes

\subsection{Completeness}

\subsection{Compactness}
