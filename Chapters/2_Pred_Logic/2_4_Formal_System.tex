\section{A Formal System for First-Order Logic}

The purpose of this section is to define, for any first-order language $\L$, a \textbf{formal deduction system $\KL$} in which we can do \textbf{first-order logic}.

Throughout this section, fix a first-order language $\L$.

\subsection{The Formal Deduction System $\KL$}

Recall \Cref{Ch1:Def:Formal_Deduction_System}, in which we define a \textbf{formal deduction system}. Informally, this is meant to be a system in which we can perform \textbf{deductions}. The purpose of this subsection is to define a \textbf{formal deduction system for $\L$}.

\begin{boxdefinition}[A Formal Deduction System for First-Order Logic]
    Define $\KL$ to be the \textbf{formal deduction system} consisting of the following alphabet, formulae, axioms and deduction rules.
    \begin{enumerate}
        \item \textbf{\underline{Alphabet.}} The \textbf{alphabet} of $\KL$ is the alphabet of $\L$ (cf. \Cref{Ch2:Def:First-Order_Language}).
        
        \item \textbf{\underline{Formulae.}} The \textbf{formulae} of $\KL$ are the formulae of $\L$ (cf. \Cref{Ch2:Def:Formula_FO_Logic}).
        
        \item \textbf{\underline{Axioms.}} For any $\L$-formulae $\phi, \psi, \chi$ and $i \in \N$, the \textbf{axioms} of $\KL$ in $\phi, \psi, \chi$ and $x_i$ are the following distinguished $\L$-formulae:
        \begin{enumerate}[label = \normalfont (A\arabic*)]
            \item\label{FO:A1}
            $\displaystyle \parenth{\phi \to \parenth{\psi \to \phi}}$
            \item\label{FO:A2}
            $\displaystyle \parenth{\parenth{\phi \to \parenth{\psi \to \chi}} \to \parenth{\parenth{\phi \to \psi} \to \parenth{\phi \to \chi}}}$
            \item\label{FO:A3}
            $\displaystyle \parenth{\parenth{\parenth{\neg \phi} \to \parenth{\neg \psi}} \to \parenth{\psi \to \chi}}$
        \end{enumerate}
        \begin{enumerate}[label = \normalfont (K\arabic*)]
            \item\label{FO:K1}
            $\displaystyle \parenth{\parenth{\forall x_i}\phiof{x_i} \to \phiof{t}}$
            where $t$ is a term free for $x_i$ in $\phi$ (cf. \sorry) and $\phi$ can have other free variables.
            \item\label{FO:K2}
            $\displaystyle \parenth{\parenth{\forall x_i}\parenth{\phi \to \psi} \to \parenth{\phi \to \parenth{\forall x_i}\psi}}$
            if $x_i$ is not free in $\phi$.
        \end{enumerate}

        \item \textbf{\underline{Deduction Rules.}} For any $\L$-formulae $\phi, \psi$ and $i \in \N$, the \textbf{deduction rules} for $\KL$ in $\phi, \psi, x_i$ are
        \begin{enumerate}[label = \normalfont (MP)]
            \item\label{FO:MP}
            \underline{Modus Ponens:} From $\phi$ and $\parenth{\phi \to \psi}$, deduce $\psi$.
        \end{enumerate}
        \begin{enumerate}[label = \normalfont (Gen)]
            \item\label{FO:Gen}
            \underline{Generalisation:} From $\phi$, deduce $\parenth{\forall x_i}\phi$.
        \end{enumerate}
    \end{enumerate}
\end{boxdefinition}

We define deductions identically to how we did for arbitrary formal deduction systems.

\begin{boxdefinition}[Deductions in $\KL$]\label{Ch2:Def:Deductions_in_KL}
    A \textbf{deduction} in $\KL$ is a finite sequence of $\KL$-formulae such that each either is one of the following:
    \begin{itemize}
        \item an axiom of $\KL$.
        \item a formula in $\Sigma$.
        \item deduced from a previous formula in the sequence via the deduction rule~\ref{FO:MP}.
        \item deduced from a previous formula in the sequence via the deduction rule~\ref{FO:Gen} with the restriction that when \ref{FO:Gen} is applied to deduce $\parenth{\forall x_i}\phi$ from $\phi$, $x_i$ does not appear as a free variable in any formula of $\Sigma$ that is used in the deduction of $\phi$.
    \end{itemize}
     We write
    \begin{align*}
        \Sigma \vkl \psi
    \end{align*}
    to mean that a $\KL$-formula $\psi$ occurs at the end of a deduction from $\Sigma$.
\end{boxdefinition}

We can similarly define proofs.

\begin{boxdefinition}[Proofs in $\KL$]
    A \textbf{proof} in $\KL$ is a finite sequence of deductions made from the empty set.
\end{boxdefinition}

Theorems are the results deduced in proofs.

\begin{boxdefinition}[Theorems in $\KL$]
    A \textbf{theorem} of $\KL$ is a $\KL$-formula that occurs at the end of a proof. We write
    \begin{align*}
        \vkl \psi
    \end{align*}
    to mean that $\psi$ is a theorem of $\KL$.
\end{boxdefinition}

There is a very deep reason why we restrict the use of \ref{FO:Gen} to deduce $\parenth{\forall x_i}\phi$ from $\phi$ in \Cref{Ch2:Def:Deductions_in_KL} and the subsequent definitions of proofs and theorems in $\KL$. Unfortunately, we are not yet in a position to discuss it. But we will in due course: see \Cref{Ch2:CEg:Gen_Restriction}.

\subsection{Tools for Deduction}

An important objective of our discussion on the formal system $\KL$ is to show that the theorems of $\KL$ are precisely the logically valid formulae. In \Cref{Ch2:Sec:Prop_Logic}, we established a bridge between Propositional and First-Order Logic. In particular, we showed that the theorems in the formal system $\bL$---that is, the tautologies---are all logically valid (see \Cref{Ch2:Thm:Tauto_Logic_Valid}). We can actually show something stronger.

\begin{boxtheorem}\label{Ch2:Thm:L_Tauto_Thm_FO}
    Let $\phi$ be an $\KL$-formula that is a substitution instance of a tautology in propositional logic. Then, $\vkl \phi$.
\end{boxtheorem}
\begin{proof}
    \sorry % Apply completeness of prop logic to turn into a proof in L. Then perform same substitution as phi in all steps.
\end{proof}

While the above gives us a way of dealing with formulae that come from first-order logic, we also have a way of getting access to the first of two formulae linked by a $\to$ connective.

\begin{boxtheorem}[Deduction Theorem for $\KL$]\label{Ch2:Thm:Deduction}
    Let $\Sigma$ be a set of $\L$-formulae and let $\phi$ and $\psi$ be $\L$-formulae. If $\Sigma \cup \set{\phi} \vkl \psi$, then $\Sigma \vkl \parenth{\phi \to \psi}$.
\end{boxtheorem}
\begin{proof}
    \sorry % Follows the proof of the deduction theorem for prop logic. Goes by induction on length of deduction.
\end{proof}



\subsection{Soundness}

In this subsection, we show one direction of our claim that the theorems of $\KL$ are precisely the logically valid formulae (cf. \Cref{Ch2:Def:Logical_Validity}).

\begin{boxtheorem}[Soundness Theorem for $\KL$]\label{Ch2:Thm:KLSoundness}
    Let $\phi$ be a $\KL$-formula. If $\vkl \phi$, then $\models \phi$. In other words, if $\phi$ is a theorem of $\KL$, then $\phi$ is logically valid.
\end{boxtheorem}
\begin{proof}
    As with the proof of the Deduction Theorem (\Cref{Ch2:Thm:Deduction}), we follow the proof of the Soundness Theorem for $\bL$ (\Cref{Ch1:Thm:LSoundness}). Our strategy for showing that all formulae deduced from the empty set are logically valid will be to prove the following.
    \begin{enumerate}
        \item \underline{The axioms of $\KL$ are logically valid.}

        We do not need to worry about the axioms~\ref{FO:A1}, \ref{FO:A2} and \ref{FO:A3}, because they are substitution instances of propositional tautologies, making them logically valid by \Cref{Ch2:Thm:Tauto_Logic_Valid}. Furthermore, the axiom~\ref{FO:K1} is logically valid by \sorry % Thm 2.3.5 in David Ang's notes - we haven't got it in our notes yet!
        Finally, the axiom~\ref{FO:K2} is logically valid by \sorry % Finish from David Ang's notes.

        \item \underline{Deductions preserve logical validity.}

        \sorry % Left as an exercise in David Ang's notes. We need to think about how to do it.
    \end{enumerate}
\end{proof}

We have an important corollary that mirrors \sorry % Thing from prop logic that was in coursework 1 - we really gotta complete that part of the notes! Aarghhhh

\begin{boxcorollary}
    Let $\Sigma$ be a set of $\KL$-formulae and let $\psi$ be a $\KL$-formula. Suppose that
    \begin{align*}
        \Sigma \vkl \psi
    \end{align*}
    Then, for any $\L$-structure $\A$ and valuation $\v$ of $\L$ in $\A$, if $\vbrac{\sigma} = \T$ for every $\sigma \in \Sigma$, then $\vbrac{\phi} = \T$.
\end{boxcorollary}
\begin{proof}
    \sorry % Left as an exercise in David Ang's notes.
\end{proof}

We are now in a position to comment on the restriction we impose on the use of the deduction rule~\ref{FO:Gen} in \Cref{Ch2:Def:Deductions_in_KL} and the subsequent definitions of proofs and theorems in $\KL$. We want our formal system to be \textbf{sound}, but it turns out that if we did not have this restriction, we would be able to construct a counterexample disproving soundness.
\begin{boxcexample}[Disproving soundness when we allow unrestricted use of \ref{FO:Gen}]\label{Ch2:CEg:Gen_Restriction}
    If we allowed \ref{FO:Gen} to deduce $\parenth{\forall x_i}\phi$ from formulae in which $x_i$ occurs freely, in a language $\L$ with a unary relation $R$, we would have
    \begin{align*}
        \set{\Rof{x_1}} \vkl \parenth{\forall x_1}\Rof{x_1}
    \end{align*}
    In structures, such a statement would make no sense, as it would say that we can deduce a fact about all constants from a fact about a single constant. Indeed, by the Deduction Theorem (\Cref{Ch2:Thm:Deduction}), if $\set{\Rof{x_1}} \vkl \parenth{\forall x_1}\Rof{x_1}$, then we would have
    \begin{align*}
        \vkl \parenth{\Rof{x_1} \to \parenth{\forall x_1}\Rof{x_1}}
    \end{align*}
    However, in an $\L$-structure like $\A = \cycl{\N; \parenth{= 0}}$, with domain $\N$ and unary relation being the condition that an element be equal to $0$, we have
    \begin{align*}
        \A \not\models \parenth{\Rof{x_1} \to \parenth{\forall x_1}\Rof{x_1}}
    \end{align*}
    because if you take the valuation $\v$ that maps $x_1$ to $0 \in \N$, we have that
    \begin{align*}
        \vof{\parenth{\Rof{x_1} \to \parenth{\forall x_1}\Rof{x_1}}} = \F
    \end{align*}
    because indeed, it is the case that $x_1 = 0$ and $\vbrac{\parenth{\forall x_1}\Rof{x_1}} = \F$ for any valuation that maps $x_1$ to $0$. In particular, we would have that
    \begin{align*}
        \not\models \parenth{\Rof{x_1} \to \parenth{\forall x_1}\Rof{x_1}}
    \end{align*}
    In other words, despite having the deduction
    \begin{align*}
        \vkl \parenth{\Rof{x_1} \to \parenth{\forall x_1}\Rof{x_1}}
    \end{align*}
    the formula
    \begin{align*}
        \parenth{\Rof{x_1} \to \parenth{\forall x_1}\Rof{x_1}}
    \end{align*}
    would not be a valid theorem in $\L$. This is a direct contradiction of the Soundness Theorem (\Cref{Ch2:Thm:KLSoundness}). Therefore, the restriction on the use of \ref{FO:Gen} \textbf{is necessary} for soundness.
\end{boxcexample}

Apart from offering an explanation for the restriction on the use of the deduction rule~\ref{FO:Gen} in first-order logic, the Soundness Theorem gives us more properties we can use to make deductions. It brings us an important step closer to proving that the theorems of first-order logic are precisely the logically valid formulae.

\subsection{Consistency}

Again, there is some overlap with propositional logic.

\begin{boxdefinition}[Consistency]
    A set $\Sigma$ of $\KL$-formulae is \textbf{consistent} if there is no formula $\phi$ such that
    \begin{align*}
        \Sigma \vkl \phi
        \qquad \text{ and } \qquad
        \Sigma \vkl \parenth{\neg\phi}
    \end{align*}
\end{boxdefinition}

The Soundness Theorem allows us to prove an important result.

\begin{boxtheorem}[The Consistency Theorem]
    The empty set $\emptyset$ of $\KL$-formulae is consistent.
\end{boxtheorem}
\begin{proof}
    \sorry
\end{proof}

\begin{boxconvention}
    We say \textbf{$\KL$ is consistent}.
\end{boxconvention}

For the remainder of this subsection, fix a set $\Sigma$ of $\KL$-formulae. We mention an analogue of the intuitive idea that `a proof of False yields anything'.

\begin{boxproposition}
    If $\Sigma$ is inconsistent, then for all $\KL$-formulae $\chi$, we have
    \begin{align*}
        \Sigma \vkl \chi
    \end{align*}
\end{boxproposition}
\begin{proof}
    \sorry % See Remark after Def 2.5.1 in David Ang's notes.
\end{proof}

We state a few more consistency results that are analogous to propositional logic.\todo{Give cross references!}

\begin{boxproposition}
    Suppose $\Sigma$ is consistent and consists entirely of closed $\KL$formulae. Let $\phi$ be a closed $\KL$-formula. If $\Sigma \not\vkl \phi$, then $\Sigma \cup \set{\parenth{\neg \phi}}$ is consistent.
\end{boxproposition}
\begin{proof}
    \sorry % Analogous to propositional logic.
\end{proof}

\begin{boxproposition}[Analogue of Lindenbaum's Lemma]\label{Ch2:Prop:Lindenbaum_FO}
    Suppose $\Sigma$ consists entirely of closed $\KL$-formulae. There is a consistent set $\Sigma^*$ of closed $\KL$-formulae such that $\Sigma \subseteq \Sigma^*$.
\end{boxproposition}
\begin{proof}
    \sorry % Use an enumeration again
\end{proof}

The above results have important consequences that we will discuss in the next section. These will be instrumental in proving a converse result for \Cref{Ch2:Thm:KLSoundness}.

% Up to Prop. 2.5.2 in David Ang's notes

\subsection{Model Existence}

In this subsection, we state and sketch the proof of an important result that will serve as a stepping-stone towards proving converse of the Soundness Theorem for $\KL$, known as the Completeness Theorem.

First, we introduce some notation.

\begin{boxconvention}
    Let $\A$ be an $\L$-structure and $\Sigma$ a set of $\L$-formulae. We write
    \begin{align*}
        \A \models \Sigma
    \end{align*}
    if for all $\sigma \in \Sigma$, $\A \models \sigma$.
\end{boxconvention}

We have a rather surprising existence result for a model for any consistent set of closed formulae.

\begin{boxtheorem}[Model Existence Theorem]\label{Ch2:Thm:Model_Existence}
    Suppose $\L$ is a \textit{countable} \fola\ and $\Sigma$ is a consistent set of $\L$-formulae. Then, there is a countable $\L$-structure $\A$ such that $\A \models \Sigma$.
\end{boxtheorem}
\begin{proof}[Proof Sketch]
    This is quite a lengthy proof, so we do not give more than a sketch here. The details can be found in~\cite[Appendix A.2]{LecNotes2018}. There are several steps.

    \begin{enumerate}[label = \underline{\textbf{Step \arabic*.}}]
        \item \textbf{Extending our Language.}
        
        We define a new language $\L^+$ that is an extension of $\L$ with countably many constant symbols. That is, the constants in $\L^+$ are the constants in $\L$ together with countably many new constants $b_0, b_1, b_2, \ldots$. Note that $\L^+$ is also countable because $\L$ is, and we only add countably many new constants.
        
        \item \textbf{Adding Witnesses.}
        
        We can regard $\Sigma$ as a set of closed $\L^+$-formulae. It is possible to prove that $\Sigma$ is still consistent as a set of $\L^+$-formulae. In fact, we can extend $\Sigma$ to a set $\Sigma_{\infty} \supseteq \Sigma$ such that
        \begin{itemize}
            \item every formula in $\Sigma_{\infty}$ is closed
            \item for every $\L^+$-formula $\theta\of{x_i}$ with one free variable $x_i$, there exists some constant $b_j$ in $\L^+$ such that
            \begin{align*}
                \Sigma_{\infty} \vdash_{\K_{\L^+}}
                \parenth{
                    \parenth{
                        \neg\parenth{\forall x_i}\theta\of{x_i}
                    }
                    \to
                    \parenth{
                        \neg \theta\of{b_j}
                    }
                }
            \end{align*}
            This formula essentially says
            \begin{align*}
                \Sigma_{\infty} \vdash_{\K_{\L^+}}
                \parenth{
                    \parenth{
                        \parenth{\exists x_i}\parenth{\neg\theta\of{x_i}}
                    }
                    \to
                    \parenth{
                        \neg \theta\of{b_j}
                    }
                }
            \end{align*}
            and we say ``$b_j$ \textbf{witnesses} the existence of $x_i$ satisfying $\parenth{\neg\theta\of{x_i}}$''.
        \end{itemize}
        We say that this process of constructing $\Sigma_{\infty}$ ``\textbf{adds witnesses}.''

        \item \textbf{Constructing a Complete Set of Formulae.}
        
        We can now apply Lindenbaum's Lemma (\Cref{Ch2:Prop:Lindenbaum_FO}) to $\Sigma_{\infty}$: there exists a consistent set $\Sigma^*$ of closed $\L^+$-formulae such that for every closed $L^+$-formula $\phi$, we have either $\Sigma^* \vdash_{\K_{\L^+}} \phi$ or $\Sigma^* \vdash_{\K_{\L^+}} \parenth{\neg \phi}$. We will not use $\Sigma^*$ right away, but it will soon become clear what the relevance of this set is.
        
        \item \textbf{Constructing an $\L^+$-structure.}
        
        \sorry
        
        \item\label{Ch2:Thm:Model_Existence:Step:Result_on_L_plus} \textbf{A Result on $L^+$.}
        
        \sorry

        \item \textbf{Restricting \ref{Ch2:Thm:Model_Existence:Step:Result_on_L_plus} to $\L$.}
        
        \sorry
    \end{enumerate}
\end{proof}

What's more surprising than the actual existence result is the accompanying countability result: the model we constructed above is countable!

We are now ready for the converse of the Soundness Theorem.

\subsection{Completeness}

Our strategy will be to first prove the result for closed formulae and then prove it in general.

We begin by introducing some notation.

\begin{boxconvention}
    Let $\Sigma$ a set of $\L$-formulae and $\phi$ an $\L$-formula. We write
    \begin{align*}
        \Sigma \models \phi
    \end{align*}
    if every model of $\Sigma$ is a model of $\phi$. That is, $\Sigma \models \phi$ if for every $\L$-structure $\A$, it holds that if $\A \models \Sigma$, then $\A \models \phi$.
\end{boxconvention}

We are now in a position to prove the converse of the Soundness Theorem for $\KL$ for closed formulae. It turns out to be a consequence of the following important theorem.

\begin{boxtheorem}\label{Ch2:Thm:Generalised_KLCompleteness_Closed_Formulae}
    Let $\Sigma$ be a set of closed $\L$-formulae and $\phi$ a closed $\L$-formula. If $\Sigma \models \phi$, then $\Sigma \vkl \phi$.
\end{boxtheorem}
\begin{proof}
    \sorry
\end{proof}

The case when $\Sigma = \emptyset$ gives us a converse of the Soundness Theorem for $\KL$ for closed formulae.

\begin{boxcorollary}[A Partial Converse of Soundness]\label{Ch2:Cor:KLCompleteness_Closed_Formulae}
    Let $\phi$ be a closed $\L$-formula. If $\models \phi$, then $\vkl \phi$.
\end{boxcorollary}
\begin{proof}
    When $\Sigma = \emptyset$, we do, indeed have that $\Sigma \models \phi$ is the same as $\models \phi$: for any $\L$-structure $\A$, since $\A$ vacuously models every element of $\Sigma$, the condition that if $\A \models \sigma$ for all $\sigma \in \Sigma$ then $\A \models \phi$ is precisely the condition that $\A \models \phi$. \Cref{Ch2:Thm:Generalised_KLCompleteness_Closed_Formulae} then gives us that $\vkl \phi$.
\end{proof}

It turns out that \Cref{Ch2:Cor:KLCompleteness_Closed_Formulae}, a converse of the Soundness Theorem for closed formulae, gives us everything we need to prove the result in general.

\begin{boxtheorem}[Gödel's Completeness Theorem for $\KL$]\label{Ch2:Thm:Godel_Completeness}
    Let $\phi$ be any $\L$-formula. If $\phi$ is logically valid, then $\phi$ is a theorem of $\L$. That is, if $\models \phi$, then $\vkl \phi$.
\end{boxtheorem}
\begin{proof}
    By \Cref{Ch2:Cor:KLCompleteness_Closed_Formulae}, it suffices to consider the case where $\phi$ is not closed, that is, where $\phi$ has free variables.
    \sorry
\end{proof}

\subsection{Compactness}
