\section{A Bridge between Propositional and First-Order Logic}

The purpose of this section is to discuss how ideas in propositional logic can be expressed in first-order logic. As we might expect from \Cref{Ch2:Def:First-Order_Language}, first-order logic is a generalisation of propositional logic. We will use this section to make this precise.

We will begin by making rigorous the notion of \textbf{logical validity}. We will see a connection with tautologies, which, as we know, are precisely the theorems of propositional logic. We will explore notions like satisfaction and substitution to make this precise.

\subsection{Valuations Satisfying Formulae}

For the remainder of this subection, fix some \fola\ $\L$ and some $\L$-structure $\A$. We begin by defining a notion of equivalence for valuations that is weaker than equality.

\begin{boxdefinition}[Equivalence with respect to a Variable]
    Suppose $x_l$ is a variable in $\L$ and valuations $\v, \w$ of $\L$ in $\A$. We say that $\v$ and $\w$ are \textbf{$x_l$-equivalent} if for all $i \in \N \setminus \set{l}$, we have $\vof{x_i} = \wof{x_i}$.
\end{boxdefinition}

One way to see that this notion is weaker than equality of valuations is that if valuations agree on all variables, then by \Cref{Ch2:Lemma:Valuation_Existence_Uniqueness}, they must be equal. The fact that there is one variable on which they \textit{may} differ is what makes this notion weaker. We also emphasise that we only stipulate that $x_l$-equivalent valuations \textit{need not} agree on $x_l$, not that they \textit{must not} agree on $x_l$. In particular, \textit{equal valuations are equivalent with respect to all variables}.

We are now ready to define precisely what it means for a valuation to satisfy a formula. The idea is that in a general setting---ie, in a \fola---formulae do not express any properties, and without an interpretation in a \fos, it is not very meaningful to talk about what it means for a formula to be `true'. A valuation translates formulae into `meaningful statements', and therefore, informally, we can think of a valuation satisfying a formula as meaning that it translates a general formula into some interpretation that we know to be `true' in some sense.

\begin{boxdefinition}[Satisfaction]
    Let $\phi$ be an $\L$-formula and $\v$ a valuation of $\L$ in $\A$. We can define what it means for \textbf{$\v$ satisfies $\phi$ in $\A$} recursively on the number of connectives or quantifiers in $\phi$.
    \begin{enumerate}
        \item If $\phi$ is an \underline{atomic formula}, ie, has no connectives or quantifiers, then we know $\phi$ is of the form $\Rof{t_1, \ldots, t_n}$ for some $n$-ary relation $R$ and terms $t_1, \ldots, t_n$ in $\L$. In this case, we say that \textbf{$\v$ satisfies $\phi$ in $\A$} if the relation
        \begin{align*}
            \Rbof{\vof{t_1}, \ldots, \vof{t_n}}
        \end{align*}
        holds in $\A$, ie, if $\parenth{\vof{t_1}, \ldots, \vof{t_n}} \in \Rb \subseteq A^n$.

        \item If $\phi$ is \underline{not an atomic formula} in $\L$, then we know $\phi$ is of one of the following forms.
        \begin{enumerate}
            \item If $\phi$ is of the form $\parenth{\neg \psi}$ for some $\L$-formula $\psi$, then we say that \textbf{$\v$ satisfies $\phi$ in $\A$} if $\v$ does not satisfy $\psi$ in $\A$.
            
            \item If $\phi$ is of the form $\parenth{\psi \to \chi}$ for $\L$-formulae $\psi$ and $\chi$, then we say that \textbf{$\v$ satisfies $\phi$ in $\A$} if it is not the case that $\v$ satisfies $\psi$ in $\A$ and $\v$ does not satisfy $\chi$ in $\A$.
            
            \item If $\phi$ is of the form $\parenth{\forall x}\psi$ for some $\L$-formula $\psi$, we say \textbf{$\v$ satisfies $\phi$ in $\A$} if for all variables $x$, if $\w$ is a valuation of $\L$ in $\A$ that is $x$-equivalent to $\v$, then $\w$ satisfies $\psi$ in $\A$.
        \end{enumerate}
        In all three cases, we have define what it means for $\v$ to satisfy $\phi$ in terms of what it means for $\v$ to satisfy formulae with one fewer connectives or quantifiers than $\phi$, making the recursion well-defined.
    \end{enumerate}
    If $\v$ satisfies $\phi$ in $\A$, we write $\vbrac{\phi} = \T$, and if $\v$ does not satisfy $\phi$ in $\A$, we write $\vbrac{\phi} = \F$.
\end{boxdefinition}

We remark that the notation $\vbrac{\phi} = \T$ is merely shorthand for a statement of fact. It does not actually represent an equality in some setting where both $\T$ and $\vbrac{\phi}$ are defined. The same goes for $\vbrac{\phi} = \F$. However, it is often convenient to abuse notation.

\begin{boxconvention}
    Given valuations $\v, \w$ of $\L$ in $\A$, we write
    \begin{align*}
        \vbrac{\phi} = \wbrac{\phi}
    \end{align*}
    to denote the condition that
    \begin{align*}
        \vbrac{\phi} = \T \tiff \wbrac{\phi} = \T
    \end{align*}
    or the equivalent condition that
    \begin{align*}
        \vbrac{\phi} = \F \tiff \wbrac{\phi} = \F
    \end{align*}
\end{boxconvention}

This notion of a valuation satisfying a formula tells us what it means for a formula to hold true in a structure for a \textit{given} assignment of meaning to the first-order language symbols in the formula. We can define a more absolute notion of truth.

\begin{boxdefinition}[Truth of a First-Order Formula in a First-Order Structure]
    Let $\phi$ be an $\L$-formula. We say \textbf{$\phi$ is true in $\A$}, or that \textbf{$\A$ is a model of $\phi$}, if $\vbrac{\phi} = \T$ for all valuations $\v$ of $\L$ in $\A$. We denote this by $\A \models \phi$.
\end{boxdefinition}

A first-order formula true in one structure might not be true in another.

\begin{boxexample}
    Let $\L$ have one binary relation. Consider the atomic formula
    \begin{align}
        \parenth{\forall x_1}\parenth{\exists x_2} \Rof{x_1, x_2}
        \label{Ch2:Eq:Forall_Exists_Rel}
    \end{align}
    This formula is true in structures like $\cycl{\N; <}$, $\cycl{\Z, <}$, and $\cycl{\Z, >}$.
\end{boxexample}
\begin{boxnexample}
    Let $\L$ be as above. The formula~\eqref{Ch2:Eq:Forall_Exists_Rel} above is NOT true in $\cycl{\N; >}$, because $0$ does not have a predecessor in $\N$. The reason for this is that relation inputs are \textit{ordered} (ie, relations are not, in general, symmetric).
\end{boxnexample}

Despite being valuation-indepentent, given that the truth of formulae is structure-dependent, it is still not the strongest notion of truth we can define in first-order logic. We want theorems to be stronger and more `absolute' notions of truth. To that end, we define logical validity, a \textit{structure-independent notion of truth}.

\begin{boxdefinition}[Logical Validity]
    Let $\phi$ be an $\L$-formula. We say that $\phi$ is logically valid if $\A \models \phi$ for all $\L$-structures $\A$.
\end{boxdefinition}

Logically valid formulae in first-order logic are meant to be analogues of tautologies in propositional logic. An important difference, though, is that in propositional logic, there is an algorithm that \textit{decides} whether a given formula is a tautology. In first-order logic, this is not usually the case, as we shall see when we study Gödel's Incompleteness Theorem.

\begin{boxexample}
    For all $\L$-formulae $\phi$, the formula
    \begin{align*}
        \parenth{\parenth{\exists x_1}\parenth{\forall x_2}\phi \to \parenth{\forall x_2}\parenth{\exists x_1}\phi}
    \end{align*}
    is logically valid.
    \begin{proof}
        \sorry
    \end{proof}
\end{boxexample}

\begin{boxnexample}
    Given an $\L$-formulae $\phi$, the formula
    \begin{align*}
        \parenth{\parenth{\forall x_1}\parenth{\exists x_2}\phi \to \parenth{\exists x_2}\parenth{\forall x_1}\phi}
    \end{align*}
    is not necessarily logically valid.
    \begin{proof}
        \sorry
    \end{proof}
\end{boxnexample}

\subsection{Substitution}

In this section, we will discuss how replacing propositional variables in propositional formulae with first-order formulae gives a bridge between the two kinds of logic. For the remainder of this subection, fix
\begin{itemize}
    \item a natural number $n$
    \item a \fola\ $\L$
    \item $\L$-formulae $\phi_1, \ldots, \phi_n$
    \item propositional variables $p_1, \ldots, p_n$
    \item a propositional formula $\chi$ with propositional variables $p_1, \ldots, p_n$ (cf. \Cref{Ch1:Def:PropFormula})
\end{itemize}

We have a name for replacing the $p_i$ $\chi$ with $\phi_i$.

\begin{boxdefinition}[Substitution Instance]
    The \textbf{substitution instance of $\chi$ with $\phi_1, \ldots, \phi_n$} is the $\L$-formula obtained by replacing each propositional variable $p_i$ in $\chi$ with the $\L$-formula $\phi_i$.
\end{boxdefinition}
\begin{remark}
    We underscore here that a substitution instance is a formula in $\L$, not a propositional formula, because the terms of which it is made are all $\L$-formulae rather than propositional variables.
\end{remark}

For the remainder of this subsection, denote by $\theta$ the substitution instance of $\chi$ with $\phi_1, \ldots, \phi_n$.

Substitution is our gateway for `embedding' propositional logic into first-order logic. This `embedding' preserves `truth' in the following manner.

\begin{boxtheorem}
    If $\chi$ is a tautology, then $\theta$ is logically valid.
\end{boxtheorem}
\begin{proof}
    Assume that $\chi$ is a tautology (cf. \Cref{Ch1:Def:Tautology}). To show that $\theta$ is logically valid, we need to show that $\A \models \theta$ for all $\L$-structures $\A$. To that end, \letal\ and \vola. We need to show that $\vbrac{\theta} = \T$. \sorry
\end{proof}

It is tempting to ask whether all logically valid formulae are substitution instances of tautologies. Such a fact would truly make the above result a two-way bridge. Unfortunately, this is not true, as shown by the following counterexample.

\begin{boxcexample}
    For all $\L$-formulae $\phi$, the $\L$-formula
    \begin{align*}
        \parenth{\parenth{\exists x_2}\parenth{\forall x_1}\phi \to \parenth{\forall x_1}\parenth{\exists x_2}\phi}
    \end{align*}
    is logically valid. However, it is not a substitution instance of any propositional formula, because there is no notion of quantification in propositional logic.
\end{boxcexample}

This tells us that first-order logic is not only stronger than propositional logic, it is \textbf{strictly} stronger.
