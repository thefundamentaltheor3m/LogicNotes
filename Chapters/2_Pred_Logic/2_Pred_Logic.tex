\chapter{First-Order Logic}
\thispagestyle{empty}

The plan for this part of the module is to examine Predicate Logic, or First-Order Logic. We will examine the following, though the distinction between semantics and syntax shall not be as pronounced as it is below:
\begin{enumerate}
    \item \underline{Semantics:}
    \begin{enumerate}[noitemsep]
        \item First-order structures
        \item First-order languages and the corresponding formulae
    \end{enumerate}

    \item \underline{Syntax:}
    \begin{enumerate}[noitemsep]
        \item A formal system for first-order logic
        \item Gödel's Completeness Theorem, which tells us that the theorems of the formal system are the ``logically valid formulae''
    \end{enumerate}
\end{enumerate}

Before we can begin the study of first-order logic, we need to make several important definitions and introduce the notation we will use for the remainder of this chapter. We will begin by studying structures and first-order languages. We will then relate ideas from propositional logic to ideas in first-order logic. Finally, we will build a formal system for first-order logic and give a rigorous syntactic foundation for the ideas we discuss.

\section{Languages, Structures and Interpretations}

In this section, we discuss the notion of first-order languages and their interpretations in first-order structures. While this is primarily a study of semantics, the definition of languages is syntactic in nature. Yet, we consider this section to be a study of semantics because the purpose is to give some sort of meaning to the syntactic expressions we have in first-order languages.

We begin by studying first-order structures in an abstract sense. We then take a syntactic detour into the study of first-order languages, before examining what it means for structures to exist inside (and give interpretations for) first-order languages.

\subsection{First-Order Structures}

We begin by discussing relations and functions of a given arity.

\begin{boxdefinition}[$n$-ary Relation on a Set]
    Suppose $A$ is a set and $n \geq 1$ is a natural number. An \textbf{$n$-ary relation on $A$} is a subset
    \begin{align*}
        \Rb \subseteq \setst{\parenth{a_1, \ldots, a_n}}{a_1, \ldots, a_n \in A}
    \end{align*}
\end{boxdefinition}

We have a similar notion for functions, with the key fact being that $n$-ary functions take in $n$ inputs and return a single output, and all inputs and outputs must come from the set in question.

\begin{boxdefinition}[$n$-ary Function on a Set]
    Given a set $A$, an \textbf{$n$-ary function on $A$} is a function
    \begin{align*}
        \fb: A^n \to A
    \end{align*}
\end{boxdefinition}

We make a subtle distinction between functions and relations in formal and informal language. This is something that will get clearer as we progress.

\begin{boxconvention}
    The reason why we put bars on top of the symbols is to distinguish functions and relations as they appear in formulae from the way that discuss them.
\end{boxconvention}

We have special terms when $n = 1, 2, 3$.

\begin{boxconvention}
    \hfill
    \begin{enumerate}
        \item A $1$-ary relation is commonly called a \textbf{unary relation}.
        \item A $2$-ary relation is commonly called a \textbf{binary relation}.
        \item A $3$-ary relation is commonly called a \textbf{ternary relation}.
    \end{enumerate}
\end{boxconvention}

These notions are not new to us.

\begin{boxexample}[Some Familiar $n$-ary Relations]
    \hfill
    \begin{enumerate}
        \item Equality is a binary relation on any set.
        \item $\leq$ is a binary relation on $\R$.
        \item $\setst{x \in \Z}{x \text{ is even}}$ is a unary relation on $\Z$.
    \end{enumerate}
\end{boxexample}

Admittedly, the fact that the third example is precisely a set is a little unusual to see. This is because in practice, the following convention is used.

\begin{boxconvention}
    Let $\Rb \subseteq A^n$ be a relation on some set $A$. For all $\parenth{a_1, \ldots, a_n} \in A^n$, we write
    \begin{align*}
        \Rbof{a_1, \ldots, a_n}
    \end{align*}
    to mean that $\parenth{a_1, \ldots, a_n} \in \Rb$.
\end{boxconvention}

We are now ready for the most important definition of this chapter.

\begin{boxdefinition}[First-Order Structure]\label{Ch2:Def:First-Order_Structure}
    A \textbf{first-order structure} is the following data:
    \begin{enumerate}
        \item a non-empty set $A$ called the \textbf{domain} of $\A$.
        
        \item a set of \textbf{relations} on $A$
        \begin{align*}
            \setst{\ol{R_i} \subseteq A^{n_i}}{i \in I}
        \end{align*}

        \item a set of \textbf{functions} on $A$
        \begin{align*}
            \setst{\ol{f_j} : A^{m_j} \to A}{j \in J}
        \end{align*}

        \item a set of \textbf{constants} that are elements of $A$
        \begin{align*}
            \setst{\ol{c_k} \in A}{k \in K}
        \end{align*}
    \end{enumerate}
    where $I, J, K$ are index sets that can be empty.
\end{boxdefinition}

Usually, the index sets of a first-order structure are subsets of $\N$, but in principle, they could be any set. We package the information about the constants and the arity of the functions and relations together in the following manner.

\begin{boxdefinition}[Signature]
    Let $\A$ be a first-order structure. The \textbf{signature} of $\A$ is the information
    \begin{align*}
        \setst{n_i}{i \in I}
        \qquad
        \setst{m_j}{j \in J}
        \qquad
        K
    \end{align*}
    with the respective sets describing the arity of the relations on $A$, the arity of the functions on $A$, and the index set of the constants in $A$.
\end{boxdefinition}

We use the following notation for first-order structures.

\begin{boxconvention}
    For a first-order structure $\A$ given as above, we wil deonte
    \begin{align*}
        \A = \cycl{A; \set{\ol{R_i}}_{i \in I}, \set{\ol{f_j}}_{j \in J}, \set{\ol{c_k}}_{k \in K}}
    \end{align*}
    More generally, we use the notation
    \begin{align*}
        \text{Structure} = \cycl{\text{Domain}; \text{Relations}, \text{Functions}, \text{Constants}}
    \end{align*}
\end{boxconvention}

We have encountered any number of first-order structures so far. Here are a few examples.

The first is a very basic example.
\begin{boxexample}[Orderings]
    We can take $A$ to be one of the sets $\N, \Z, \Q, \R$. We can define a first-order structure on $A$ with only one unary relation---that of ordering---and no functions or constants.
\end{boxexample}
It is important to note that while the sets $\N, \Z, \Q, \R$ all admit richer structures on them, they are not needed to define ordering. We don't even include equality in this description because a formula that contains an equality symbol is not about ordering.

In the next example, we look at an algebraic structure.
\begin{boxexample}[Groups]
    Every group is a first-order structure with the following data.
    \begin{enumerate}
        \item The domain is the set of elements of the group.
        \item The sole relation is the binary relation of equality.
        \item There is a binary function for the group operation and a unary function for inversion.
        \item There is a constant for the identity element.
    \end{enumerate}
\end{boxexample}
We can make a similar definition for rings.

We do not even need to talk about objects that we usually deal with as sets. We can also talk about graphs, which, while defined in terms of sets, are usually studied visually.
\begin{boxexample}[Graphs]
    Graphs (or, more precisely, their vertices), along with two binary relations---equality and adjacency---and no functions or constants, form a first-order structure.
\end{boxexample}

\subsection{First-Order Languages}

We are now ready to formally define the notion of first-order languages.

\begin{boxdefinition}[First-Order Language]\label{Ch2:Def:First-Order_Language}
    A \textbf{first-order language} $\L$ consists of the following data.
    \begin{enumerate}
        \item Index sets $I, J, K$ where $I$ is non-empty but $J$ and $K$ can be empty.
        \item An \textbf{alphabet} of symbols, consisting of
        \begin{enumerate}[noitemsep]
            \item \textbf{Variables} $x_0 \ x_1 \ x_2 \ \cdots$
            \item \textbf{Connectives} $\neg \ \to$
            \item \textbf{Punctuation} $( \ ) \ ,$
            \item The \textbf{Quantifier} $\forall$
            \item \textbf{Relation symbols} $R_i$ for $i \in I$
            \item \textbf{Function symbols} $f_j$ for $j \in J$
            \item \textbf{Constant symbols} $c_k$ for $k \in K$
        \end{enumerate}
        \item An \textbf{arity} for every relation and function symbol.
    \end{enumerate}
\end{boxdefinition}

The arity and cardinality information of a first-order language is encoded in the following manner.

\begin{boxdefinition}[Signature]
    Let $\L$ be a first-order language with index sets $I, J, K$ such that $I$ is non-empty and
    \begin{enumerate}
        \item The relations $\setst{R_i}{i \in I}$ have arities $\setst{n_i}{i \in I}$
        \item The functions $\setst{f_j}{j \in J}$ have arities $\setst{m_j}{j \in J}$
        \item The constants are given by $\setst{c_k}{k \in K}$
    \end{enumerate}
    The information
    \begin{align*}
        \setst{n_i}{i \in I}
        \qquad
        \setst{m_j}{j \in J}
        \qquad
        K
    \end{align*}
    is called the \textbf{signature} of $\L$.
\end{boxdefinition}

In principle, one should very precisely define punctuation rules and the general notation for expressing formulae in a first-order language. Indeed, what this means is that one should define what it means for a string of symbols to be `well-formed'. This is somewhat laborious, so we simply adopt the following convention.

\begin{boxconvention}
    We use the punctuation symbols
    \begin{align*}
        ( \quad ) \quad ,
    \end{align*}
    in the following manner.
    \begin{itemize}
        \item We enclose all expressions involving connectives in parentheses, barring those involving a single variable or a single constant.
        \item We enclose statements of the form ``$\forall x$'' in parentheses.
        \item We denote applications functions $f$ by $f(\cdots)$ and relations $R$ by $R(\cdots)$.
        \item We use commas to separate the arguments of functions and relations.
    \end{itemize}
\end{boxconvention}

There are numerous symbols in first-order languages. It makes sense to isolate the ones that form the `objects' with which we `reason' in this language.

\begin{boxdefinition}[Terms]
    Let $\L$ be a first-order language. The set of \textbf{terms} of $\L$ is the smallest set such that
    \begin{enumerate}
        \item Every variable is a term.
        \item Every constant is a term.
        \item If $t_1, \ldots, t_n$ are terms and $f$ is an $n$-ary function symbol, then $\fof{t_1, \ldots, t_n}$ is a term.
    \end{enumerate}
    Moreover, we shall stipulate that every term arises in this manner.
\end{boxdefinition}

We can define very basic first order languages.

\begin{boxexample}\label{Ch2:Eg:Terms_FO_Logic}
    Let $\L$ be a first-order language such that
    \begin{enumerate}
        \item There are no relations
        \item There is a binary function $f$
        \item There are two constants $c_1, c_2$
    \end{enumerate}
    Some terms of $\L$ are
    \begin{align*}
        c_1 \quad c_2 \quad x_1 \quad \fof{c_1, c_2} \quad \fof{x_1, c_1} \quad \fof{x_1, \fof{c_1, c_2}} \quad \fof{\fof{c_1, x_1}, c_2}
    \end{align*}
    There are many other terms. Note that we automatically assumed the existence of variables $x_1, x_2, \ldots$, which is consistent with \Cref{Ch2:Def:First-Order_Language}. 
\end{boxexample}
\begin{boxnexample}
    Let $\L$ be the \fola\ given in \Cref{Ch2:Eg:Terms_FO_Logic}. A string of symbols from the alphabet of $\L$ that is \textbf{not} a term is
    \begin{align*}
        f f x_1
    \end{align*}
    The reason for this is that $x_1$ is not applied to $f$, and even if we were to ignore the punctuation convention of writing function arguments inside parentheses, we would have that the arity of $f$ is violated, because a function of two variables is being applied (twice) to a single input (or the leftmost $f$ is being applied to both $f$ and $x_1$, which contradicts the fact that $f$ takes inputs that are both terms, and $f$ alone is not a term). It is precisely to avoid ambiguities of this sort that we have punctuation conventions; either way, in this case, there are too many errors for $f f x_1$ to be a term in $\L$.
\end{boxnexample}

We now define a way of using the quantifiers and connectives of \fola~s to build \textit{formulae}. The idea is to define a fundamental notion of formulae using purely the relations of the language and then define how more complex formulae can be built from them.

\begin{boxdefinition}[Atomic Formula]
    \letla. An \textbf{atomic formula} of $\L$, or an \textbf{$\L$-atomic formula}, is an expression of the form
    \begin{align*}
        R\!\parenth{t_1, \ldots, t_n}
    \end{align*}
    where $R$ is an $n$-ary relation symbol in $\L$ and $t_1, \ldots, t_n$ are terms.
\end{boxdefinition}

Note that in the above definition, we do not require the inputs $t_1, \ldots, t_n$ to be variables or constants. We merely require them to be \textit{terms}. This means they could be variables, constants, or outputs of functions applied to other terms. Being `atomic' has only to do with being the output of a relation symbol. We can now define what a formula is in a broader sense.

\begin{boxdefinition}[Formula]
    \letla. The \textbf{formulae} of $\L$, or the \textbf{$\L$-formulae}, are defined as follows.
    \begin{enumerate}
        \item Every atomic formula is a formula.
        \item If $\phi$ is a formula, then so is $\parenth{\neg \phi}$.
        \item If $\phi$ and $\psi$ are formulae, then so is $\parenth{\phi \to \psi}$.
        \item If $\phi$ is a formula, then so is $\parenth{\forall x} \phi$.
    \end{enumerate}
    Moreover, we stipulate that every formula arises in this manner.
\end{boxdefinition}

We can define very simple formulae in very simple \fola~s.

\begin{boxexample}
    \letla\ with
    \begin{enumerate}
        \item One unary relation symbol $P$ and one binary relation symbol $R$
        \item One binary function symbol $f$
        \item Two constants $c_1$ and $c_2$
    \end{enumerate}
    Then, the following are all atomic formulae:
    \begin{align*}
        P\!\parenth{x_1} \quad R\!\parenth{c_1, x_1} \quad R\!\parenth{\fof{x_1, c_1}, c_2}
    \end{align*}
    Similarly, the following are all formulae:
    \begin{align*}
        \neg P\!\parenth{x_1} \quad \parenth{P\!\parenth{x_1} \to R\!\parenth{c_1, x_1}} \quad \parenth{\forall x} R\!\parenth{x, c_1}
    \end{align*}
\end{boxexample}

There is a reason why we only allowed \fola~s to have connectives $\to$ and $\neg$ and quantifier $\forall$: we can build the other connectives ($\land, \lor, \lr, \ldots$) and the other quantifier ($\exists$) from these.

\begin{boxdefinition}[The Existential Quantifier]
    \letla\ and let $\phi$ be an $\L$-formula. Then, we define
    \begin{align*}
        \parenth{\exists x} \phi
    \end{align*}
    to be shorthand for the formula
    \begin{align*}
        \parenth{\neg\parenth{\forall x}\parenth{\neg \phi}}
    \end{align*}
\end{boxdefinition}

We also define the other connectives as in propositional logic.

\begin{boxdefinition}[Connectives]
    \letla. Let $\phi$ and $\psi$ be $\L$-formulae. We define the connectives $\land, \lor, \lr, \uparrow, \downarrow$ as follows:
    \begin{align*}
        \parenth{\phi \land \psi} &\text{ is shorthand for } \parenth{\neg (\phi \to \neg \psi)} \\
        \parenth{\phi \lor \psi} &\text{ is shorthand for } \parenth{\parenth{\neg \phi} \to \psi} \\
        \parenth{\phi \lr \psi} &\text{ is shorthand for } \parenth{(\phi \to \psi) \land (\psi \to \phi)} \\
        \parenth{\phi \uparrow \psi} &\text{ is shorthand for } \parenth{\neg (\phi \land \psi)} \\
        \parenth{\phi \downarrow \psi} &\text{ is shorthand for } \parenth{\neg (\phi \lor \psi)}
    \end{align*}
\end{boxdefinition}

We are now ready to explore the utility of first-order logic in the study of mathematics, where we relate the study of first-order structures to first-order languages.

\subsection{First-Order Structures Revisited}

Throughout this subsection, we will fix a first-order language $\L$ with signature
\begin{align*}
    \setst{n_i}{i \in I}
    \qquad
    \setst{m_j}{j \in J}
    \qquad
    K
\end{align*}
where the $n_i$s are the arities of the relation symbols $R_i$, the $m_j$ are the arities of the function symbols $f_j$, and $K$ is the index set of the constants $c_k$.

\begin{boxdefinition}[First-Order Structures in First-Order Languages]\label{Ch2:Def:First-Order_Structure_in_First-Order_Language}
    A \textbf{structure} in $\L$, or an \textbf{$\L$-structure}, is a first-order structure
    \begin{align}
        \A = \cycl{A; \set{\ol{R_i}}_{i \in I}, \set{\ol{f_j}}_{j \in J}, \set{\ol{c_k}}_{k \in K}}
        \label{Ch2:Eq:FOS_Generic}
    \end{align}
    such that the signature of $\A$ is the same as that of $\L$, ie, the arities of the relations and functions in $\A$ are the same as those in $\L$.
\end{boxdefinition}

If one takes a closer look at the definitions of first-order structures and first-order languages (resp. \Cref{Ch2:Def:First-Order_Structure} and \Cref{Ch2:Def:First-Order_Language}), one notices that the former involves \textbf{actual} relations and functions, which are defined as subsets of certain sets, whereas the latter merely involves function and relation \textbf{symbols}. This is a crucial distinction.

\begin{boxconvention}
    The reason why we put bars on top of the symbols is to distinguish functions and relations as they appear in first-order structures---\textit{with} bars on top---from the way we express them in the ambient first-order language.
\end{boxconvention}

For the remainder of this subsection, fix a \fos\ $\A$. Denote its domain, relations, functions and constants as in~\eqref{Ch2:Eq:FOS_Generic}, with bars above as per our convention for this module. We now discuss the significance of first-order structures and underscore the power and versatility of first-order languages.

When we say, in \Cref{Ch2:Def:First-Order_Structure_in_First-Order_Language}, that the signature of an $\L$-structure is the same as that of the first-order language $\L$, we mean that the arities of the relations in the structure must match up with the arities of the relation symbols in the language, and similarly for functions and constants. More precisely, we have the following.

\begin{boxdefinition}[Interpretation]
    \letal. The correspondence
    \begin{align*}
        \ol{R_i} \lrsquig R_i
        \qquad
        \ol{f_j} \lrsquig f_j
        \qquad
        \ol{c_k} \lrsquig c_k
    \end{align*}
    between relations, functions and constants in $\A$ and relation symbols, function symbols and constant symbols in $\L$ is called an \textbf{interpretation of $\L$}.
\end{boxdefinition}

We have a special term for the ``$\L$ to $\A$'' direction of this correspondence.

\begin{boxdefinition}[Valuation]\label{Ch2:Def:Valuation}
    Let $\A$ be an $\L$-structure. A \textbf{valuation} in $\A$ is a function
    \begin{align*}
        \v : \set{\text{Terms of } \L} \to A
    \end{align*}
    that assigns terms of $\L$ to their interpretations in $\A$ in the following manner.
    \begin{enumerate}
        \item For all constant symbols $c_k$ in $\L$, we have
        \begin{align*}
            \vof{c_k} = \ol{c_k}
        \end{align*}
        
        \item For all terms $t_1, \ldots, t_m$ and $m$-ary function symbols $f$ in $\L$, we have
        \begin{align*}
            \vof{\fof{t_1, \ldots, t_m}} = \ol{f}\!\parenth{\vof{t_1}, \ldots, \vof{t_m}}
        \end{align*}
        where $\ol{f}$ is the interpretation of $f$ in $\A$.
    \end{enumerate}
\end{boxdefinition}

The idea is that a first-order language gives a purely symbolic way of expressing relationships between objects and their properties. When studying statements expressed in first-order languages, one must purely view them symbolically, as formal expressions that do not carry any \textit{meaning} per se. The `meaning' comes from valuations that allow us to interpret symbols in first-order languages as ideas in first-order structures.

We have an existence and uniqueness result.
\begin{boxlemma}\label{Ch2:Lemma:Valuation_Existence_Uniqueness}
    \letal\ with domain $A$. Fix elements $a_0, a_1, a_2, \ldots \in A$. There exists a unique valuation $\v$ of $\L$ in $\A$ such that $\vof{x_i} = a_i$ for all $i \in \N$.
\end{boxlemma}
\begin{proof}
    We begin by showing existence. We can define $\v$ explicitly for all terms of $\L$ by performing recursion on their length. For terms of length $1$, we deal with the variables and the distinguished constants separately.
    \begin{enumerate}
        \item $\vof{x_i} := a_i$ for all $i \in \N$
        \item $\vof{c_k} := \ol{c_k}$ for all $k \in K$
    \end{enumerate}
    We can then define $\v$ for terms arising from functions and relations of arity $\geq 2$ by setting
    \begin{align*}
        \vof{\fof{t_1, \ldots, t_m}} &:= \fbof{\vof{t_1}, \ldots, \vof{t_m}}
    \end{align*}
    for all terms $t_1, \ldots, t_m$ and $m$-ary function symbols $f$ in $\L$ with interpretation $\ol{f}$ in $\A$. Such a valuation is unique because if there are two valuations satisfying the condition on the variables, then they agree on \underline{all} terms of length $1$ (because they must agree on all constants---see \Cref{Ch2:Def:Valuation}). The fact that they obey the recursion relation by definition then gives us the result.
\end{proof}

\begin{boxexample}[A Formula in Groups]
    Let $\L$ have the following data.
    \begin{enumerate}
        \item Function symbols: a binary symbol $m$ and a unary symbol $i$.
        \item Relation symbols: a binary symbol $R$.
        \item Constant symbols: a constant symbol $e$.
    \end{enumerate}
    The following is an $\L$-formula:
    \begin{align*}
        m\of{m\of{x_0, x_1}, i\of{x_0}}
    \end{align*}
    Consider the first-order structure $\A$, meant to represent a group, with the underlying data.
    \begin{enumerate}
        \item Domain: $G$, a set.
        \item Functions: the binary function $\ol{m}$ of multplication (the group operation) and the unary function $\ol{i}$ of inversion.
        \item Relations: the binary relation $\Rb$ of equality.
        \item Constants: the identity element $\ol{e}$.
    \end{enumerate}
    Fix arbitrary group elements $g, h \in G$. We know, from \Cref{Ch2:Lemma:Valuation_Existence_Uniqueness}, that there exists a unique valuation $\v$ of $\L$ in $\A$ such that $\vof{x_0} = g$ and $\vof{x_1} = h$. This valuation will allow us to interpret the formula above as a statement about the group elements $g$ and $h$:
    \begin{align*}
        \vof{m\of{m\of{x_0, x_1}, i\of{x_0}}}
        &= \ol{m}\of{\vof{m\of{x_0, x_1}}, \vof{i\of{x_0}}} \\
        &= \ol{m}\of{\ol{m}\of{\vof{x_0}, \vof{x_1}}, \ol{i}\of{\vof{x_0}}} \\
        &= \ol{m}\of{\ol{m}\of{g, h}, \ol{i}\of{g}} \\
        &= \ol{m}\of{g \cdot h, g\inv} \\
        &= ghg\inv
    \end{align*}
    If we had defined the structure $\A$ a little differently---for example, if we had defined $\ol{m}\of{g, h} := hg$ instead---then the interpretation of the formula in $\A$ would have been different, despite the formula itself being exactly the same in $\L$. This illustrates the role of valuations and interpretations in moving between purely formal, syntactic expressions in first-order languages to meaningful expressions or statements in first-order structures.
\end{boxexample}

\begin{remark}
    It is worth remarking that in the above example, we use the $=$ symbol somewhat frivolously. On the one hand, left- and right-hand sides of each equation lie in $G$, and $=$ can be understood as equality in $\A$. However, one can also view these as being a \textit{syntactic} equalities in $\A$, since the properties we use to manipulate the above expressions are independent of the group structure of $\A$. The only simplifications we make come from the \textit{definitions} of  $\v$, $\ol{m}$, and $\ol{i}$ (for that matter, we can even view $\cdot$ and $\inv$ as notation for $\ol{m}$ and $\ol{i}$ instead of viewing $\ol{m}$ and $\ol{i}$ as notation for $\cdot$ and $\inv$). We sidestep these syntactic nuances in this module, but we do mention that a more syntax-heavy treatment is necessary for a computer-scientific study of first-order logic.
\end{remark}

With this, we have studied first-order languages and structures in sufficient detail to be able to talk about how we can express and implement ideas from propositional logic in first-order logic.

\section{A Bridge between Propositional and First-Order Logic}

The purpose of this section is to discuss how ideas in propositional logic can be expressed in first-order logic. As we might expect from \Cref{Ch2:Def:First-Order_Language}, first-order logic is a generalisation of propositional logic. We will use this section to make this precise.

We will begin by making rigorous the notion of \textbf{logical validity}. We will see a connection with tautologies, which, as we know, are precisely the theorems of propositional logic. We will explore notions like satisfaction and substitution to make this precise. We will then study free and bounded variables, discussing the notion of closed formulae in first-order languages.

\subsection{Satisfaction}

For the remainder of this subection, fix some \fola\ $\L$ and some $\L$-structure $\A$. We begin by defining a notion of equivalence for valuations that is weaker than equality.

\begin{boxdefinition}[Equivalence with respect to a Variable]
    Suppose $x_l$ is a variable in $\L$ and valuations $\v, \w$ of $\L$ in $\A$. We say that $\v$ and $\w$ are \textbf{$x_l$-equivalent} if for all $i \in \N \setminus \set{l}$, we have $\vof{x_i} = \wof{x_i}$.
\end{boxdefinition}

One way to see that this notion is weaker than equality of valuations is that if valuations agree on all variables, then by \Cref{Ch2:Lemma:Valuation_Existence_Uniqueness}, they must be equal. The fact that there is one variable on which they \textit{may} differ is what makes this notion weaker. We also emphasise that we only stipulate that $x_l$-equivalent valuations \textit{need not} agree on $x_l$, not that they \textit{must not} agree on $x_l$. In particular, \textit{equal valuations are equivalent with respect to all variables}.

We are now ready to define precisely what it means for a valuation to satisfy a formula. The idea is that in a general setting---ie, in a \fola---formulae do not express any properties, and without an interpretation in a \fos, it is not very meaningful to talk about what it means for a formula to be `true'. A valuation translates formulae into `meaningful statements', and therefore, informally, we can think of a valuation satisfying a formula as meaning that it translates a general formula into some interpretation that we know to be `true' in some sense.

\begin{boxdefinition}[Satisfaction]
    Let $\phi$ be an $\L$-formula and $\v$ a valuation of $\L$ in $\A$. We can define what it means for \textbf{$\v$ satisfies $\phi$ in $\A$} recursively on the number of connectives or quantifiers in $\phi$.
    \begin{enumerate}
        \item If $\phi$ is an \underline{atomic formula}, ie, has \underline{no connectives or quantifiers}, then we know $\phi$ is of the form $\Rof{t_1, \ldots, t_n}$ for some $n$-ary relation $R$ and terms $t_1, \ldots, t_n$ in $\L$. In this case, we say that \textbf{$\v$ satisfies $\phi$ in $\A$} if
        \begin{align*}
            \Rbof{\vof{t_1}, \ldots, \vof{t_n}}
        \end{align*}
        holds in $\A$, ie, if $\parenth{\vof{t_1}, \ldots, \vof{t_n}} \in \Rb \subseteq A^n$.
$
        \item If $\phi$ is \underline{not an atomic formula} in $\L$, then we know $\phi$ is of one of the following forms.
        \begin{enumerate}
            \item If $\phi$ is of the form $\parenth{\neg \psi}$ for some $\L$-formula $\psi$, then we say that \textbf{$\v$ satisfies $\phi$ in $\A$} if $\v$ does not satisfy $\psi$ in $\A$.
            
            \item 
        \end{enumerate}
    \end{enumerate}
\end{boxdefinition}

\begin{comment}
\begin{boxexample}
    Let $\L$ have one binary relation. Consider the atomic formula
    \begin{align}
        \parenth{\forall x_1}\parenth{\exists x_2} \Rof{x_1, x_2}
        \label{Ch2:Eq:Forall_Exists_Rel}
    \end{align}
    This formula is true in structures like $\cycl{\N; <}$, $\cycl{\Z, <}$, and $\cycl{\Z, >}$.
\end{boxexample}
\begin{boxnexample}
    Let $\L$ be as above. The formula~\eqref{Ch2:Eq:Forall_Exists_Rel} above is NOT true in $\cycl{\N; >}$, because $0$ does not have a predecessor in $\N$. The reason for this is that relation inputs are \textit{ordered} (ie, relations are not, in general, symmetric).
\end{boxnexample}
\end{comment}

\section{Variables and the Universal Quantifier}

The purpose of this section is to understand how to work with variables and the universal quantifier in first-order languages. Throughout this section, fix a \fola\ $\L$.

\subsection{Bound and Free Variables}

Consider the formula
\begin{align}
    \psi_1 : \parenth{R_1\of{x_1, x_2} \to \parenth{\forall x_3}R_2\of{x_1, x_3}}
    \label{Ch2:Eq:Free_Bound_Motivation}
\end{align}
where $R_1, R_2$ are relation symbols in $\L$ and $x_1, x_2, x_3$ are variables in $\L$. Intuitively, we do not want $x_3$ to `exist', or be `known' or `accessible', outside of the sub-formula\footnote{When we say sub-formula, we mean exactly what it sounds like: since formulae are built from smaller formulae using connectives and the quantifier, when we say sub-formula, we mean a formula that arose at an intermediate step in this inductive construction of the formula of which it is a sub-formula.} $\parenth{\forall x_3}R_2\of{x_1, x_3}$. We can mae this notion precise.

\begin{boxdefinition}[Scope of a Quantifier]
    Suppose $\phi$ and $\psi$ are $\L$-formulae. If $\parenth{\forall x_i}\phi$ occurs as a sub-formula of $\psi$, we say that $\phi$ is the  \textbf{scope of $\parenth{\forall x_i}$ in $\psi$}.
\end{boxdefinition}
\begin{boxexample}
    In~\eqref{Ch2:Eq:Free_Bound_Motivation}, $R_2\of{x_1, x_3}$ is the scope of the quantifier $\parenth{\forall x_3}$.
\end{boxexample}

We use the existence of a scope to characterise variables.

\begin{boxdefinition}[Bound Variables]
    Let $\psi$ be an $\L$-formula and let $x_j$ be a variable that appears in $\psi$. We say that $x_j$ is \textbf{bound in $\psi$} if it is within the scope of a quantifier.
\end{boxdefinition}

\begin{boxexample}
    In~\eqref{Ch2:Eq:Free_Bound_Motivation}, the variable $x_3$ is bound, because it lies in the scope $R_2\of{x_1, x_3}$ of the quantifier $\parenth{\forall x_3}$.
\end{boxexample}

We also have a special name for variables that are not bound.

\begin{boxdefinition}[Free Variables]
    Let $\psi$ be an $\L$-formula and let $x_j$ be a variable that appears in $\psi$. We say that $x_j$ is \textbf{free in $\psi$} if it is not bound in $\psi$.
\end{boxdefinition}

\begin{boxexample}
    In~\eqref{Ch2:Eq:Free_Bound_Motivation}, the variable $x_1$ is bound, because it does not lie in the scope of any quantifiers, meaning it is not bound.
\end{boxexample}

The same variable can have both free and bound occurrences within a given formula.

\begin{boxexample}
    In the formula
    \begin{align*}
        \parenth{\parenth{\forall x_1}R_1\of{x_1, x_2} \to R_2\of{x_1, x_2}}
    \end{align*}
    the variable $x_1$ is bound in the expression
    \begin{align*}
        \parenth{\forall x_1}\underbrace{R_1\of{x_1, x_2}}_{\text{scope of } \parenth{\forall x_1}}
    \end{align*}
    but free in the expression
    \begin{align*}
        R_2\of{x_1, x_2}
    \end{align*}
\end{boxexample}

We can also show that existentially quantified variables are bound.

\begin{boxlemma}
    Let $\psi$ and $\phi$ be an $\L$-formulae. If $\psi$ contains
    \begin{align*}
        \parenth{\exists x_1}\phi
    \end{align*}
    as a sub-formula, then $x_1$ is bound in $\psi$.
\end{boxlemma}
\begin{proof}
    This follows immediately from the definition of the existential quantifier. \sorry
\end{proof}

The occurrence of free variables inside a formula means that it is very general, as the variables in it are purely formal symbols that we have no machinery to reason with. We give a special term to, and have a special interest in, formulae with no free variables.

\begin{boxdefinition}[Closed Formulae]
    An $\L$-formula with no free variables is called a \textbf{closed formula} or a \textbf{sentence}.
\end{boxdefinition}

Formulae that are not closed can be thought of as being `dependent' on their free variables, in the sense that we would intuitively want to only define a notion of substitution only for free variables. To that end, we adopt some notation.

\begin{boxconvention}
    Let $\psi$ be an $\L$-formula. If $\L$ has free variables $x_1, \ldots, x_n$, then we denote $\psi$ by 
    \begin{align*}
        \psi\of{x_1, \ldots, x_n}
    \end{align*}
    when we wish to underscore the dependence of $\psi$ on $x_1, \ldots, x_n$ or the fact that $x_1, \ldots x_n$ are free in $\psi$ or perform a substitution.
\end{boxconvention}

We now define substitution.

\begin{boxdefinition}[Substitution]\label{Ch2:Def:Substitution}
    Let $t_1, \ldots, t_n$ be terms in $\L$ and $\psi\of{x_1, \ldots, x_n}$ an $\L$-formula with free variables $x_1, \ldots, x_n$\footnote{By the aforementioned convention, we would ordinarily not mention that $x_1, \ldots, x_n$ are free in $\psi$.}. We define the formula obtained by \textbf{substituting the $t_i$ for the $x_i$} to be the $\L$-formula obtained by replacing each occurrence of $x_i$ in $\psi$ with an occurrence of the corresponding $t_i$. We denote this formula
    \begin{align*}
        \psi\of{t_1, \ldots, t_n}
    \end{align*}
\end{boxdefinition}

\subsection{An Analogue of Completeness}

Throughout this subsection, fix an $\L$-formula $\phi$ and an $\L$-structure $\A$.

We begin by mentioning a result about valuations.

\begin{boxtheorem}
    Fix $n \in \N \cup \set{0}$. If $\phi$ has free variables $x_1, \ldots, x_n$ and $\v, \w$ are valuations of $\L$ in $\A$ with $\vof{x_i} = \wof{x_i}$ for all $1 \leq i \leq n$, then
    \begin{align*}
        \vbrac{\phi} = \T \tiff \wbrac{\phi} = \T
    \end{align*}
    In other words, we have $\vbrac{\phi} = \wbrac{\phi}$.
\end{boxtheorem}
\begin{proof}
    Fix valuations $\v$ and $\w$ of $\L$ in $\A$ that agree on all the free variables of $\phi$. We argue that $\vbrac{\phi} = \wbrac{\phi}$ by performing induction on the total number connectives and quantifiers in $\phi$.

    The base case is when $\phi$ has no connectives or quantifiers, ie, when $\phi$ is atomic in terms that contain no quantifiers. In this case, we have that $\phi$ is of the form
    \begin{align*}
        \Rof{t_1, \ldots, t_m}
    \end{align*}
    where $t_1, \ldots, t_m$ are terms that contain no quantifiers. In this case, each $t_i$ is either a constant or a variable. Since $\phi$ is assumed not to contain any quantifiers, the $t_i$ that are variables must be free variables, and the rest must be constants. Since $\v$ and $\w$ agree on all constants by definition and on all free variables by assumption, we must have
    $\vof{t_i} = \wof{t_i}$ for all $i$. Then, \sorry
\end{proof}

This gives us a completeness-like result about closed formulae.

% Main Thm: 2.3.3 (David Ang's notes)
\begin{boxcorollary}
    If $\phi$ is closed, then either $\A \models \phi$ or $\A \models \parenth{\neg \phi}$.
\end{boxcorollary}
\begin{proof}
    If $\phi$ is closed, $\phi$ has no free variables. Then, any two valuations of $\L$ in $\A$ agree vacuously on the set of all free variables of $\phi$. Therefore, there cannot be two distinct valuations such that one satisfies $\phi$ and one does not. In other words, either $\phi$ is true in $\A$ or $\phi$ is not true in $\A$. Equivalently, either $\phi$ or $\parenth{\neg \phi}$ is true in $\A$. 
\end{proof}

\subsection{Understanding the Universal Quantifier}

The purpose of this subsection is to discuss the relationship between the syntax and the semantics of the universal quantifier $\forall$. We have not been too precise so far about what it \textit{means} to write an expression like
\begin{align*}
    \parenth{\forall x_1}\phi
\end{align*}
for some formula $\phi$ with a bound variable $x_1$. We particularly delve into nuances that arise due to the fact that in our strict syntactic definition of $\L$-formulae (\Cref{Ch2:Def:Formula_FO_Logic}, we only allow the quantifier to be succeeded by a \textbf{variable} in $\L$, which must be one of $x_1, x_2, \ldots$. The really confusing thing about this syntactic definition of the universal quantifier is the fact that it is not obvious that this quantifier means ``for all'': instead of saying ``for all variables $x_1$, we have $\phi$'', on a strictly syntactic level, the expression $\parenth{\forall x_1}\phi$ is using the \textit{specific variable} $x_1$.

What we would want---say, in an interactive theorem prover like Lean---is a mechanism to be able to \textit{remove} the quantifier and instead \textit{introduce} a free variable into our formula and label that variable differently from all other free and bound variables in our formula.

In this subsection, we will make this notion precise. Throughout, fix a \fola\ $\L$ and a \fos\ $\A$ in $\L$. Denote the domain of $\A$ by $A$.

We begin by describing what it means for a structure to model a formula under a valuation.

\begin{boxdefinition}
    Let $\psi\of{x_1, \ldots, x_n}$ be an $\L$-formula whose free variables are $x_1, \ldots, x_n$. Let $a_1, \ldots, a_n \in A$ be constants. We say that \textbf{$\A$ models $\psi\of{a_1, \ldots, a_n}$}, denoted
    \begin{align*}
        \A \models \psi\of{a_1, \ldots, a_n}
    \end{align*}
    if $\vbrac{\psi} = \T$ for every valuation $\v$ of $\L$ in $\A$.
\end{boxdefinition}
Note that by \Cref{Ch2:Lemma:Valuation_Existence_Uniqueness}, to prove that $\A \models \psi\of{a_1, \ldots, a_n}$, it suffices to prove that $\vbrac{\psi} = \T$ for \textit{some} valuation $\v$ of $\L$ in $\A$.

We would want to be able to use the quantifier $\forall$ in the way that we are used to using it. In other words, from a hypothesis
\begin{align*}
    \parenth{\forall x_1} \phi\of{x_1}
\end{align*}
we would want to be able to deduce
\begin{align*}
    \phi\of{t}
\end{align*}
for any term $t$ in $\L$. Unfortunately, given that $x_1$ is \textit{itself} a term in $\L$, and $\parenth{\forall x_1} \phi\of{x_1}$ is does not involve some `general variable' but the very concrete variable $x_1$, \underline{this deduction is simply not true!}\footnote{It is at times like this that it is particularly important to distinguish semantics and syntax. When we say ``simply not true'', that is an inherently semantic statement: syntactic reasoning is not a reasoning of \textit{truth}, but one of \textit{provability}. What we are really attempting to do is establish \textit{motivation} for the axioms of the formal deduction system for a first-order language. We will see that in our formal system, we will not allow this type of deduction. This section underscores that it is actually \textit{sensible} to disallow this explicitly, because even in an `intuitive' sense of reasoning, this would not `make sense'.}

\begin{boxcexample}[Indiscriminate specialisation is not a good idea]
    We provide a counterexample that disproves the claim that we can indiscriminately specialise a formula $\parenth{\forall x_1} \phi\of{x_1}$ to \textit{any} $\L$-term $t$ to obtain a true statement $\phi\of{t}$. More precisely, we give
    \begin{itemize}
        \item a first-order language $\L$
        \item an $\L$-structure $\A$
        \item an $\L$-formula $\phi\of{x_1}$ with a free variable $x_1$
        \item an $\L$-term $t$
        \item a valuation $\v$ of $\L$ in $\A$
    \end{itemize}
    such that
    \begin{align*}
        \A \models \parenth{\forall x_1}\phiof{x_1}
    \end{align*}
    but
    \begin{align*}
        \vbrac{\phiof{t}} = \F
    \end{align*}
    Let $\L$ have one binary relation $R$ and one unary relation $S$. Let
    \begin{align*}
        \phiof{x_1} : \parenth{\parenth{\forall x_2}\Rof{x_1, x_2} \to S\of{x_1}}
    \end{align*}
    We can specialise this formula $\phi$ via a \textbf{substitution instance} (cf. \Cref{Ch2:Def:Substitution}). Let $t$ be the term $x_2$. Since \Cref{Ch2:Def:Substitution} imposes \textbf{no conditions} on what can be substituted for a free variable, we can substitute $t$ for $x_1$ in $\phi$. Then, we have that $\phiof{t}$ is the formula
    \begin{align*}
        \parenth{\forall x_2}\Rof{x_2, x_2} \to S\of{x_2}
    \end{align*}
    The problem with this substitution is that the $x_1$ that served as the first argument of $R$ prior to substitution was free, whereas the $x_2$ was bound to the quantifier $\parenth{\forall x_2}$ that preceded $\Rof{x_1, x_2}$. After substitution, both arguments of $R$ are now bound to the quantifier $\parenth{\forall x_2}$. We can see this in the following model. \newline

    Let $\A = \cycl{\N; \le, \parenth{= 0}}$. That is, $\A$ is the $\L$-structure with the following data.
    \begin{itemize}
        \item The domain of $\A$ is $\N$.
        \item The binary relation in $\A$ is the ordering $\le$.
        \item The unary relation on $\A$ is the condition that the argument be equal to zero.
    \end{itemize}
    In this case, we can certainly see that
    \begin{align*}
        \A \models \parenth{\forall x_1}\phiof{x_1}
    \end{align*}
    because for any natural number $x_1$, if $x_1 \leq x_2$ for all natural numbers $x_2$, then $x_1 = 0$.
    \sorry
\end{boxcexample}

\sorry

% Main Thm: 2.3.6 (David Ang's notes)

\section{A Formal System for First-Order Logic}

The purpose of this section is to define, for any first-order language $\L$, a \textbf{formal deduction system $\KL$} in which we can do \textbf{first-order logic}.

Throughout this section, fix a first-order language $\L$.

\subsection{The Formal Deduction System $\KL$}

Recall \Cref{Ch1:Def:Formal_Deduction_System}, in which we define a \textbf{formal deduction system}. Informally, this is meant to be a system in which we can perform \textbf{deductions}. The purpose of this subsection is to define a \textbf{formal deduction system for $\L$}.

\begin{boxdefinition}[A Formal Deduction System for First-Order Logic]
    Define $\KL$ to be the \textbf{formal deduction system} consisting of the following alphabet, formulae, axioms and deduction rules.
    \begin{enumerate}
        \item \textbf{\underline{Alphabet.}} The \textbf{alphabet} of $\KL$ is the alphabet of $\L$ (cf. \Cref{Ch2:Def:First-Order_Language}).
        
        \item \textbf{\underline{Formulae.}} The \textbf{formulae} of $\KL$ are the formulae of $\L$ (cf. \Cref{Ch2:Def:Formula_FO_Logic}).
        
        \item \textbf{\underline{Axioms.}} For any $\L$-formulae $\phi, \psi, \chi$ and $i \in \N$, the \textbf{axioms} of $\KL$ in $\phi, \psi, \chi$ and $x_i$ are the following distinguished $\L$-formulae:
        \begin{enumerate}[label = \normalfont (A\arabic*)]
            \item\label{FO:A1}
            $\displaystyle \parenth{\phi \to \parenth{\psi \to \phi}}$
            \item\label{FO:A2}
            $\displaystyle \parenth{\parenth{\phi \to \parenth{\psi \to \chi}} \to \parenth{\parenth{\phi \to \psi} \to \parenth{\phi \to \chi}}}$
            \item\label{FO:A3}
            $\displaystyle \parenth{\parenth{\parenth{\neg \phi} \to \parenth{\neg \psi}} \to \parenth{\psi \to \chi}}$
        \end{enumerate}
        \begin{enumerate}[label = \normalfont (K\arabic*)]
            \item\label{FO:K1}
            $\displaystyle \parenth{\parenth{\forall x_i}\phiof{x_i} \to \phiof{t}}$
            where $t$ is a term free for $x_i$ in $\phi$ (cf. \sorry) and $\phi$ can have other free variables.
            \item\label{FO:K2}
            $\displaystyle \parenth{\parenth{\forall x_i}\parenth{\phi \to \psi} \to \parenth{\phi \to \parenth{\forall x_i}\psi}}$
            if $x_i$ is not free in $\phi$.
        \end{enumerate}

        \item \textbf{\underline{Deduction Rules.}} For any $\L$-formulae $\phi, \psi$ and $i \in \N$, the \textbf{deduction rules} for $\KL$ in $\phi, \psi, x_i$ are
        \begin{enumerate}[label = \normalfont (MP)]
            \item\label{FO:MP}
            \underline{Modus Ponens:} From $\phi$ and $\parenth{\phi \to \psi}$, deduce $\psi$.
        \end{enumerate}
        \begin{enumerate}[label = \normalfont (Gen)]
            \item\label{FO:Gen}
            \underline{Generalisation:} From $\phi$, deduce $\parenth{\forall x_i}\phi$.
        \end{enumerate}
    \end{enumerate}
\end{boxdefinition}

We define deductions identically to how we did for arbitrary formal deduction systems.

\begin{boxdefinition}[Deductions in $\KL$]\label{Ch2:Def:Deductions_in_KL}
    A \textbf{deduction} in $\KL$ is a finite sequence of $\KL$-formulae such that each either is one of the following:
    \begin{itemize}
        \item an axiom of $\KL$.
        \item a formula in $\Sigma$.
        \item deduced from a previous formula in the sequence via the deduction rule~\ref{FO:MP}.
        \item deduced from a previous formula in the sequence via the deduction rule~\ref{FO:Gen} with the restriction that when \ref{FO:Gen} is applied to deduce $\parenth{\forall x_i}\phi$ from $\phi$, $x_i$ does not appear as a free variable in any formula of $\Sigma$ that is used in the deduction of $\phi$.
    \end{itemize}
     We write
    \begin{align*}
        \Sigma \vkl \psi
    \end{align*}
    to mean that a $\KL$-formula $\psi$ occurs at the end of a deduction from $\Sigma$.
\end{boxdefinition}

We can similarly define proofs.

\begin{boxdefinition}[Proofs in $\KL$]
    A \textbf{proof} in $\KL$ is a finite sequence of deductions made from the empty set.
\end{boxdefinition}

Theorems are the results deduced in proofs.

\begin{boxdefinition}[Theorems in $\KL$]
    A \textbf{theorem} of $\KL$ is a $\KL$-formula that occurs at the end of a proof. We write
    \begin{align*}
        \vkl \psi
    \end{align*}
    to mean that $\psi$ is a theorem of $\KL$.
\end{boxdefinition}

There is a very deep reason why we restrict the use of \ref{FO:Gen} to deduce $\parenth{\forall x_i}\phi$ from $\phi$ in \Cref{Ch2:Def:Deductions_in_KL} and the subsequent definitions of proofs and theorems in $\KL$. Unfortunately, we are not yet in a position to discuss it. But we will in due course: see \Cref{Ch2:CEg:Gen_Restriction}.

\subsection{Tools for Deduction}

An important objective of our discussion on the formal system $\KL$ is to show that the theorems of $\KL$ are precisely the logically valid formulae. In \Cref{Ch2:Sec:Prop_Logic}, we established a bridge between Propositional and First-Order Logic. In particular, we showed that the theorems in the formal system $\bL$---that is, the tautologies---are all logically valid (see \Cref{Ch2:Thm:Tauto_Logic_Valid}). We can actually show something stronger.

\begin{boxtheorem}\label{Ch2:Thm:L_Tauto_Thm_FO}
    Let $\phi$ be an $\KL$-formula that is a substitution instance of a tautology in propositional logic. Then, $\vkl \phi$.
\end{boxtheorem}
\begin{proof}
    \sorry % Apply completeness of prop logic to turn into a proof in L. Then perform same substitution as phi in all steps.
\end{proof}

While the above gives us a way of dealing with formulae that come from first-order logic, we also have a way of getting access to the first of two formulae linked by a $\to$ connective.

\begin{boxtheorem}[Deduction Theorem for $\KL$]\label{Ch2:Thm:Deduction}
    Let $\Sigma$ be a set of $\L$-formulae and let $\phi$ and $\psi$ be $\L$-formulae. If $\Sigma \cup \set{\phi} \vkl \psi$, then $\Sigma \vkl \parenth{\phi \to \psi}$.
\end{boxtheorem}
\begin{proof}
    \sorry % Follows the proof of the deduction theorem for prop logic. Goes by induction on length of deduction.
\end{proof}



\subsection{Soundness}

In this subsection, we show one direction of our claim that the theorems of $\KL$ are precisely the logically valid formulae (cf. \Cref{Ch2:Def:Logical_Validity}).

\begin{boxtheorem}[Soundness Theorem for $\KL$]\label{Ch2:Thm:KLSoundness}
    Let $\phi$ be a $\KL$-formula. If $\vkl \phi$, then $\models \phi$. In other words, if $\phi$ is a theorem of $\KL$, then $\phi$ is logically valid.
\end{boxtheorem}
\begin{proof}
    As with the proof of the Deduction Theorem (\Cref{Ch2:Thm:Deduction}), we follow the proof of the Soundness Theorem for $\bL$ (\Cref{Ch1:Thm:LSoundness}). Our strategy for showing that all formulae deduced from the empty set are logically valid will be to prove the following.
    \begin{enumerate}
        \item \underline{The axioms of $\KL$ are logically valid.}

        We do not need to worry about the axioms~\ref{FO:A1}, \ref{FO:A2} and \ref{FO:A3}, because they are substitution instances of propositional tautologies, making them logically valid by \Cref{Ch2:Thm:Tauto_Logic_Valid}. Furthermore, the axiom~\ref{FO:K1} is logically valid by \sorry % Thm 2.3.5 in David Ang's notes - we haven't got it in our notes yet!
        Finally, the axiom~\ref{FO:K2} is logically valid by \sorry % Finish from David Ang's notes.

        \item \underline{Deductions preserve logical validity.}

        \sorry % Left as an exercise in David Ang's notes. We need to think about how to do it.
    \end{enumerate}
\end{proof}

We have an important corollary that mirrors \sorry % Thing from prop logic that was in coursework 1 - we really gotta complete that part of the notes! Aarghhhh

\begin{boxcorollary}
    Let $\Sigma$ be a set of $\KL$-formulae and let $\psi$ be a $\KL$-formula. Suppose that
    \begin{align*}
        \Sigma \vkl \psi
    \end{align*}
    Then, for any $\L$-structure $\A$ and valuation $\v$ of $\L$ in $\A$, if $\vbrac{\sigma} = \T$ for every $\sigma \in \Sigma$, then $\vbrac{\phi} = \T$.
\end{boxcorollary}
\begin{proof}
    \sorry % Left as an exercise in David Ang's notes.
\end{proof}

We are now in a position to comment on the restriction we impose on the use of the deduction rule~\ref{FO:Gen} in \Cref{Ch2:Def:Deductions_in_KL} and the subsequent definitions of proofs and theorems in $\KL$. We want our formal system to be \textbf{sound}, but it turns out that if we did not have this restriction, we would be able to construct a counterexample disproving soundness.
\begin{boxcexample}[Disproving soundness when we allow unrestricted use of \ref{FO:Gen}]\label{Ch2:CEg:Gen_Restriction}
    If we allowed \ref{FO:Gen} to deduce $\parenth{\forall x_i}\phi$ from formulae in which $x_i$ occurs freely, in a language $\L$ with a unary relation $R$, we would have
    \begin{align*}
        \set{\Rof{x_1}} \vkl \parenth{\forall x_1}\Rof{x_1}
    \end{align*}
    In structures, such a statement would make no sense, as it would say that we can deduce a fact about all constants from a fact about a single constant. Indeed, by the Deduction Theorem (\Cref{Ch2:Thm:Deduction}), if $\set{\Rof{x_1}} \vkl \parenth{\forall x_1}\Rof{x_1}$, then we would have
    \begin{align*}
        \vkl \parenth{\Rof{x_1} \to \parenth{\forall x_1}\Rof{x_1}}
    \end{align*}
    However, in an $\L$-structure like $\A = \cycl{\N; \parenth{= 0}}$, with domain $\N$ and unary relation being the condition that an element be equal to $0$, we have
    \begin{align*}
        \A \not\models \parenth{\Rof{x_1} \to \parenth{\forall x_1}\Rof{x_1}}
    \end{align*}
    because if you take the valuation $\v$ that maps $x_1$ to $0 \in \N$, we have that
    \begin{align*}
        \vof{\parenth{\Rof{x_1} \to \parenth{\forall x_1}\Rof{x_1}}} = \F
    \end{align*}
    because indeed, it is the case that $x_1 = 0$ and $\vbrac{\parenth{\forall x_1}\Rof{x_1}} = \F$ for any valuation that maps $x_1$ to $0$. In particular, we would have that
    \begin{align*}
        \not\models \parenth{\Rof{x_1} \to \parenth{\forall x_1}\Rof{x_1}}
    \end{align*}
    In other words, despite having the deduction
    \begin{align*}
        \vkl \parenth{\Rof{x_1} \to \parenth{\forall x_1}\Rof{x_1}}
    \end{align*}
    the formula
    \begin{align*}
        \parenth{\Rof{x_1} \to \parenth{\forall x_1}\Rof{x_1}}
    \end{align*}
    would not be a valid theorem in $\L$. This is a direct contradiction of the Soundness Theorem (\Cref{Ch2:Thm:KLSoundness}). Therefore, the restriction on the use of \ref{FO:Gen} \textbf{is necessary} for soundness.
\end{boxcexample}

Apart from offering an explanation for the restriction on the use of the deduction rule~\ref{FO:Gen} in first-order logic, the Soundness Theorem gives us more properties we can use to make deductions. It brings us an important step closer to proving that the theorems of first-order logic are precisely the logically valid formulae.

\subsection{Consistency}

Again, there is some overlap with propositional logic.

\begin{boxdefinition}[Consistency]
    A set $\Sigma$ of $\KL$-formulae is \textbf{consistent} if there is no formula $\phi$ such that
    \begin{align*}
        \Sigma \vkl \phi
        \qquad \text{ and } \qquad
        \Sigma \vkl \parenth{\neg\phi}
    \end{align*}
\end{boxdefinition}

The Soundness Theorem allows us to prove an important result.

\begin{boxtheorem}[The Consistency Theorem]
    The empty set $\emptyset$ of $\KL$-formulae is consistent.
\end{boxtheorem}
\begin{proof}
    \sorry
\end{proof}

\begin{boxconvention}
    We say \textbf{$\KL$ is consistent}.
\end{boxconvention}

For the remainder of this subsection, fix a set $\Sigma$ of $\KL$-formulae. We mention an analogue of the intuitive idea that `a proof of False yields anything'.

\begin{boxproposition}
    If $\Sigma$ is inconsistent, then for all $\KL$-formulae $\chi$, we have
    \begin{align*}
        \Sigma \vkl \chi
    \end{align*}
\end{boxproposition}
\begin{proof}
    \sorry % See Remark after Def 2.5.1 in David Ang's notes.
\end{proof}

We state a few more consistency results that are analogous to propositional logic.\todo{Give cross references!}

\begin{boxproposition}
    Suppose $\Sigma$ is consistent and consists entirely of closed $\KL$formulae. Let $\phi$ be a closed $\KL$-formula. If $\Sigma \not\vkl \phi$, then $\Sigma \cup \set{\parenth{\neg \phi}}$ is consistent.
\end{boxproposition}
\begin{proof}
    \sorry % Analogous to propositional logic.
\end{proof}

\begin{boxproposition}[Analogue of Lindenbaum's Lemma]\label{Ch2:Prop:Lindenbaum_FO}
    Suppose $\Sigma$ consists entirely of closed $\KL$-formulae. There is a consistent set $\Sigma^*$ of closed $\KL$-formulae such that $\Sigma \subseteq \Sigma^*$.
\end{boxproposition}
\begin{proof}
    \sorry % Use an enumeration again
\end{proof}

The above results have important consequences that we will discuss in the next section. These will be instrumental in proving a converse result for \Cref{Ch2:Thm:KLSoundness}.

% Up to Prop. 2.5.2 in David Ang's notes

\subsection{Model Existence}

In this subsection, we state and sketch the proof of an important result that will serve as a stepping-stone towards proving converse of the Soundness Theorem for $\KL$, known as the Completeness Theorem.

First, we introduce some notation.

\begin{boxconvention}
    Let $\A$ be an $\L$-structure and $\Sigma$ a set of $\L$-formulae. We write
    \begin{align*}
        \A \models \Sigma
    \end{align*}
    if for all $\sigma \in \Sigma$, $\A \models \sigma$.
\end{boxconvention}

We have a rather surprising existence result for a model for any consistent set of closed formulae.

\begin{boxtheorem}[Model Existence Theorem]\label{Ch2:Thm:Model_Existence}
    Suppose $\L$ is a \textit{countable} \fola\ and $\Sigma$ is a consistent set of $\L$-formulae. Then, there is a countable $\L$-structure $\A$ such that $\A \models \Sigma$.
\end{boxtheorem}
\begin{proof}[Proof Sketch]
    This is quite a lengthy proof, so we do not give more than a sketch here. The details can be found in~\cite[Appendix A.2]{LecNotes2018}. There are several steps.

    \begin{enumerate}[label = \underline{\textbf{Step \arabic*.}}]
        \item \textbf{Extending our Language.}
        
        We define a new language $\L^+$ that is an extension of $\L$ with countably many constant symbols. That is, the constants in $\L^+$ are the constants in $\L$ together with countably many new constants $b_0, b_1, b_2, \ldots$. Note that $\L^+$ is also countable because $\L$ is, and we only add countably many new constants.
        
        \item \textbf{Adding Witnesses.}
        
        We can regard $\Sigma$ as a set of closed $\L^+$-formulae. It is possible to prove that $\Sigma$ is still consistent as a set of $\L^+$-formulae. In fact, we can extend $\Sigma$ to a set $\Sigma_{\infty} \supseteq \Sigma$ such that
        \begin{itemize}
            \item every formula in $\Sigma_{\infty}$ is closed
            \item for every $\L^+$-formula $\theta\of{x_i}$ with one free variable $x_i$, there exists some constant $b_j$ in $\L^+$ such that
            \begin{align*}
                \Sigma_{\infty} \vdash_{\K_{\L^+}}
                \parenth{
                    \parenth{
                        \neg\parenth{\forall x_i}\theta\of{x_i}
                    }
                    \to
                    \parenth{
                        \neg \theta\of{b_j}
                    }
                }
            \end{align*}
            This formula essentially says
            \begin{align*}
                \Sigma_{\infty} \vdash_{\K_{\L^+}}
                \parenth{
                    \parenth{
                        \parenth{\exists x_i}\parenth{\neg\theta\of{x_i}}
                    }
                    \to
                    \parenth{
                        \neg \theta\of{b_j}
                    }
                }
            \end{align*}
            and we say ``$b_j$ \textbf{witnesses} the existence of $x_i$ satisfying $\parenth{\neg\theta\of{x_i}}$''.
        \end{itemize}
        We say that this process of constructing $\Sigma_{\infty}$ ``\textbf{adds witnesses}.''

        \item \textbf{Constructing a Complete Set of Formulae.}
        
        We can now apply Lindenbaum's Lemma (\Cref{Ch2:Prop:Lindenbaum_FO}) to $\Sigma_{\infty}$: there exists a consistent set $\Sigma^*$ of closed $\L^+$-formulae such that for every closed $L^+$-formula $\phi$, we have either $\Sigma^* \vdash_{\K_{\L^+}} \phi$ or $\Sigma^* \vdash_{\K_{\L^+}} \parenth{\neg \phi}$. We will not use $\Sigma^*$ right away, but it will soon become clear what the relevance of this set is.
        
        \item \textbf{Constructing an $\L^+$-structure.}
        
        \sorry
        
        \item\label{Ch2:Thm:Model_Existence:Step:Result_on_L_plus} \textbf{A Result on $L^+$.}
        
        \sorry

        \item \textbf{Restricting \ref{Ch2:Thm:Model_Existence:Step:Result_on_L_plus} to $\L$.}
        
        \sorry
    \end{enumerate}
\end{proof}

What's more surprising than the actual existence result is the accompanying countability result: the model we constructed above is countable!

It turns out that the Model Existence Theorem has a very important consequence. So important is this consequence that we shall give it its own subsection.

\subsection{Compactness}

We now state and prove one of the most important theorems of first-order logic, known as the Compactness Theorem.

\begin{boxtheorem}[The Compactness Theorem for First-Order Logic]\label{Ch2:Thm:Compactness}
    Assume that $\L$ is a \textit{countable} first-order language. Let $\Sigma$ be a set of closed $\L$-formulae. If every finite subset of $\Sigma$ has a model, then $\Sigma$ has a model.
\end{boxtheorem}
\begin{proof}
    Suppose $\Sigma$ has no model. Then, by the Model Existence Theorem (\Cref{Ch2:Thm:Model_Existence}), $\Sigma$ must be inconsistent (because if it is consistent, \Cref{Ch2:Thm:Model_Existence} tells us it must have a model).
    
    If $\Sigma$ is inconsistent, there exists an $\L$-formula $\chi$ such that $\Sigma \vkl \chi$ and $\Sigma \vkl \parenth{\neg \chi}$. Since deductions in $\KL$ are finite, there is a finite subset $\Sigma_{0} \subseteq \Sigma$ such that $\Sigma_{0} \vkl \chi$ and $\Sigma_0 \vkl \parenth{\neg \chi}$. This makes $\Sigma_0$ inconsistent. However, by our assumption that every finite subset of $\Sigma$ has a model, we have that $\Sigma_0$ has a model. This contradicts the Soundness Theorem (\Cref{Ch2:Thm:KLSoundness}), because it tells us that we then have a model of both $\chi$ and $\parenth{\neg \chi}$, which the Soundness Theorem does not allow\todo{prove this as a corollary in the soundness section}. Therefore, $\Sigma$ must have a model.
\end{proof}

We are now ready for the converse of the Soundness Theorem.

\subsection{Completeness}

Our strategy will be to first prove the result for closed formulae and then prove it in general.

We begin by introducing some notation.

\begin{boxconvention}
    Let $\Sigma$ a set of $\L$-formulae and $\phi$ an $\L$-formula. We write
    \begin{align*}
        \Sigma \models \phi
    \end{align*}
    if every model of $\Sigma$ is a model of $\phi$. That is, $\Sigma \models \phi$ if for every $\L$-structure $\A$, it holds that if $\A \models \Sigma$, then $\A \models \phi$.
\end{boxconvention}

We are now in a position to prove the converse of the Soundness Theorem for $\KL$ for closed formulae. It turns out to be a consequence of the following important theorem.

\begin{boxtheorem}\label{Ch2:Thm:Generalised_KLCompleteness_Closed_Formulae}
    Let $\Sigma$ be a set of closed $\L$-formulae and $\phi$ a closed $\L$-formula. If $\Sigma \models \phi$, then $\Sigma \vkl \phi$.
\end{boxtheorem}
\begin{proof}
    \sorry
\end{proof}

The case when $\Sigma = \emptyset$ gives us a converse of the Soundness Theorem for $\KL$ for closed formulae.

\begin{boxcorollary}[A Partial Converse of Soundness]\label{Ch2:Cor:KLCompleteness_Closed_Formulae}
    Let $\phi$ be a closed $\L$-formula. If $\models \phi$, then $\vkl \phi$.
\end{boxcorollary}
\begin{proof}
    When $\Sigma = \emptyset$, we do, indeed have that $\Sigma \models \phi$ is the same as $\models \phi$: for any $\L$-structure $\A$, since $\A$ vacuously models every element of $\Sigma$, the condition that if $\A \models \sigma$ for all $\sigma \in \Sigma$ then $\A \models \phi$ is precisely the condition that $\A \models \phi$. \Cref{Ch2:Thm:Generalised_KLCompleteness_Closed_Formulae} then gives us that $\vkl \phi$.
\end{proof}

It turns out that \Cref{Ch2:Cor:KLCompleteness_Closed_Formulae}, a converse of the Soundness Theorem for closed formulae, gives us everything we need to prove the result in general.

\begin{boxtheorem}[Gödel's Completeness Theorem for $\KL$]\label{Ch2:Thm:Godel_Completeness}
    Let $\phi$ be any $\L$-formula. If $\phi$ is logically valid, then $\phi$ is a theorem of $\L$. That is, if $\models \phi$, then $\vkl \phi$.
\end{boxtheorem}
\begin{proof}
    By \Cref{Ch2:Cor:KLCompleteness_Closed_Formulae}, it suffices to consider the case where $\phi$ is not closed, that is, where $\phi$ has free variables.
    \sorry
\end{proof}

\section{Important Examples}

In this section, we present examples that illustrate everything we have studied in this chapter.