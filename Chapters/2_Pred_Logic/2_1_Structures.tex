\section{Important Definitions and First Examples}

\subsection{Representations and Modules}

\begin{boxdefinition}[Representation]\label{Ch2:Def:Representation}
    A \textbf{representation} of $L$ is a Lie algebra homomorpshism $\rho : L \to \gl{V}$ for some finite-dimensional $\C$-vector space $V$.
\end{boxdefinition}

\begin{boxexample}[The Adjoint Representation]
    The adjoint map $\ad : L \to \gl{L}$ is a representation: \Cref{Ch1:Prop:AdjointLieAlgHom} tells us it is a Lie algebra homomorphism.
\end{boxexample}

If $\rho : L \to \gl{V}$ is a representation, then we can define a map $\parenth{\ell, v} \mapsto \rho(\ell)(v) : L \times V \to V$. We can use this to define a Lie algebra module, similar to the concept of group modules when defining complex representations thereof.

\begin{boxdefinition}[Lie Module]\label{Ch2:Def:LieAlgModule}
    A \textbf{Lie module}, or \textbf{$L$-module}, is a finite-dimensional vector space $V$ with a pairing $\rho : L \times V \to V$ such that
    \begin{enumerate}
        \item $\rho$ is $\C$-bilinear.
        \item $\rho(\brac{a, b}, v) = \rho(a, \rho(b, v)) - \rho(b, \rho(a, v))$.
    \end{enumerate}
\end{boxdefinition}

Indeed, one can show that any representation $\rho : L \to V$ satisfies the above properties with respect to the pairing $\parenth{\ell, v} \mapsto \rho(\ell)(v)$ and that any Lie algebra module $V$ with pairing $\rho$ admits a uniquely defined representation $\ell \mapsto \rhoof{\ell, \cdot} : L \to \gl{V}$. Thus, the two concepts are equivalent.

\begin{boxconvention}
    We will abuse notation and not distinguish the notions of representations and modules. Similarly, we will not split hairs about the notation for the two: $\rhoof{\cdot, \cdot}$ should be interpreted as meaning the same thing as $\rho(\cdot)(\cdot)$.
\end{boxconvention}

Indeed, as with representations of groups and group modules, we have an equivalence of categories between the category of representations of $L$ and that of $L$-modules. One would need to argue a bit more rigorously, by defining the morphisms in each one, but we will not do this and take this as an implicit fact.

\subsection{Homomorphisms, Submodules and Quotient Modules}

Throughout this subsection, we denote by $M$ an $L$-module of finite $\C$-dimension.

\begin{boxdefinition}[Lie Submodule]
    An \textbf{$L$-submodule} of $M$ is a sub-vector space $N \leq M$ that is invariant under the action of $L$, ie,
    \begin{align*}
        \forall l \in L, x \in N, \ l \cdot x \in N
    \end{align*}
\end{boxdefinition}



\sorry % Defs of the 3 things above

\subsection{Simple or Irreducibile $L$-Modules}

Throughout this subsection, we denote by $M$ an $L$-module of finite $\C$-dimension.

\begin{boxdefinition}[Irreducibility]\label{Ch2:Def:IrreducibleModule}
    $M$ is \textbf{irreducible} if it has no proper, non-zero submodules.
\end{boxdefinition}
\begin{remark}
    We can define a similar notion of irreducibility of representations, which we can show to be equivalent to the irreducibility of the corresponding module. This is why we choose to call a module that is \textbf{simple}, ie, that satisfies \Cref{Ch2:Def:IrreducibleModule}, `irreducible'.
\end{remark}
Nevertheless, we use the following convention.
\begin{boxconvention}
    We will use the terms `simple' and `irreducible' interchangeably.
\end{boxconvention}

There exist a large number of irreducible modules of any Lie algebra.

\begin{boxexample}\label{Ch2:Eg:1DModuleIrred}
    Any one-dimensional $L$-module is irreducible.
\end{boxexample}

We remind the reader of the famed Jordan-Hölder Theorem from commutative algebra. We will mention that all Lie algebra modules admit composition series. We will combine this existence result with the standard formulation of the Jordan-Hölder Theorem into the following theorem.

\begin{boxtheorem}[Jordan-Hölder]\label{Ch2:Thm:JordanHolder}
    If $M \neq 0$, then there exists a sequence
    \begin{align}
        0 = M_0 \subset M_1 \subset \ldots \subset M_n = M
    \end{align}
    of submodules of $M$ such that $\quotient{M_i}{M_{i-1}}$ is irreducible for all $i$. Furthermore, each $\quotient{M_i}{M_{i-1}}$ is unique up to permutation and isomorphism---that is, they do not depend on the sequence itself, but only on the isomorphism class of $M$.
\end{boxtheorem}

Finally, we mention some useful results on irreducible representations of nilpotent and solvable Lie algebras that follow from Engel's Theorema and Lie's Theorem respectively.

\begin{boxproposition}
    Let $L$ be solvbale. Then, every irreducible $L$-module is one-dimensional over $\C$.
\end{boxproposition}
\begin{proof}
    \sorry 
\end{proof}



\subsection{Semi-Simple $L$-Modules}

We have yet more parallels with the representation theory of finite groups.

\begin{boxdefinition}[Direct Sum]\label{Ch2:Def:ModuleDirectSum}
    The \textbf{direct sum} of two $L$-modules $M$ and $N$ is the $L$-module $M \oplus N$ with the action defined by
    \begin{align}
        \rho(\ell)(m, n) = \parenth{\rho(\ell)(m), \rho(\ell)(n)}
    \end{align}
\end{boxdefinition}

We then have a notion of semi-simplicity for $L$-modules, which is essentially the same as that of any module.

\begin{boxdefinition}[Semi-Simplicity]\label{Ch2:Def:SemiSimpleModule}
    An $L$-module $M$ is \textbf{semi-simple} if it is a direct sum of simple (irreducible) $L$-modules.
\end{boxdefinition}

The following theorem, which is quite hard to prove, explains the reason we define semi-simplicity of Lie algberas the way we do in \Cref{Ch1:Def:SemiSimple}.

\begin{boxtheorem}\label{Ch2:Thm:SemiSimpleLieToModule}
    If $L$ is semi-simple, then every finite-dimensional $L$-module is semi-simple.
\end{boxtheorem}

We do not prove this theorem here, but we will take it for granted going forward.

\begin{boxexample}
    Let $L$ be solvable. \Cref{Ch1:Thm:Lie} (Lie's Theorem) tells us that \sorry % Look at recording of lecture from Oct 24
\end{boxexample}


