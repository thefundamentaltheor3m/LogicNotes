\section{The Semantics of First-Order Logic}

\subsection{First-Order Structures}

We begin by discussing relations and functions of a given arity.

\begin{boxdefinition}[$n$-ary Relation on a Set]
    Suppose $A$ is a set and $n \geq 1$ is a natural number. An \textbf{$n$-ary relation on $A$} is a subset
    \begin{align*}
        \Rb \subseteq \setst{\parenth{a_1, \ldots, a_n}}{a_1, \ldots, a_n \in A}
    \end{align*}
\end{boxdefinition}

We have a similar notion for functions, with the key fact being that $n$-ary functions take in $n$ inputs and return a single output, and all inputs and outputs must come from the set in question.

\begin{boxdefinition}[$n$-ary Function on a Set]
    Given a set $A$, an \textbf{$n$-ary function on $A$} is a function
    \begin{align*}
        \fb: A^n \to A
    \end{align*}
\end{boxdefinition}

We make a subtle distinction between functions and relations in formal and informal language. This is something that will get clearer as we progress.

\begin{boxconvention}
    The reason why we put bars on top of the symbols is to distinguish functions and relations as they appear in formulae from the way that discuss them.
\end{boxconvention}

We have special terms when $n = 1, 2, 3$.

\begin{boxconvention}
    \hfill
    \begin{enumerate}
        \item A $1$-ary relation is commonly called a \textbf{unary relation}.
        \item A $2$-ary relation is commonly called a \textbf{binary relation}.
        \item A $3$-ary relation is commonly called a \textbf{ternary relation}.
    \end{enumerate}
\end{boxconvention}

These notions are not new to us.

\begin{boxexample}[Some Familiar $n$-ary Relations]
    \hfill
    \begin{enumerate}
        \item Equality is a binary relation on any set.
        \item $\leq$ is a binary relation on $\R$.
        \item $\setst{x \in \Z}{x \text{ is even}}$ is a unary relation on $\Z$.
    \end{enumerate}
\end{boxexample}

Admittedly, the fact that the third example is precisely a set is a little unusual to see. This is because in practice, the following convention is used.

\begin{boxconvention}
    Let $\Rb \subseteq A^n$ be a relation on some set $A$. For all $\parenth{a_1, \ldots, a_n} \in A^n$, we write
    \begin{align*}
        \Rbof{a_1, \ldots, a_n}
    \end{align*}
    to mean that $\parenth{a_1, \ldots, a_n} \in \Rb$.
\end{boxconvention}

We are now ready for the most important definition of this chapter.

\begin{boxdefinition}[First-Order Structure]
    A \textbf{first-order structure} is the following data:
    \begin{enumerate}
        \item a non-empty set $A$ called the \textbf{domain} of $\A$.
        
        \item a set of \textbf{relations} on $A$
        \begin{align*}
            \setst{\ol{R_i} \subseteq A^{n_i}}{i \in I}
        \end{align*}

        \item a set of \textbf{functions} on $A$
        \begin{align*}
            \setst{\ol{f_j} : A^{m_j} \to A}{j \in J}
        \end{align*}

        \item a set of \textbf{constants} that are elements of $A$
        \begin{align*}
            \setst{\ol{c_k} \in A}{k \in K}
        \end{align*}
    \end{enumerate}
    where $I, J, K$ are index sets that can be empty.
\end{boxdefinition}

Usually, the index sets of a first-order structure are subsets of $\N$, but in principle, they could be any set. We package the information about the constants and the arity of the functions and relations together in the following manner.

\begin{boxdefinition}[Signature]
    Let $\A$ be a first-order structure. The \textbf{signature} of $\A$ is the information
    \begin{align*}
        \setst{n_i}{i \in I}
        \qquad
        \setst{m_j}{j \in J}
        \qquad
        K
    \end{align*}
    with the respective sets describing the arity of the relations on $A$, the arity of the functions on $A$, and the index set of the constants in $A$.
\end{boxdefinition}

We use the following notation for first-order structures.

\begin{boxconvention}
    For a first-order structure $\A$ given as above, we wil deonte
    \begin{align*}
        \A = \cycl{A; \set{\ol{R_i}}_{i \in I}, \set{\ol{f_j}}_{j \in J}, \set{\ol{c_k}}_{k \in K}}
    \end{align*}
    More generally, we use the notation
    \begin{align*}
        \text{Structure} = \cycl{\text{Domain}; \text{Relations}, \text{Functions}, \text{Constants}}
    \end{align*}
\end{boxconvention}

We have encountered any number of first-order structures so far. Here are a few examples.

The first is a very basic example.
\begin{boxexample}[Orderings]
    We can take $A$ to be one of the sets $\N, \Z, \Q, \R$. We can define a first-order structure on $A$ with only one unary relation---that of ordering---and no functions or constants.
\end{boxexample}
It is important to note that while the sets $\N, \Z, \Q, \R$ all admit richer structures on them, they are not needed to define ordering. We don't even include equality in this description because a formula that contains an equality symbol is not about ordering.

In the next example, we look at an algebraic structure.
\begin{boxexample}[Groups]
    Every group is a first-order structure with the following data.
    \begin{enumerate}
        \item The domain is the set of elements of the group.
        \item The sole relation is the binary relation of equality.
        \item There is a binary function for the group operation and a unary function for inversion.
        \item There is a constant for the identity element.
    \end{enumerate}
\end{boxexample}
We can make a similar definition for rings.

We do not even need to talk about objects that we usually deal with as sets. We can also talk about graphs, which, while defined in terms of sets, are usually studied visually.
\begin{boxexample}[Graphs]
    Graphs (or, more precisely, their vertices), along with two binary relations---equality and adjacency---and no functions or constants, form a first-order structure.
\end{boxexample}

\subsection{First-Order Languages}

\begin{boxdefinition}[First-Order Language]
    A \textbf{first-order language} $\L$ consists of the following data.
    \begin{enumerate}
        \item Index sets $I, J, K$ where $I$ is non-empty but $J$ and $K$ can be empty.
        \item An \textbf{alphabet} of symbols, consisting of
        \begin{enumerate}[noitemsep]
            \item \textbf{Variables} $x_1 \ x_2 \cdots$
            \item \textbf{Connectives} $\neg \ \to$
            \item \textbf{Punctuation} $( \ ) \ ,$
            \item The \textbf{Quantifier} $\forall$
            \item \textbf{Relation symbols} $R_i$ for $i \in I$
            \item \textbf{Function symbols} $f_j$ for $j \in J$
            \item \textbf{Constant symbols} $c_k$ for $k \in K$
        \end{enumerate}
        \item An \textbf{arity} for every relation and function symbol.
    \end{enumerate}
\end{boxdefinition}

% TODO: Finish
