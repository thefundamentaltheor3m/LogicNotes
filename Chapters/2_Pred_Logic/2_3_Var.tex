\section{Variables and the Universal Quantifier}

The purpose of this section is to understand how to work with variables and the universal quantifier in first-order languages. Throughout this section, fix a \fola\ $\L$.

\subsection{Bound and Free Variables}

Consider the formula
\begin{align}
    \psi_1 : \parenth{R_1\of{x_1, x_2} \to \parenth{\forall x_3}R_2\of{x_1, x_3}}
    \label{Ch2:Eq:Free_Bound_Motivation}
\end{align}
where $R_1, R_2$ are relation symbols in $\L$ and $x_1, x_2, x_3$ are variables in $\L$. Intuitively, we do not want $x_3$ to `exist', or be `known' or `accessible', outside of the sub-formula\footnote{When we say sub-formula, we mean exactly what it sounds like: since formulae are built from smaller formulae using connectives and the quantifier, when we say sub-formula, we mean a formula that arose at an intermediate step in this inductive construction of the formula of which it is a sub-formula.} $\parenth{\forall x_3}R_2\of{x_1, x_3}$. We can mae this notion precise.

\begin{boxdefinition}[Scope of a Quantifier]
    Suppose $\phi$ and $\psi$ are $\L$-formulae. If $\parenth{\forall x_i}\phi$ occurs as a sub-formula of $\psi$, we say that $\phi$ is the  \textbf{scope of $\parenth{\forall x_i}$ in $\psi$}.
\end{boxdefinition}
\begin{boxexample}
    In~\eqref{Ch2:Eq:Free_Bound_Motivation}, $R_2\of{x_1, x_3}$ is the scope of the quantifier $\parenth{\forall x_3}$.
\end{boxexample}

We use the existence of a scope to characterise variables.

\begin{boxdefinition}[Bound Variables]
    Let $\psi$ be an $\L$-formula and let $x_j$ be a variable that appears in $\psi$. We say that $x_j$ is \textbf{bound in $\psi$} if it is within the scope of a quantifier.
\end{boxdefinition}

\begin{boxexample}
    In~\eqref{Ch2:Eq:Free_Bound_Motivation}, the variable $x_3$ is bound, because it lies in the scope $R_2\of{x_1, x_3}$ of the quantifier $\parenth{\forall x_3}$.
\end{boxexample}

We also have a special name for variables that are not bound.

\begin{boxdefinition}[Free Variables]
    Let $\psi$ be an $\L$-formula and let $x_j$ be a variable that appears in $\psi$. We say that $x_j$ is \textbf{free in $\psi$} if it is not bound in $\psi$.
\end{boxdefinition}

\begin{boxexample}
    In~\eqref{Ch2:Eq:Free_Bound_Motivation}, the variable $x_1$ is bound, because it does not lie in the scope of any quantifiers, meaning it is not bound.
\end{boxexample}

The same variable can have both free and bound occurrences within a given formula.

\begin{boxexample}
    In the formula
    \begin{align*}
        \parenth{\parenth{\forall x_1}R_1\of{x_1, x_2} \to R_2\of{x_1, x_2}}
    \end{align*}
    the variable $x_1$ is bound in the expression
    \begin{align*}
        \parenth{\forall x_1}\underbrace{R_1\of{x_1, x_2}}_{\text{scope of } \parenth{\forall x_1}}
    \end{align*}
    but free in the expression
    \begin{align*}
        R_2\of{x_1, x_2}
    \end{align*}
\end{boxexample}

We can also show that existentially quantified variables are bound.

\begin{boxlemma}
    Let $\psi$ and $\phi$ be an $\L$-formulae. If $\psi$ contains
    \begin{align*}
        \parenth{\exists x_1}\phi
    \end{align*}
    as a sub-formula, then $x_1$ is bound in $\psi$.
\end{boxlemma}
\begin{proof}
    This follows immediately from the definition of the existential quantifier. \sorry
\end{proof}

The occurrence of free variables inside a formula means that it is very general, as the variables in it are purely formal symbols that we have no machinery to reason with. We give a special term to, and have a special interest in, formulae with no free variables.

\begin{boxdefinition}[Closed Formulae]
    An $\L$-formula with no free variables is called a \textbf{closed formula} or a \textbf{sentence}.
\end{boxdefinition}

Formulae that are not closed can be thought of as being `dependent' on their free variables, in the sense that we would intuitively want to only define a notion of substitution only for free variables. To that end, we adopt some notation.

\begin{boxconvention}
    Let $\psi$ be an $\L$-formula. If $\L$ has free variables $x_1, \ldots, x_n$, then we denote $\psi$ by 
    \begin{align*}
        \psi\of{x_1, \ldots, x_n}
    \end{align*}
    when we wish to underscore the dependence of $\psi$ on $x_1, \ldots, x_n$ or the fact that $x_1, \ldots x_n$ are free in $\psi$ or perform a substitution.
\end{boxconvention}

We now define substitution.

\begin{boxdefinition}[Substitution]\label{Ch2:Def:Substitution}
    Let $t_1, \ldots, t_n$ be terms in $\L$ and $\psi\of{x_1, \ldots, x_n}$ an $\L$-formula with free variables $x_1, \ldots, x_n$\footnote{By the aforementioned convention, we would ordinarily not mention that $x_1, \ldots, x_n$ are free in $\psi$.}. We define the formula obtained by \textbf{substituting the $t_i$ for the $x_i$} to be the $\L$-formula obtained by replacing each occurrence of $x_i$ in $\psi$ with an occurrence of the corresponding $t_i$. We denote this formula
    \begin{align*}
        \psi\of{t_1, \ldots, t_n}
    \end{align*}
\end{boxdefinition}

\subsection{An Analogue of Completeness}

Throughout this subsection, fix an $\L$-formula $\phi$ and an $\L$-structure $\A$.

We begin by mentioning a result about valuations.

\begin{boxtheorem}
    Fix $n \in \N \cup \set{0}$. If $\phi$ has free variables $x_1, \ldots, x_n$ and $\v, \w$ are valuations of $\L$ in $\A$ with $\vof{x_i} = \wof{x_i}$ for all $1 \leq i \leq n$, then
    \begin{align*}
        \vbrac{\phi} = \T \tiff \wbrac{\phi} = \T
    \end{align*}
    In other words, we have $\vbrac{\phi} = \wbrac{\phi}$.
\end{boxtheorem}
\begin{proof}
    Fix valuations $\v$ and $\w$ of $\L$ in $\A$ that agree on all the free variables of $\phi$. We argue that $\vbrac{\phi} = \wbrac{\phi}$ by performing induction on the total number connectives and quantifiers in $\phi$.

    The base case is when $\phi$ has no connectives or quantifiers, ie, when $\phi$ is atomic in terms that contain no quantifiers. In this case, we have that $\phi$ is of the form
    \begin{align*}
        \Rof{t_1, \ldots, t_m}
    \end{align*}
    where $t_1, \ldots, t_m$ are terms that contain no quantifiers. In this case, each $t_i$ is either a constant or a variable. Since $\phi$ is assumed not to contain any quantifiers, the $t_i$ that are variables must be free variables, and the rest must be constants. Since $\v$ and $\w$ agree on all constants by definition and on all free variables by assumption, we must have
    $\vof{t_i} = \wof{t_i}$ for all $i$. Then, \sorry
\end{proof}

This gives us a completeness-like result about closed formulae.

% Main Thm: 2.3.3 (David Ang's notes)
\begin{boxcorollary}
    If $\phi$ is closed, then either $\A \models \phi$ or $\A \models \parenth{\neg \phi}$.
\end{boxcorollary}
\begin{proof}
    If $\phi$ is closed, $\phi$ has no free variables. Then, any two valuations of $\L$ in $\A$ agree vacuously on the set of all free variables of $\phi$. Therefore, there cannot be two distinct valuations such that one satisfies $\phi$ and one does not. In other words, either $\phi$ is true in $\A$ or $\phi$ is not true in $\A$. Equivalently, either $\phi$ or $\parenth{\neg \phi}$ is true in $\A$. 
\end{proof}

\subsection{Understanding the Universal Quantifier}

The purpose of this subsection is to discuss the relationship between the syntax and the semantics of the universal quantifier $\forall$. We have not been too precise so far about what it \textit{means} to write an expression like
\begin{align*}
    \parenth{\forall x_1}\phi
\end{align*}
for some formula $\phi$ with a bound variable $x_1$. We particularly delve into nuances that arise due to the fact that in our strict syntactic definition of $\L$-formulae (\Cref{Ch2:Def:Formula_FO_Logic}, we only allow the quantifier to be succeeded by a \textbf{variable} in $\L$, which must be one of $x_1, x_2, \ldots$. The really confusing thing about this syntactic definition of the universal quantifier is the fact that it is not obvious that this quantifier means ``for all'': instead of saying ``for all variables $x_1$, we have $\phi$'', on a strictly syntactic level, the expression $\parenth{\forall x_1}\phi$ is using the \textit{specific variable} $x_1$.

What we would want---say, in an interactive theorem prover like Lean---is a mechanism to be able to \textit{remove} the quantifier and instead \textit{introduce} a free variable into our formula and label that variable differently from all other free and bound variables in our formula.

In this subsection, we will make this notion precise. Throughout, fix a \fola\ $\L$ and a \fos\ $\A$ in $\L$. Denote the domain of $\A$ by $A$.

We begin by describing what it means for a structure to model a formula under a valuation.

\begin{boxdefinition}
    Let $\psi\of{x_1, \ldots, x_n}$ be an $\L$-formula whose free variables are $x_1, \ldots, x_n$. Let $a_1, \ldots, a_n \in A$ be constants. We say that \textbf{$\A$ models $\psi\of{a_1, \ldots, a_n}$}, denoted
    \begin{align*}
        \A \models \psi\of{a_1, \ldots, a_n}
    \end{align*}
    if $\vbrac{\psi} = \T$ for every valuation $\v$ of $\L$ in $\A$.
\end{boxdefinition}
Note that by \Cref{Ch2:Lemma:Valuation_Existence_Uniqueness}, to prove that $\A \models \psi\of{a_1, \ldots, a_n}$, it suffices to prove that $\vbrac{\psi} = \T$ for \textit{some} valuation $\v$ of $\L$ in $\A$.

We would want to be able to use the quantifier $\forall$ in the way that we are used to using it. In other words, from a hypothesis
\begin{align*}
    \parenth{\forall x_1} \phi\of{x_1}
\end{align*}
we would want to be able to deduce
\begin{align*}
    \phi\of{t}
\end{align*}
for any term $t$ in $\L$. Unfortunately, given that $x_1$ is \textit{itself} a term in $\L$, and $\parenth{\forall x_1} \phi\of{x_1}$ is does not involve some `general variable' but the very concrete variable $x_1$, \underline{this deduction is simply not true!}\footnote{It is at times like this that it is particularly important to distinguish semantics and syntax. When we say ``simply not true'', that is an inherently semantic statement: syntactic reasoning is not a reasoning of \textit{truth}, but one of \textit{provability}. What we are really attempting to do is establish \textit{motivation} for the axioms of the formal deduction system for a first-order language. We will see that in our formal system, we will not allow this type of deduction. This section underscores that it is actually \textit{sensible} to disallow this explicitly, because even in an `intuitive' sense of reasoning, this would not `make sense'.}

\begin{boxcexample}[Indiscriminate specialisation is not a good idea]
    We provide a counterexample that disproves the claim that we can indiscriminately specialise a formula $\parenth{\forall x_1} \phi\of{x_1}$ to \textit{any} $\L$-term $t$ to obtain a true statement $\phi\of{t}$. More precisely, we give
    \begin{itemize}
        \item a first-order language $\L$
        \item an $\L$-structure $\A$
        \item an $\L$-formula $\phi\of{x_1}$ with a free variable $x_1$
        \item an $\L$-term $t$
        \item a valuation $\v$ of $\L$ in $\A$
    \end{itemize}
    such that
    \begin{align*}
        \A \models \parenth{\forall x_1}\phiof{x_1}
    \end{align*}
    but
    \begin{align*}
        \vbrac{\phiof{t}} = \F
    \end{align*}
    Let $\L$ have one binary relation $R$ and one unary relation $S$. Let
    \begin{align*}
        \phiof{x_1} : \parenth{\parenth{\forall x_2}\Rof{x_1, x_2} \to S\of{x_1}}
    \end{align*}
    We can specialise this formula $\phi$ via a \textbf{substitution instance} (cf. \Cref{Ch2:Def:Substitution}). Let $t$ be the term $x_2$. Since \Cref{Ch2:Def:Substitution} imposes \textbf{no conditions} on what can be substituted for a free variable, we can substitute $t$ for $x_1$ in $\phi$. Then, we have that $\phiof{t}$ is the formula
    \begin{align*}
        \parenth{\forall x_2}\Rof{x_2, x_2} \to S\of{x_2}
    \end{align*}
    The problem with this substitution is that the $x_1$ that served as the first argument of $R$ prior to substitution was free, whereas the $x_2$ was bound to the quantifier $\parenth{\forall x_2}$ that preceded $\Rof{x_1, x_2}$. After substitution, both arguments of $R$ are now bound to the quantifier $\parenth{\forall x_2}$. We can see this in the following model. \newline

    Let $\A = \cycl{\N; \le, \parenth{= 0}}$. That is, $\A$ is the $\L$-structure with the following data.
    \begin{itemize}
        \item The domain of $\A$ is $\N$.
        \item The binary relation in $\A$ is the ordering $\le$.
        \item The unary relation on $\A$ is the condition that the argument be equal to zero.
    \end{itemize}
    In this case, we can certainly see that
    \begin{align*}
        \A \models \parenth{\forall x_1}\phiof{x_1}
    \end{align*}
    because for any natural number $x_1$, if $x_1 \leq x_2$ for all natural numbers $x_2$, then $x_1 = 0$.
    \sorry
\end{boxcexample}

\sorry

% Main Thm: 2.3.6 (David Ang's notes)
