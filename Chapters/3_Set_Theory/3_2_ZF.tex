\section{The Zermelo-Fraenkel Axioms}

In this section, we define the \textbf{Zermelo-Fraenkel Axioms} on which most of modern mathematics is built.\footnote{I personally prefer the Calculus of Inductive Constructions, but what do I know...}

We will begin by mentioning what we \textit{mean} be a Zerlemo-Fraenkel Axiom. This is not a formal definition per se, but a convention we adopt for our own convenience.

\begin{boxconvention}
    The Zermelo-Fraenkel Axioms are first-order formulae in some first-order language $\L$ that contains
    \begin{itemize}
        \item Variable symbols representing sets
        \item A binary relation symbol $=$ representing equality
        \item A binary relation symbol $\in$ representing membership
    \end{itemize}
    The Zermelo-Fraenkel Axioms are contained in a set of formulae $\Delta_{\ZF}$ that is modelled by all modern mathematics that does not use the Axiom of Choice. When we discuss the Axiom of Choice, we will consider the set $\Delta_{\ZFC}$ instead, which consists of all elements of $\Delta_{\ZF}$ as well as the Axiom of Choice. All structures in modern mathematics are models of $\Delta_{\ZFC}$.
\end{boxconvention}

We also now 

% Also need to include a convention that says that a statement like ``$A$ is a set in which (some formula in $A$)'' means that $A$ is a model of that statement/that statement holds in $A$ (we defined holding in $A$ in Chapter 2).

We are now ready to begin stating the Zermelo-Fraenkel Axioms.

\begin{baxiom}
    \sorry
\end{baxiom}

\begin{baxiom}
    \sorry
\end{baxiom}

\begin{baxiom}
    \sorry
\end{baxiom}

\begin{baxiom}
    \sorry
\end{baxiom}

\begin{baxiom}
    \sorry
\end{baxiom}

\begin{baxiom}
    \sorry
\end{baxiom}

\sorry

% 3 March, 2025

\subsection{The Axiom of Infinity}

The seventh axiom posits the existence of a notion of infinity. Before we can state it, we need to define a notion of \textit{inductive sets}.

\begin{boxdefinition}[The Successor]\label{Ch3:Def:succ_dg}
    Let $a$ be an arbitrary set. The \textbf{successor} of $a$ is the set
    \begin{align*}
        a\dg := a \cup \set{a}
    \end{align*}
\end{boxdefinition}

The construction of the natural numbers, with which we expect the reader to be familiar, involves applying the successor function repeatedly to the empty set.

We can now define what it means for a set to be inductive. There is an important difference between the definition below and the notion of inductive types in programming languages like Lean that are built using the Calculus of Inductive Constructions: here, we have \textit{specified} a successor \textit{operation} that takes in a \textit{set} and gives out another \textit{set}, whereas when constructing $\N$ in Lean, we have two \textit{constructors}: the distinguished number \texttt{0} and the \textit{function} \texttt{succ~:~$\N \to \N$}

\begin{boxdefinition}[Inductivity]
    A set $A$ is said to be \textbf{inductive} if
    \begin{align*}
        \Ind(A) : \parenth{\parenth{\emptyset \in A} \land \parenth{\forall x}\parenth{\parenth{x \in A} \to \parenth{x\dg \in A}}}
    \end{align*}
    holds in $A$.
\end{boxdefinition}

The idea behind the Axiom of Infinity is that by our informal understanding of infinity, an inductive set cannot be finite.

\begin{baxiom}[The Axiom of Infinity]\label{ZF:Infinity}
    The Axiom of Infinity posits the existence of an inductive set. Ie, it is the first-order formula
    \begin{align*}
        \parenth{\exists A}\parenth{\Ind(A)}
    \end{align*}
\end{baxiom}

\subsection{The Axiom of Replacement}

The primary motivation for this axiom comes from the study of ordinals. In particular, we will see it used in the proof of \sorry, where without the axiom of replacement, the constructions in the argument become dubious. \textbf{It is not a bad idea to read \sorry\ before reading this section, because that is where we find the motivation.} Indeed, in lectures, too, the material on ordinals was convered \textbf{before} the material in this section.

We begin by defining an operation on sets.

\begin{boxdefinition}[Operation on Sets]\label{Ch3:Def:Op_on_Sets}
    We say an $\L$-formula $F\of{x, y, z_1, z_2, z_3, \ldots, z_r}$ is an \textbf{operation on sets} if it satisfies the property that whenever $s_1, \ldots, s_r$ are sets and $a$ is a set, there is a \textit{unique} set $b$ such that $F\of{a, b, s_1, \ldots, s_r}$ holds.
\end{boxdefinition}

We can think of an operation of sets as mapping the set $a$ to the set $b$ given \textbf{parameters} $s_1, \ldots, s_r$. Indeed, the variables $z_1, \ldots, z_r$ are sometimes referred to as the \textbf{parameter variables of $F$}.

We are no stranger to the concept of an operation on sets. Indeed, endofunctors in the category of sets are all operations on sets, though these are not the only examples.

\begin{boxexample}[Familiar Operations on Sets]
     We give two examples, one without parameters and one with one parameter.
     \begin{enumerate}
         \item Define $F\of{a, b}$ to mean `$b$ is the power set of $a$.'
         % \item Define $F\of{a, b}$ to mean `if $a$ is a well-ordered set, then $b$ is an ordinal similar to $a$; else, $b$ is $\emptyset$.'\footnote{This example will make sense once we state the Axiom of Regularity and }
         \item Define $F\of{a, b, s}$, with \underline{parameter $s$}, to mean `$b$ is the set of functions from $a$ to $s$.'
     \end{enumerate}
\end{boxexample}

We are now ready to state the Axiom of Replacement. Technically, we don't have an \textit{axiom} but an \textit{axiom scheme}, which means we have \textit{several axioms indexed by formulae}.

\begin{baxiom}[The Axiom Scheme of Replacemenet]
    Suppose $F\of{x, y, z_1, \ldots, z_r}$ is an operation on sets. Let $A, s_1, \ldots, s_r$ be sets. Then, there is a set $B$ given by
    \begin{align*}
        B = \setst{b}{\parenth{\exists a}\parenth{\parenth{a \in A} \land F\of{a, b, s_1, \ldots, s_r}}}
    \end{align*}
\end{baxiom}

This axiom scheme gets its name from the fact that $B$ is constructed by ``replacing'' every $a \in A$ with the corresponding set $b$ given by the operation $F$ and parameters $s_1, \ldots, s_r$.