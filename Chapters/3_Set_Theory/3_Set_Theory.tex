\chapter{Set Theory}
\thispagestyle{empty}

We now come to the final component of this module: the theory of sets. The purpose of this chapter is to motivate and discuss the results that have led to the development of the modern theory of sets.

The structure of this chapter will be as follows. We will begin by discussing naïve set theory and the motivations for the axiomatisation of set theory by Zermelo and Fraenkel. We will then state and explain some of the Zermelo-Fraenkel axioms. We will then develop some theory and prove them using \textit{naïve} set-theoretic arguments, and mention what breaks down in the axiomatic sense when we use the axioms developed up to that point. This will motivate the addition of an appropriate axiom to make the naïve arguments valid in the axiomatic sense, all the while steering clear of the paradoxes that come with a non-axiomatic treatment of set theory.

\section{Naïve Set Theory}

We begin by discussing the reasons why set theory was axiomatised, as well as a few basic axioms that will already allow us to do a non-trivial amount of mathematics, following which we will motivate, state and explore the more nuanced axioms.

\subsection{Familiar Set-Theoretic Constructions}

There are five important notions from naïve set theory with which we are already quite familiar:
\begin{enumerate}
    \item Extensionality
    \item The Natural Numbers
    \item \sorry % Lectures but not David Ang
    \item Ordered Pairs
    \item Functions
\end{enumerate}





% Naïve Set Theory

\subsection{The Concept of Cardinality}

Informally, the cardinality of a set is the number of elements it has. We briefly explore this notion in more detail in this section.

\begin{boxdefinition}[Equinumerosity]
    We say sets $A$ and $B$ are \textbf{equinumerous}, or that $A$ and $B$ \textbf{have the same cardniality}, if there is a bijection between $A$ and $B$.
\end{boxdefinition}

There are two ways by which we will denote this.

\begin{boxconvention}
    If $A$ and $B$ are equinumerous, we write
    \begin{align*}
        A \approx B \text{ or } \abs{A} = \abs{B}
    \end{align*}
    (regardless of the actual cardinalities of $A$ and $B$).
\end{boxconvention}

We also define finiteness and countability.

\begin{boxdefinition}[Finiteness]
    A set $A$ is \textbf{finite} if it is equinumerous with some element of $\N$.
\end{boxdefinition}

The idea is that we view any element $n \in \N$ as a set with $b$ elements. Ie, for all $n \in \N$,
\begin{align*}
    n = \set{0, 1, \ldots, n-1}
\end{align*}
with $0 = \emptyset$, $1 = \set{\emptyset}$, and so on.

Admittedly, this is not the soundest thing to do, since we have yet to define things like $\emptyset$, $\cup$, and $\N$. However, since the purpose of this section is to offer motivation, we do not make anything of it. We now define countable infiniteness.

\begin{boxdefinition}[Countable Infiniteness]
    A set $A$ is \textbf{countably infinite} if it is equinumerous with all of $\N$.
\end{boxdefinition}

We adopt the following definition for countability.

\begin{boxdefinition}[Countability and Uncountability]
    A set $A$ is \textbf{countable} if it is either finite or countably infinite; similarly, we say $A$ \textbf{uncountable} if it is neither (ie, not countable).
\end{boxdefinition}

We have many conditions for countability. We recall some of them.

\begin{boxproposition}\label{Ch3:Prop:Countable_Conditions}
    Let $A$ and $B$ be countable sets.
    \begin{enumerate}[label = \normalfont \arabic*.]
        \item Every subset of $A$ (and of $B$) is countable.
        \item $A \times B$ is countable.
        \item $A \cup B$ is countable.
        \item $A \sqcup B$ is countable.
    \end{enumerate}
\end{boxproposition}

The above basic facts do not warrant proving in this module.

We even have a somewhat surprising result about countable unions of countable sets.

\begin{boxproposition}
    Let $A_0, A_1, A_2, \ldots$ all be countable sets. Then, the countable union
    \begin{align*}
        \bigcup_{n \in \N} A_n
    \end{align*}
    is countable.
\end{boxproposition}

We do not prove this fact either, but we do mention that \textbf{every proof of it uses the Axiom of Chocie\todo{add reference}}.

\begin{boxexample}
    
\end{boxexample}



% Include a word on Russel's Paradox
\section{The Zermelo-Fraenkel Axioms}

In this section, we define the \textbf{Zermelo-Fraenkel Axioms} on which most of modern mathematics is built.\footnote{I personally prefer the Calculus of Inductive Constructions, but what do I know...}

We will begin by mentioning what we \textit{mean} be a Zerlemo-Fraenkel Axiom. This is not a formal definition per se, but a convention we adopt for our own convenience.

\begin{boxconvention}
    The Zermelo-Fraenkel Axioms are first-order formulae in some first-order language $\L$ that contains
    \begin{itemize}
        \item Variable symbols representing sets
        \item A binary relation symbol $=$ representing equality
        \item A binary relation symbol $\in$ representing membership
    \end{itemize}
    The Zermelo-Fraenkel Axioms are contained in a set of formulae $\Delta_{\ZF}$ that is modelled by all modern mathematics that does not use the Axiom of Choice. When we discuss the Axiom of Choice, we will consider the set $\Delta_{\ZFC}$ instead, which consists of all elements of $\Delta_{\ZF}$ as well as the Axiom of Choice. All structures in modern mathematics are models of $\Delta_{\ZFC}$.
\end{boxconvention}

We also now 

% Also need to include a convention that says that a statement like ``$A$ is a set in which (some formula in $A$)'' means that $A$ is a model of that statement/that statement holds in $A$ (we defined holding in $A$ in Chapter 2).

We are now ready to begin stating the Zermelo-Fraenkel Axioms.

\begin{baxiom}
    \sorry
\end{baxiom}

\begin{baxiom}
    \sorry
\end{baxiom}

\begin{baxiom}
    \sorry
\end{baxiom}

\begin{baxiom}
    \sorry
\end{baxiom}

\begin{baxiom}
    \sorry
\end{baxiom}

\begin{baxiom}
    \sorry
\end{baxiom}

\sorry

% 3 March, 2025

\subsection{The Axiom of Infinity}

The seventh axiom posits the existence of a notion of infinity. Before we can state it, we need to define a notion of \textit{inductive sets}.

\begin{boxdefinition}[The Successor]\label{Ch3:Def:succ_dg}
    Let $a$ be an arbitrary set. The \textbf{successor} of $a$ is the set
    \begin{align*}
        a\dg := a \cup \set{a}
    \end{align*}
\end{boxdefinition}

The construction of the natural numbers, with which we expect the reader to be familiar, involves applying the successor function repeatedly to the empty set.

We can now define what it means for a set to be inductive. There is an important difference between the definition below and the notion of inductive types in programming languages like Lean that are built using the Calculus of Inductive Constructions: here, we have \textit{specified} a successor \textit{operation} that takes in a \textit{set} and gives out another \textit{set}, whereas when constructing $\N$ in Lean, we have two \textit{constructors}: the distinguished number \texttt{0} and the \textit{function} \texttt{succ~:~$\N \to \N$}

\begin{boxdefinition}[Inductivity]
    A set $A$ is said to be \textbf{inductive} if
    \begin{align*}
        \Ind(A) : \parenth{\parenth{\emptyset \in A} \land \parenth{\forall x}\parenth{\parenth{x \in A} \to \parenth{x\dg \in A}}}
    \end{align*}
    holds in $A$.
\end{boxdefinition}

The idea behind the Axiom of Infinity is that by our informal understanding of infinity, an inductive set cannot be finite.

\begin{baxiom}[The Axiom of Infinity]\label{ZF:Infinity}
    The Axiom of Infinity posits the existence of an inductive set. Ie, it is the first-order formula
    \begin{align*}
        \parenth{\exists A}\parenth{\Ind(A)}
    \end{align*}
\end{baxiom}

\subsection{The Axiom of Replacement}

The primary motivation for this axiom comes from the study of ordinals. In particular, we will see it used in the proof of \sorry, where without the axiom of replacement, the constructions in the argument become dubious. \textbf{It is not a bad idea to read \sorry\ before reading this section, because that is where we find the motivation.} Indeed, in lectures, too, the material on ordinals was convered \textbf{before} the material in this section.

We begin by defining an operation on sets.

\begin{boxdefinition}[Operation on Sets]
    We say an $\L$-formula $F\of{x, y, z_1, z_2, z_3, \ldots, z_r}$ is an \textbf{operation on sets} if it satisfies the property that whenever $s_1, \ldots, s_r$ are sets and $a$ is a set, there is a \textit{unique} set $b$ such that $F\of{a, b, s_1, \ldots, s_r}$ holds.
\end{boxdefinition}

We can think of an operation of sets as mapping the set $a$ to the set $b$ given \textbf{parameters} $s_1, \ldots, s_r$. Indeed, the variables $z_1, \ldots, z_r$ are sometimes referred to as the \textbf{parameter variables of $F$}.

We are no stranger to the concept of an operation on sets. Indeed, endofunctors in the category of sets are all operations on sets, though these are not the only examples.

\begin{boxexample}[Familiar Operations on Sets]
     We give two examples, one without parameters and one with one parameter.
     \begin{enumerate}
         \item Define $F\of{a, b}$ to mean `$b$ is the power set of $a$.'
         % \item Define $F\of{a, b}$ to mean `if $a$ is a well-ordered set, then $b$ is an ordinal similar to $a$; else, $b$ is $\emptyset$.'\footnote{This example will make sense once we state the Axiom of Regularity and }
         \item Define $F\of{a, b, s}$, with \underline{parameter $s$}, to mean `$b$ is the set of functions from $a$ to $s$.'
     \end{enumerate}
\end{boxexample}

We are now ready to state the Axiom of Replacement. Technically, we don't have an \textit{axiom} but an \textit{axiom scheme}, which means we have \textit{several axioms indexed by formulae}.

\begin{baxiom}[The Axiom Scheme of Replacemenet]
    Suppose $F\of{x, y, z_1, \ldots, z_r}$ is an operation on sets. Let $A, s_1, \ldots, s_r$ be sets. Then, there is a set $B$ given by
    \begin{align*}
        B = \setst{b}{\parenth{\exists a}\parenth{\parenth{a \in A} \land F\of{a, b, s_1, \ldots, s_r}}}
    \end{align*}
\end{baxiom}

This axiom scheme gets its name from the fact that $B$ is constructed by ``replacing'' every $a \in A$ with the corresponding set $b$ given by the operation $F$ and parameters $s_1, \ldots, s_r$.
\section{Well-Ordered Sets}

\subsection{Products and Sums}

\subsection{Segments}
\section{Introduction to Ordinals}

Broadly speaking, ordinals are nice well-ordered sets that mimic the properties of the natural numbers. The idea is to try and replicate the property in $\N$---at least, when $\N$ is defined using the successor $\dg$ (cf. \Cref{Ch3:Def:succ_dg})---that the ordering comes from both membership and inclusion.

\subsection{Transitive Sets}

We begin by defining the notion of transitive sets.

\begin{boxdefinition}[Transitivity]
    A set $X$ is \textbf{transitive} if every element of $X$ is also a subset of $X$, that is, if
    \begin{align*}
        \parenth{\forall y}\parenth{\parenth{y \subset x} \to \parenth{y \in X}}
    \end{align*}
    holds in $X$.
    % NOTE TO SELF: Include convention for \subset symbol
\end{boxdefinition}

We have examples of transitive sets that come from the set of natural numbers.

\begin{boxexample}\label{Ch3:Eg:3_trans}
    The set $3 = \set{0, 1, 2} = \set{\emptyset, \set{\emptyset}, \set{\emptyset, \set{\emptyset}}}$ is transitive. For example, $2 \in 3$ and $2 \subset 3$ both hold.
\end{boxexample}

However, not all subsets of the naturals are transitive.

\begin{boxnexample}
    $\set{0, 2}$ is not a transitive set. $2 \in \set{0, 2}$ but $2 \not\subset \set{0, 2}$.
\end{boxnexample}

The reason why \Cref{Ch3:Eg:3_trans} works is the following.

\begin{boxlemma}\label{Ch2:Lemma:succ_trans}
    If $X$ is a transitive set, then so is $X\dg = X \cup \set{X}$.
\end{boxlemma}
\begin{proof}
    \sorry
\end{proof}

It does, of course, remain to show that $2$ (and by recursion, $1$ and $0$) are all transitive to show that $3$ is, indeed, transitive, but this is easily done. In fact, we can go one step further.

\begin{boxtheorem}\label{Ch3:Thm:Nat_trans}
    $\N$ is transitive.
\end{boxtheorem}
\begin{proof}
    \sorry % Induction - but use the definition of \N instead of induction as we know it!!!!!!!!
\end{proof}

We are now ready to define ordinals.

\subsection{Ordinals: The Fundamentals}

The idea is for an ordinal to be a transitive set in which the ordering and membership relations are related. This is true in the natural numbers, for example, provided we define the successor function the way we have defined the $\dg$ operation (cf. \Cref{Ch3:Def:succ_dg}).

\begin{boxdefinition}[Ordinal]\label{Ch3:Def:Ordinal}
    A set $\alpha$ is an \textbf{ordinal} if
    \begin{enumerate}
        \item $\alpha$ is a transitive set
        \item The relation $<$ on $\alpha$ given by
        \begin{align*}
            x < y \iff x \in y
        \end{align*}
        for all $x, y \in \alpha$ is a strict well-ordering on $\alpha$.
    \end{enumerate}
\end{boxdefinition}

Here, we pause to make an important observation. In \Cref{Ch3:Def:Ordinal}, we \textbf{explicitly disallow $\alpha \in \alpha$}. The reason for this is that if we had $\alpha \in \alpha$, we would have $\alpha < \alpha$, which contradicts strictness. It is therefore supremely important that the strictness of the well-ordering be included in the definition of an ordinal.

We now relate ordinals and the successor $\dg$.

\begin{boxlemma}
    If $\alpha$ is an ordinal, then $\alpha\dg = \alpha \cup \set{\alpha}$ is also an ordinal.
\end{boxlemma}
\begin{proof}
    \sorry
\end{proof}

We can, in fact, show that $\N$ is an ordinal. However, we are not yet ready to do this. We need more machinery.

We have the following properties of ordinals that follow from the definition.

\begin{boxproposition}[Relationships between Ordinals]
    Let $\alpha$ and $\beta$ be sets.
    \begin{enumerate}
        \item If $\alpha$ is an ordinal and $\beta \in \alpha$, then $\beta$ is an ordinal.
        \item If $\alpha$ and $\beta$ are both ordinals and $\alpha \subsetneq \beta$, then $\alpha \in \beta$.
    \end{enumerate}
\end{boxproposition}
\begin{proof}\hfill
    \begin{enumerate}
        \item \sorry

        \item \sorry
    \end{enumerate}
\end{proof}

It turns out the above properties allow us to compare ordinals and define a notion of ordering between them.

\subsection{Ordering Ordinals}

While there are too many ordinals for there to be a \textit{set} of all ordinals, we can reason collectively about the \textbf{class of all ordinals}, where the word ``class'' is used in the category theoretic sense.  In fact, one can show that this class is well-ordered.


\section{The Theory of Cardinals}

\subsection{The Axiom of Choice}

\sorry

\begin{boxlemma}
    Assume all the ZFC Axioms, that is, \sorry. Suppose $A$ and $B$ are non-empty sets. Then, there is an injective function from $A$ to $B$ if and only if there is a surjective function from $B$ to $A$.
\end{boxlemma}
\begin{proof}
    We need the Axiom of Choice to construct an injection given a surjection, but we do not need it to construct a surjection given an injection.

    \begin{description}
        \item[$\parenth{\implies}$]
        Assume that there exists an injection $f : A \inj B$. Since $A$ is non-empty, there exists some element $a_0 \in A$. Define $h : B \to A$ in the following manner:
        \begin{align*}
            \forall b \in B, \qquad
            h(b) =
            \begin{cases}
                f\inv(b) &\text{ if } b \in f(A) \\
                a_0 &\text{ if } b \notin f(A)
            \end{cases}
        \end{align*}
        where $f\inv(b)$ is the unique $a \in A$ such that $f(a) = b$, as guaranteed by the injectivity of $f$. It is easily seen that $h$ is surjective, because every $a \in A$ is the image of $f(a)$ in $h$.

        \item[$\parenth{\impliedby}$] \sorry
    \end{description}
\end{proof}

\subsection{The Notion of Cardinality}

Throughout this subsection, assume \textbf{all} the ZFC Axioms.

We remarked, at the beginning of our discussion of ordinals, that the cardinal numbers are the numbers of size and the ordinal numbers are the numbers of ordering. In this section, we will discuss what the `numbers of size' actually represent.

We begin by defining our primary object of study in this section.

\begin{boxdefinition}[Cardinal]
    An ordinal $\alpha$ is a \textbf{cardinal} if for all ordinals $\beta < \alpha$, $\alpha$ is not equinumerous with $\beta$.
\end{boxdefinition}

Informally, cardinals are merely sets with a well-understood notion of cardinality. There are many examples.

\begin{boxexample}[Familiar Cardinals]
    Every natural number is a cardinal. Furthermore, the set of all natural numbers is a cardinal.
\end{boxexample}

We can also see that infinite ordinals do not behave nicely with the notion of cardinality, in the sense that they might be too large.

\begin{boxnexample}
    If $\gamma$ is an infinite ordinal, then $\gamma\dg$ is \textit{not} a cardinal, because $\gamma\dg$ and $\gamma$ are equinumerous, but $\gamma < \gamma\dg$.
\end{boxnexample}

\subsection{The Sequence of Alephs}

In this subsection, we define the cardinals $\aleph_{\alpha}$, indexed by ordinals $\alpha$, using the Principle of Transfinite Recursion (cf. \Cref{Ch3:Thm:Transfinite_Recursion}).

\begin{boxdefinition}[The Sequence of Alephs]\label{Ch3:Def:Alephs}
    Define $\aleph_0 := \omega$, and for all 
    \sorry
\end{boxdefinition}

\sorry

\subsection{Cardinal Arithmetic}

In this section, we essentially generalise the operations of addition, subtraction and exponentiation in $\N$ to arbitrary cardinals.

\begin{boxdefinition}[Cardinal Arithmetic]\label{Ch3:Def:CardArith}
    Suppose $A$ and $B$ are disjoint sets. Let them be cardinals, with $\abs{A} = \alpha$ and $\abs{B} = \beta$ for ordinals $\alpha$ and $\beta$. We define the operations of addition, multiplication and exponentiation in the following manner:
    \sorry
\end{boxdefinition}

\sorry

Below, we give an example where we use \sorry\ result\todo{we really gotta complete our notes} to compute cardinalities.

\begin{boxexample}
    Suppose $X$ is an infinite set. That is, suppose $\abs{X} = \lambda$ for some ordinal $\lambda \geq \omega$. Let $S$ be the set of finite sequences of elements of $X$. That is,
    \begin{align*}
        S = \bigcup_{n \in \Omega}X^n
    \end{align*}
    We claim that the cardinality of $S$ is also $\lambda$.
\end{boxexample}

We have more concrete applications as well.

\begin{boxexample}[Showing that $\dim_{\Q}\of{\R} = \abs{\R}$]
    Consider $\R$ as a $\Q$-vector space. Suppose $X \subseteq \R$ spans $\R$: that is, suppose that for all $r \in \R$, there exists some $s \in \N$, $q_1, \ldots, q_s \in \Q$, and $x_1, \ldots, x_s \in X$ such that
    \begin{align*}
        r = q_1 x_1 + \cdots + q_s x_s
    \end{align*}
    We claim that $\abs{X} = \abs{\R}$. We know already that $\abs{X} \leq \abs{\R}$, because $X \subseteq \R$, so it only remains to show that $\abs{\R} \leq \abs{X}$. We do this by applying transitivity. \newline

    Let $P$ be the set of pairs
    \begin{align*}
        P &= \setst{\Big({\parenth{q_1, \ldots, q_s}, \parenth{x_1, \ldots, x_s}}\Big)}{s \in \N, \ q_i \in \Q, \ x_i \in X} \\
        &\subseteq \parenth{\bigcup_{n \in \omega} \Q^n} \times \parenth{\bigcup_{n \in \omega} X^n}
    \end{align*}
    Then, by \sorry, we have that
    \begin{align*}
        \abs{P} \leq \abs{\Q}\abs{X} = \omega \cdot \abs{X} = \abs{X}
    \end{align*}
    where the $\cdot$ is the notion of multiplication seen in \Cref{Ch3:Def:CardArith}. If we can show that $\abs{\R} \leq \abs{P}$, then we can show that $\abs{\R} \leq \abs{X}$, from which it will follow, as argued earlier, that $\abs{\R} = \abs{X}$. \sorry
\end{boxexample}

\subsection{Zorn's Lemma}

We have already seen the power of the Axiom of Choice. But as the reader might be aware, the mysterious assumption about the nature of mathematics that underlies it takes many equivalent forms, the best known of which are the Axiom of Choice itself, the Well-Ordering Principle, and Zorn's Lemma. Indeed, there is the following well-known mathematical adage:
\begin{quote}
    \textit{The Axiom of Choice is obviously true; the Well-Ordering Principle is obviously false; and as for Zorn's Lemma, who can say?}
\end{quote}
In this subsection, we explore Zorn's Lemma and some of its consequences. Most readers would have encountered Zorn's Lemma when studying other areas of mathematics---particularly algebraic ones---but we will not assume any familiarity with these applications of this strange form of the Axiom of Choice.

\sorry % Begin by defining partially ordered sets.

\sorry % Then, state Zorn's Lemma.

We now come to the most important theorem of this section.
\begin{boxtheorem}[An Equivalence between the Axiom of Choice and Zorn's Lemma]
    Assume all the Zermelo-Fraenkel Axioms. Then, the Axiom of Choice is true if and only if Zorn's Lemma is true. That is,
    \begin{align*}
        \ZF \vklem \parenth{\text{Axiom of Choice} \lr \text{Zorn's Lemma}}
    \end{align*}
\end{boxtheorem}
\begin{proof}
    Assume that all the Zermelo-Freankel Axioms hold.
\end{proof}

\sorry % Then, prove that ZF ⊢ Zorn ↔︎ Choice

We now come to an important consequence of Zorn's Lemma: every vector space has a basis. In particular, this applies to infinite-dimensional vector spaces as well.

\begin{boxtheorem}
    Let $F$ be a field and $V$ an $F$-vector space. Then, $V$ has an $F$-basis.
\end{boxtheorem}
\begin{proof}
    Let $\AA$ be the set of all linearly independent subsets of $V$, partially ordered by inclusion $\subseteq$. The idea is to apply Zorn's Lemma to show that $\AA$ contains a maximal element $\B$, which we will then show to be an $F$-basis of $V$.

    In order to apply Zorn's Lemma, we first need to show that if $\parenth{C_i}_{i \in \I} \subseteq \AA$ is a chain, indexed by some set $\I$ ordered by inclusions, then $\bigcup_{i \in \I} C_i \in \AA$ as well.
    
    \sorry
\end{proof}