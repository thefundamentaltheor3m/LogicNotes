\chapter{Set Theory}
\thispagestyle{empty}

We now come to the final component of this module: the theory of sets. The purpose of this chapter is to motivate and discuss the results that have led to the development of the modern theory of sets.

The structure of this chapter will be as follows. We will begin by discussing naïve set theory and the motivations for the axiomatisation of set theory by Zermelo and Fraenkel. We will then state and explain some of the Zermelo-Fraenkel axioms. We will then develop some theory and prove them using \textit{naïve} set-theoretic arguments, and mention what breaks down in the axiomatic sense when we use the axioms developed up to that point. This will motivate the addition of an appropriate axiom to make the naïve arguments valid in the axiomatic sense, all the while steering clear of the paradoxes that come with a non-axiomatic treatment of set theory.

\section{Axiomatic Set Theory}

We begin by discussing the reasons why set theory was axiomatised, as well as a few basic axioms that will already allow us to do a non-trivial amount of mathematics, following which we will motivate, state and explore the more nuanced axioms.

\subsection{Familiar Set-Theoretic Constructions}

There are five important notions from naïve set theory with which we are already quite familiar:
\begin{enumerate}
    \item Extensionality
    \item The Natural Numbers
    \item \sorry % Lectures but not David Ang
    \item Ordered Pairs
    \item Functions
\end{enumerate}





% Naïve Set Theory

\subsection{The Concept of Cardinality}

Informally, the cardinality of a set is the number of elements it has. We briefly explore this notion in more detail in this section.

\begin{boxdefinition}[Equinumerosity]
    We say sets $A$ and $B$ are \textbf{equinumerous}, or that $A$ and $B$ \textbf{have the same cardniality}, if there is a bijection between $A$ and $B$.
\end{boxdefinition}

There are two ways by which we will denote this.

\begin{boxconvention}
    If $A$ and $B$ are equinumerous, we write
    \begin{align*}
        A \approx B \text{ or } \abs{A} = \abs{B}
    \end{align*}
    (regardless of the actual cardinalities of $A$ and $B$).
\end{boxconvention}

We also define finiteness and countability.

\begin{boxdefinition}[Finiteness]
    A set $A$ is \textbf{finite} if it is equinumerous with some element of $\N$.
\end{boxdefinition}

The idea is that we view any element $n \in \N$ as a set with $b$ elements. Ie, for all $n \in \N$,
\begin{align*}
    n = \set{0, 1, \ldots, n-1}
\end{align*}
with $0 = \emptyset$, $1 = \setlength{emptyset}$, and so on.

Admittedly, this is not the soundest thing to do, since we have yet to define things like $\emptyset$, $\cup$, and $\N$. However, since the purpose of this section is to offer motivation, we do not make anything of it.

% Include a word on Russel's Paradox

\subsection{The Zermelo-Fraenkel Axioms}

% \begin{baxiom}\label{trial}
%     Hi
% \end{baxiom}

\section{Ordinals and Ordering}

\subsection{Well-Ordered Sets}

\subsection{An Introduction to Ordinals}

% Definition 3.4.1 to Corollary 3.4.7 in David Ang's notes

\subsection{Ordinals and Well-Ordered Sets}

% Theorem 3.4.8

\subsection{Transfinite Induction}

\subsection{Transfinite Recursion}
\section{The Axiom of Choice}

% State it here

\subsection{The Well-Ordering Principle}

\subsection{An Introduction to Cardinal Arithmetic}

\subsection{Zorn's Lemma}