\section{Axiomatic Set Theory}

We begin by discussing the reasons why set theory was axiomatised, as well as a few basic axioms that will already allow us to do a non-trivial amount of mathematics, following which we will motivate, state and explore the more nuanced axioms.

\subsection{Familiar Set-Theoretic Constructions}

There are five important notions from naïve set theory with which we are already quite familiar:
\begin{enumerate}
    \item Extensionality
    \item The Natural Numbers
    \item \sorry % Lectures but not David Ang
    \item Ordered Pairs
    \item Functions
\end{enumerate}





% Naïve Set Theory

\subsection{The Concept of Cardinality}

Informally, the cardinality of a set is the number of elements it has. We briefly explore this notion in more detail in this section.

\begin{boxdefinition}[Equinumerosity]
    We say sets $A$ and $B$ are \textbf{equinumerous}, or that $A$ and $B$ \textbf{have the same cardniality}, if there is a bijection between $A$ and $B$.
\end{boxdefinition}

There are two ways by which we will denote this.

\begin{boxconvention}
    If $A$ and $B$ are equinumerous, we write
    \begin{align*}
        A \approx B \text{ or } \abs{A} = \abs{B}
    \end{align*}
    (regardless of the actual cardinalities of $A$ and $B$).
\end{boxconvention}

We also define finiteness and countability.

\begin{boxdefinition}[Finiteness]
    A set $A$ is \textbf{finite} if it is equinumerous with some element of $\N$.
\end{boxdefinition}

The idea is that we view any element $n \in \N$ as a set with $b$ elements. Ie, for all $n \in \N$,
\begin{align*}
    n = \set{0, 1, \ldots, n-1}
\end{align*}
with $0 = \emptyset$, $1 = \set{\emptyset}$, and so on.

Admittedly, this is not the soundest thing to do, since we have yet to define things like $\emptyset$, $\cup$, and $\N$. However, since the purpose of this section is to offer motivation, we do not make anything of it. We now define countable infiniteness.

\begin{boxdefinition}[Countable Infiniteness]
    A set $A$ is \textbf{countably infinite} if it is equinumerous with all of $\N$.
\end{boxdefinition}

We adopt the following definition for countability.

\begin{boxdefinition}[Countability and Uncountability]
    A set $A$ is \textbf{countable} if it is either finite or countably infinite; similarly, we say $A$ \textbf{uncountable} if it is neither (ie, not countable).
\end{boxdefinition}

We have many conditions for countability. We recall some of them.

\begin{boxproposition}\label{Ch3:Prop:Countable_Conditions}
    Let $A$ and $B$ be countable sets.
    \begin{enumerate}[label = \normalfont \arabic*.]
        \item Every subset of $A$ (and of $B$) is countable.
        \item $A \times B$ is countable.
        \item $A \cup B$ is countable.
        \item $A \sqcup B$ is countable.
    \end{enumerate}
\end{boxproposition}

The above basic facts do not warrant proving in this module.

We even have a somewhat surprising result about countable unions of countable sets.

\begin{boxproposition}
    Let $A_0, A_1, A_2, \ldots$ all be countable sets. Then, the countable union
    \begin{align*}
        \bigcup_{n \in \N} A_n
    \end{align*}
    is countable.
\end{boxproposition}

We do not prove this fact either, but we do mention that \textbf{every proof of it uses the Axiom of Chocie\todo{add reference}}.

\begin{boxexample}
    
\end{boxexample}



% Include a word on Russel's Paradox

\subsection{The Zermelo-Fraenkel Axioms}

% \begin{baxiom}\label{trial}
%     Hi
% \end{baxiom}
