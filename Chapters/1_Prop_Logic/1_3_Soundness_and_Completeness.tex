\section{Important Properties of $\bL$}

In this section, we prove important properties about $\bL$, with our first major goal being to prove that the theorems in $\bL$ are precisely the tautologies.

\subsection{Propositional Valuations}

\begin{boxdefinition}[Propositional Valuation]
    A \textbf{propositional valuation} $\v$ is an assignment of truth values to propositions $p_1, p_2, \ldots$, so that
    \begin{align*}
        \vof{p_i} \in \set{\T, \F}
    \end{align*}
    for all $i \in \N$.
\end{boxdefinition}

Using the truth table rules by which we defined the connectives $\to$ and $\neg$ of $\bL$ (cf. \Cref{Ch1:Def:Connective}), we can assign a truth value $\vof{\phi}$ to any $\bL$-formula $\phi$.

\begin{boxlemma}[Behaviour of Propositional Valuations]\label{Ch1:Lemma:PropValBhv}
    Let $\phi, \psi$ be $\bL$-formulae. Then,
    \begin{enumerate}
        \item $\vof{\parenth{\neg \phi}} \neq \vof{\phi}$
        \item $\vof{\phi \to \psi} = \F$ if and only if $\vof{\phi} = \T$ and $\vof{\psi} = \F$
    \end{enumerate}
\end{boxlemma}

We do not prove these results.

We can now be precise about what a tautology is.

\begin{boxdefinition}[Tautology]
    An $\bL$-formula $\phi$ is a \textbf{tautology} if $\vof{\phi} = \T$ for all propositional valuations $\v$.
\end{boxdefinition}

We are now ready to prove that $\bL$ is sound.

\subsection{Soundness}

\begin{boxdefinition}[Soundness]
    We say a formal system is \textbf{sound} if every theorem in it is a tautology.
\end{boxdefinition}

Now, the much-awaited result.

\begin{boxtheorem}[Soundness of $\bL$]\label{Ch1:Thm:LSoundness}
    The formal system $\bL$ of propositional logic is sound.
\end{boxtheorem}
\begin{proof}
    Let $\phi$ be a theorem in $\bL$. We prove that $\phi$ is a tautology by performing induction on the length of $\phi$. The base case involves proving that the axioms~\ref{Ch1:Def:PLAxiom:1}-\ref{Ch1:Def:PLAxiom:3} are tautologies, and the inductive case involves proving that the modus ponens deduction rule~\ref{Ch1:Def:PLDedRule:MP} preserves tautologies.

    While these proofs can be done manually, there is a rather clever trick that can be used in the case of~\ref{Ch1:Def:PLAxiom:2}. This is given in \cite[Proof of 1.3.1]{LecNotes2018}. \sorry % TODO: include proof here
\end{proof}

We have a more general version of this result.

\begin{boxtheorem}[A Generalisation of Soundness]
    Let $\Gamma$ be a set of $\bL$-formulae, and let $\phi$ be an $\bL$-formula such that $\Gamma \vdash_{\bL} \phi$. If $\vof{\psi} = \T$ for all $\psi \in \Gamma$, then $\vof{\phi} = \T$.
\end{boxtheorem}
\begin{proof}
    The proof is actually identical to that of \Cref{Ch1:Thm:LSoundness}, but with $\Gamma$ being a part of the base case. In this case, the proof follows from the assumption that $\v$ is true on $\Gamma$.
\end{proof}

Now that we have shown that $\bL$ is sound, we show that $\bL$ has other important properties that will enable us to reason in $\bL$ the way we would like to.

\subsection{Consistency}

For the remainder of this section, fix a set $\Gamma$ of $\bL$-formulae.

\begin{boxdefinition}[Consistency]\label{Ch1:Def:Consistent}
    $\Gamma$ is \textbf{consistent} if there is no $\bL$-formula $\phi$ such that $\Gamma \vdash_{\bL} \phi$ and $\Gamma \vdash_{\bL} \parenth{\neg \phi}$.
\end{boxdefinition}

Soundness tells us something about the consistency of the empty set in $\bL$, which we will refer to more generally as the \textbf{consistency of $\bL$}.

\begin{boxtheorem}[Consistency of $\bL$]
    There is no $\bL$-formula $\phi$ such that $\vdash_{\bL} \phi$ and $\vdash_{\bL} \parenth{\neg \phi}$.
\end{boxtheorem}
\begin{proof}
    This follows from \Cref{Ch1:Thm:LSoundness}: if such an $\bL$-formula $\phi$ did exist, both $\phi$ and $\parenth{\neg \phi}$ would need to be tautologies, violating \Cref{Ch1:Lemma:PropValBhv}. Therefore, such an $\bL$-formula cannot exist.
\end{proof}

We have a more general result.

\begin{boxproposition}\label{Ch1:Prop:ConsistencyUnion}
    Suppose $\Gamma$ is a consistent set of $\bL$-formulae. Let $\phi$ be an $\bL$-formula that is \emph{not} a consequence of $\Gamma$. Then, $\Gamma \cup \set{\parenth{\neg \phi}}$ is consistent.
\end{boxproposition}
\begin{proof}
    Suppose that $\Gamma \cup \set{\parenth{\ neg \phi}}$ is not consistent. Then, there exists a formula $\psi$ such that
    \begin{align}
        \Gamma \cup \set{\parenth{\neg \phi}} \vdash_{\bL} \psi
        \label{Ch1:Eq:ConsistencyProofEq1} \\
        \Gamma \cup \set{\parenth{\neg \phi}} \vdash_{\bL} \parenth{\neg \psi}
        \label{Ch1:Eq:ConsistencyProofEq2}
    \end{align}
    We can apply the Deduction Theorem (\Cref{Ch1:Thm:DeductionThm}) to~\eqref{Ch1:Eq:ConsistencyProofEq2} to deduce that
    \begin{align}
        \Gamma \vdash_{\bL} \parenth{\parenth{\neg \phi} \to \parenth{\neg \psi}}
        \label{Ch1:Eq:ConsistencyProofEq3}
    \end{align}
    Then, applying~\ref{Ch1:Def:PLAxiom:3} to~\eqref{Ch1:Eq:ConsistencyProofEq3}, we have
    \begin{align}
        \Gamma \vdash_{\bL} \parenth{\psi \to \phi}
        \label{Ch1:Eq:ConsistencyProofEq4}
    \end{align}
    In similar fashion, we can apply the Deduction Theorem followed by~\ref{Ch1:Def:PLAxiom:3} to~\eqref{Ch1:Eq:ConsistencyProofEq1} to deduce that
    \begin{align}
        \Gamma \vdash_{\bL} \parenth{\parenth{\neg \psi} \to \phi}
        \label{Ch1:Eq:ConsistencyProofEq5}
    \end{align}
    \sorry % From this, we can conclude that $\Gamma \vdash_{\bL} \phi$, contradicting our assumption. TODO: include reference to something that allows us to do this.
\end{proof}

\subsection{Completeness}

We begin by making an observation about $\bL$: in general, it is not true that for any given $\bL$-formula, either it or its negation is a theorem.

\begin{boxdefinition}[Completeness]\label{Ch1:Def:Complete}
    Let $\Gamma$ be a consistent set of $\bL$-formulae. We say $\Gamma$ is \textbf{complete} if for all $\bL$-formulae $\phi$, either $\Gamma \vdash_{\bL} \phi$ or $\Gamma \vdash_{\bL} \parenth{\neg \phi}$.
\end{boxdefinition}

It turns out that all consistent sets of $\bL$-formulae are contained in complete sets of formulae.

\begin{boxtheorem}[Lindenbaum's Lemma]\label{Ch1:Thm:Lindenbaum}
    Suppose $\Gamma$ is a consistent set of $\bL$-formulae. Then, there exists a complete set of $\bL$-formulae $\Gamma^*$ such that $\Gamma \subseteq \Gamma^*$.
\end{boxtheorem}
\begin{proof}
    The idea is to construct $\Gamma^*$ by brute force. We know that the alphabet of $\bL$ is countable. Therefore, so is the set of all possible $\bL$-formulae. We can list them as $\phi_0, \phi_1, \ldots$. We perform the fomllowing inductive construction: let $\Gamma_0 := \Gamma$ and for all $n \geq 0$, define
    \begin{align}
        \Gamma_{n + 1} :=
        \begin{cases}
            \Gamma_n & \text{ if } \Gamma_n \vdash_{\bL} \phi_n \\
            \Gamma_{n} \cup \set{\parenth{\neg \phi_n}} & \text{ if } \Gamma_n \not\vdash_{\bL} \phi_n
        \end{cases}
        \label{Ch1:Eq:LindenbaumConstruction}
    \end{align}
    One can perform induction on $n$ to show that for all $n \geq 0$,
    \begin{itemize}[noitemsep]
        \item $\Gamma_n \subseteq \Gamma_{n + 1}$ (using purely~\eqref{Ch1:Eq:LindenbaumConstruction})
        \item $\Gamma_n$ is consistent (using \Cref{Ch1:Prop:ConsistencyUnion})
    \end{itemize}
    Define
    \begin{align*}
        \Gamma^* := \bigcup_{n = 0}^{\infty} \Gamma_n
    \end{align*}
    Before we can show that $\Gamma^*$ is complete, we need to show that it is consistent: see \Cref{Ch1:Def:Complete}. To that end, suppose there exists some $\bL$-formula $\phi$ such that $\Gamma^* \vdash_{\bL} \phi$ and $\Gamma^* \vdash_{\bL} \parenth{\neg \phi}$. This means there is a finite sequence of deductions from $\Gamma^*$ and the axioms of $\bL$ using only~\ref{Ch1:Def:PLDedRule:MP} that ends in $\phi$ and another one that ends in $\parenth{\neg \phi}$. There exist $i, j \in \N$ such that these sequences are contained in $\Gamma_i$ and $\Gamma_j$ respectively, because the $\Gamma_n$s represent all possible deductions that can be made from $\Gamma$ (and the axioms). Since we have a chain of inclusions, we have either $\Gamma_i \subseteq \Gamma_j$ or $\Gamma_j \subseteq \Gamma_i$. In either case, we have a contradiction, because $\Gamma_i$ and $\Gamma_j$ are both consistent, but we would be able to deduce both $\phi$ and $\parenth{\neg \phi}$ from one of them. Therefore, $\Gamma^*$ is consistent too.

    Finally, we show that $\Gamma^*$ is complete. Let $\phi$ be a formula. By the enumeration above, we know that $\phi = \phi_n$ for some $n \in \N$. We know either $\Gamma_n \vdash_{\bL} \phi_n$ or $\Gamma_n \not\vdash_{\bL} \phi_n$. If the former is true, we have that $\Gamma^* \vdash_{\bL} \phi_n$ and we are done. Else, by our construction in~\eqref{Ch1:Eq:LindenbaumConstruction}, we have that $\Gamma_{n+1} \vdash_{\bL} \parenth{\neg \phi}$, and therefore, $\Gamma^* \vdash_{\bL} \parenth{\neg \phi}$. Therefore, from $\Gamma^*$, we can deduce either $\phi$ or $\parenth{\neg \phi}$, making $\Gamma^*$ complete, as required.
\end{proof}

Going forward, we will adopt the following notation.
\begin{boxconvention}
    Let $\Gamma$ be a consistent set of $\bL$-formulae. We will denote by $\Gamma^*$ the complete set of $\bL$-formulae containing $\Gamma$ as given in \Cref{Ch1:Thm:Lindenbaum}.
\end{boxconvention}

It turns out that the construction allows us to prove the existence of an important kind of valuation.

\begin{boxproposition}
    Let $\Gamma$ be a consistent set of $\bL$-formulae. Then, there exists a propositional valuation $\v$ such that
    \begin{align*}
        \vof{\phi} = \T \text{ if and only if } \Gamma^* \vdash_{\bL} \phi
    \end{align*}
\end{boxproposition}
\begin{proof}
    \sorry
\end{proof}


