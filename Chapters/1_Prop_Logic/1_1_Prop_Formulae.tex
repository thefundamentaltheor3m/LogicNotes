\section{Propositional Formulae}

Broadly speaking, the study of propositional logic involves studying its two major components: \textbf{syntax} and \textbf{semantics}. While the most formal approach to the study of propositional logic is to study them in that order, in this module, we study semantics before syntax, because while syntax must precede semantics, semantics can serve as motivation for the syntactic choices we make when defining a `formal system' for propositional logic (see \Cref{Ch1:Sec:1_2}). In this section, we will study the semantics of propositional logic.

\subsection{Propositions and Connectives}

We begin by defining the notion of a proposition.

\begin{boxdefinition}[Proposition]
    A \textbf{proposition} is a statement that is either true or false.
\end{boxdefinition}

\begin{boxconvention}
    We will denote the state of being \textbf{true} by $\T$ and that of being \textbf{false} by $\F$.
\end{boxconvention}

Propositions can be connected to each other using tools known as \textbf{connectives}. These can be thought of as \textbf{truth table rules}.

\begin{boxconvention}
    Before we define the actual connectives we shall use, we list them down, along with notation.
    \begin{enumerate}[noitemsep]
        \item Conjunction ($\land$)
        \item Disjunction ($\lor$)
        \item Negation ($\neg$)
        \item Implication ($\to$)
        \item The Biconditional ($\lr$)
    \end{enumerate}
    In particular, we will only use the $\implies$ and $\iff$ symbols when reasoning \textbf{informally}. For \textbf{formal} use, we will stick to the $\to$ and $\lr$ symbols. In more precise terms, we will use $\implies$ and $\iff$ when reasoning \textbf{about} the language we are constructing, whereas we will use $\to$ and $\lr$ when reasoning \textbf{within} the language. As we shall see, it will be of paramount importance to distinguish between these two modes of reasoning.
\end{boxconvention}

We define them exhaustively as follows.
\begin{boxdefinition}[Connectives]\label{Ch1:Def:Connective}
    Let $p$ and $q$ be true/false variables. We define each of the connectives listed above to take on truth values depending on those of $p$ and $q$ as follows.
    \begin{table}[H]
        \centering
        \begin{tabular}{cc|cccccc}
             $p$ & $q$ & $\parenth{\neg p}$ & $\parenth{\neg q}$ & $\parenth{p \land q}$ & $\parenth{p \lor q}$ & $\parenth{p \to q}$ & $\parenth{p \lr q}$ \\
             \hline
             $\T$ & $\T$ & $\F$ & $\F$ & $\T$ & $\T$ & $\T$ & $\T$ \\
             $\T$ & $\F$ & $\F$ & $\T$ & $\F$ & $\T$ & $\F$ & $\F$ \\
             $\F$ & $\T$ & $\T$ & $\F$ & $\F$ & $\T$ & $\T$ & $\F$ \\
             $\F$ & $\F$ & $\T$ & $\T$ & $\F$ & $\F$ & $\T$ & $\T$
        \end{tabular}
        \label{tab:my_label}
    \end{table}
\end{boxdefinition}

We are now ready to define the main object of study in this section: propositional formulae.

\begin{boxdefinition}[Propositional Formula]\label{Ch1:Def:PropFormula}
    A \textbf{propositional formula} is obtained from propositional variables and connectives via the following rules:
    \begin{enumerate}[label=\normalfont (\roman*), noitemsep]
        \item Any propositional variable is a propositional formula.
        \item If $\phi$ and $\psi$ are formulae, then so are $\parenth{\neg \phi}$, $\parenth{\neg \psi}$, $\parenth{\phi \land \psi}$, $\parenth{\phi \lor \psi}$, $\parenth{\phi \to \psi}$, $\parenth{\psi \to \phi}$, and $\parenth{\phi \lr \psi}$.
        \item Any formula arises in this manner after a finite number of steps.
    \end{enumerate}
\end{boxdefinition}

What this means is that a propositional formula is a string of symbols consisting of
\begin{enumerate}[noitemsep]
    \item variables that take on true/false values,
    \item connectors that express the relationship between these variables, and
    \item parentheses/brackets that separate formulae within formulae and specify the order in which they must be evaluated when the constituent variables are assigned specific values.
\end{enumerate}
In particular, every propositional formula is either a propositional variable or is built from `shorter' formulae, where by `shorter' we mean consisting of fewer symbols.

\begin{boxconvention}
    Throughout this module, we will adopt two important conventions when dealing with propositional formulae.
    \begin{enumerate}[noitemsep]
        \item All propositional formulae, barring those consisting of a single variable, shall be enclosed in parentheses.
        \item When we want to denote a propositional formula by a certain symbol, we will use the notation ``$\text{symobol} : \text{formula}$''.
    \end{enumerate}
\end{boxconvention}

As a concluding remark on the nature of propositional formulae, we will note that just as we use trees to evaluate expressions on the computer when performing arithmetic, we can use them to express and evaluate propositional formulae as well. We will not usually do this, however, as it takes up a lot of space. In any event, we would first need to make precise the notion of \textit{evaluating} a propositional formula. For this, we will turn to the concept of a truth function.

\subsection{Truth Functions}

Any assignment of truth values to the propositional variables in a formula $\phi$ determines the truth value for $\phi$ in a \textbf{unique} manner, using the exhaustive definitions of the connectives given in \Cref{Ch1:Def:Connective}. We often express all possible values of a propositional formula in a \textbf{truth table}, much like we did in \Cref{Ch1:Def:Connective} when defining the connectives.

\begin{boxexample}
    Consider the formula $\phi : \parenth{\parenth{p \to \parenth{\neg q}} \to p}$, where $p$ and $q$ are propositional variables. We construct a truth table as follows.
    \begin{table}[H]
        \centering
        \begin{tabular}{cc|cc|c}
        $p$ & q & $\parenth{\neg q}$ & $\parenth{p \to \parenth{\neg q}}$ & $\parenth{\parenth{p \to \parenth{\neg q}} \to p}$ \\ \hline
        $\T$ & $\T$ & $\F$ & $\F$ & $\T$ \\
        $\T$ & $\F$ & $\T$ & $\T$ & $\T$ \\
        $\F$ & $\T$ & $\F$ & $\T$ & $\F$ \\
        $\F$ & $\F$ & $\F$ & $\T$ & $\F$
        \end{tabular}
    \end{table}
\end{boxexample}

From this table, it is clear that the truth value of $\phi$ depends on the truth values of $p$ and $q$ in some manner (to be perfectly precise, it only depends on the truth value of $p$, and is, in fact, biconditionally equivalent to $p$). We would like to have a formal notion of navigating this dependence to `compute a value for $\phi$ given values of $p$ and $q$'.

Throughout this subsection, $n$ will denote an arbitrary natural number.

\begin{boxdefinition}[Truth Function]
    A \textbf{truth function} of $n$ variables is a function
    \begin{align*}
        f : \set{\T, \F}^n \to \set{\T, \F}
    \end{align*}
\end{boxdefinition}

Before discussing the relevance of truth functions, we will mention a very natural fact.

\begin{boxlemma}\label{Ch1:Lemma:TruthFunEq}
    To show two truth functions are equal, it suffices that they take the value $\T$ on precisely the same inputs or that they take the value $\F$ on precisely the same inputs.
\end{boxlemma}
\begin{proof}
    This is obvious, because any truth function can only take one of two values. If they take one value on precisely the same inputs, they must take the other value on the other inputs. This precisely corresponds to what it means for functions to be equal by extensionality.
\end{proof}

These are very directly related to propositional formulae.

\begin{boxdefinition}[Truth Function of a Propositional Formula]
    Let $\phi$ be a propositional formula whose variables are $p_1, \ldots, p_n$. We can associate to $\phi$ a truth function whose truth value at any $\parenth{x_1, \ldots, x_n} \in \set{\T, \F}^n$ corresponds to the truth value of $\phi$ that arises from setting $p_i$ to $x_i$ for all $1 \leq i \leq n$. We define this truth function to be the \textbf{truth function of $\phi$}, denoted $F_{\phi}$.
\end{boxdefinition}

We can now construct a truth function for the example we saw at the very beginning involving Mr and Mrs Jones (cf.~\eqref{Ch1:eq:JonesExample}).

\begin{boxexample}
    \sorry % Jones example
\end{boxexample}

We see something quite remarkable here: the truth function of the propositional formula defined in~\eqref{Ch1:eq:JonesExample} maps every possible input to $\T$! We have a special term for this.

\begin{boxdefinition}[Tautology]\label{Ch1:Def:Tautology}
    A propositional formula $\phi$ is a \textbf{tautology} if its truth function $F_{\phi}$ maps every possible input to $\T$.
\end{boxdefinition}

We can also be more precise about what the biconditional connective actually tells us.

\begin{boxdefinition}[Logical Equivalence]\label{Ch1:Def:LogicalEquivalence}
    The propositional formulae $\psi$ and $\chi$ are \textbf{logically equivalent} if the truth function $F_{\psi \lr \chi}$ of their biconditional is a tautology.
\end{boxdefinition}

We have a fairly basic result about logical equivalence.

\begin{boxlemma}
    Let $p_1, \ldots, p_n$ be propositional variables and let $\psi$ and $\chi$ be formulae in $p_1, \ldots, p_n$. Then, $\psi$ and $\chi$ are logically equivalent if and only if $F_{\psi} = F_{\chi}$.
\end{boxlemma}

We omit the proof of this result as it merely involves checking things manually. A computer should be able to do this almost instantaneously.

We can also say something about composing formulae together.

\begin{boxlemma}
    Suppose that $\phi$ is a propositional formula with variables $p_1, \ldots, p_n$. Let $\phi_1, \ldots, \phi_n$ be propositional formulae. Denote by $\vartheta$ the result of substituting each $p_i$ with $\phi_i$ in $\phi$. Then,
    \begin{enumerate}[label = \normalfont (\roman*), noitemsep]
        \item $\vartheta$ is a propositional formula.
        \item if $\phi$ is a tautology, so is $\vartheta$.
        \item the truth function of $\vartheta$ is the result of composing the truth function of $\phi$ with the Cartesian product of the truth functions of $\phi_1, \ldots, \phi_n$.
    \end{enumerate}
\end{boxlemma}

We do not prove this result either, as it merely involves manual verification.

\begin{boxexample}
    For propositional variables $p_1, p_2$, the statement $\parenth{\parenth{\parenth{\neg p_2} \to \parenth{\neg p_1}} \to \parenth{p_1 \to p_2}}$ is a tautology. Therefore, if $\phi_1$ and $\phi_2$ are propositional formulae, then $\parenth{\parenth{\parenth{\neg \phi_2} \to \parenth{\neg \phi_1}} \to \parenth{\phi_1 \to \phi_2}}$ is a tautology as well.
\end{boxexample}

We will also mention that a composition being a tautology does not mean the outermost proposition of the composition is a tautology.

\begin{boxnexample}
    Let $p$ be a propositional variable. The formula $\phi : \parenth{p \to \parenth{\neg p}}$ is not a tautology. However, we can find a propositional formula $\phi'$ such that $\parenth{\phi_1 \to \parenth{\neg \phi_1}}$ is a tautology: for example, we can define $\phi'$ to be identically $\F$.
\end{boxnexample}

There are numerous propositional formulae that we know to be logically equivalent. Here is a (non-exhaustive) list.

\begin{boxexample}[Logically Equivalent Formulae]
    Let $p_1, p_2, p_3$ be logically equivalent formulae. Then, the following equivalences hold.
    \begin{enumerate}[noitemsep]
        \item $\parenth{p_1 \land \parenth{p_2 \land p_3}}$ is logically equivalent to $\parenth{\parenth{p_1 \land p_2} \land p_3}$.
        
        \item $\parenth{p_1 \lor \parenth{p_2 \lor p_3}}$ is logically equivalent to $\parenth{\parenth{p_1 \lor p_2} \lor p_3}$.

        \item $\parenth{p_1 \lor \parenth{p_2 \land p_3}}$ is logically equivalent to $\parenth{\parenth{p_1 \lor p_2} \land \parenth{p_1 \land p_3}}$.

        \item $\parenth{\neg \parenth{\neg p_1}}$ is logically equivalent to $p_1$.

        \item $\parenth{\neg \parenth{p_1 \land p_2}}$ is logically equivalent to $\parenth{\parenth{\neg p_1} \lor \parenth{\neg p_2}}$.

        \item $\parenth{\neg \parenth{p_1 \lor p_2}}$ is logically equivalent to $\parenth{\parenth{\neg p_1} \land \parenth{\neg p_2}}$.
    \end{enumerate}
\end{boxexample}

Upon inspection, one can find algebraic patterns in the above logical equivalences. There are similarities to the axioms of a \textbf{boolean algebra}. % TODO: READ UP ON THIS
We will not explore this further in this module, but we will adopt the convention used in algebra where parentheses are dropped when dealing with associative operations.

\begin{boxconvention}
    We will denote both $\parenth{p_1 \land \parenth{p_2 \land p_3}}$ and $\parenth{\parenth{p_1 \land p_2} \land p_3}$ by $\parenth{p_1 \land p_2 \land p_3}$. Similarly, we will denote both $\parenth{p_1 \lor \parenth{p_2 \lor p_3}}$ and $\parenth{\parenth{p_1 \lor p_2} \lor p_3}$ by $\parenth{p_1 \lor p_2 \lor p_3}$.
\end{boxconvention}

We will end with a combinatorial fact about truth functions.

\begin{boxlemma}
    There are $2^{2^n}$ possible truth functions on $n$ variables.
\end{boxlemma}
\begin{proof}
    A truth function is any function from a the set $\set{\T, \F}^n$ to the set $\set{\T, \F}$, with no further restrictions. The former set has $2^n$ elements and the latter set has $2$ elements. Therefore, there are $2^{2^n}$ possible truth functions.
\end{proof}

\subsection{Adequacy}

We have defined several connectives so far, but we have yet to say anything about whether we will be defining any more connectives going forward. To begin, we will state an important definition.

\begin{boxdefinition}[Adequacy]\label{Ch1:Def:Adequacy}
    We say that a set $S$ of connectives is \textbf{adequate} if for every $n \geq 1$, every truth function on $n$ variables can be expressed as the truth function as a propositional formula which only involves connectives from $S$ (and $n$ propositional variables).
\end{boxdefinition}

The idea that this definition seeks to express is that a set is adequate if and only if for evern $n$, every propositional formula in $n$ variables is logically equivalent to a propositional formula that only contains those $n$ variables and connectives from the set in question. In other words, every propositional formula should admit an equivalent expression that does not contain any connectives apart from those in the set in question. The reason this is expressed in terms of truth functions is that that is how logical equivalence is \textit{defined} (cf. \Cref{Ch1:Def:LogicalEquivalence}).

We now have the first theorem of this module.

\begin{boxtheorem}\label{Ch1:Thm:DNFConnectivesAdequate}
    The set $\set{\neg, \land, \lor}$ is adequate.
\end{boxtheorem}
\begin{proof}
    Fix some $n \geq 1$, and let $G : \set{\T, \F}^n \to \set{\T, \F}$ be a truth function. We have two cases.

    \begin{description}
        \item[\textbf{\underline{Case 1.}}] The first case is a trivial case. There are two trivial truth functions on $n$ variables, namely, the constrant truth functions that take the values $\T$ and $\F$ for all inputs. Truth is not something encoded `naturally' into the connectives $\set{\neg, \land, \lor}$, but falsity is: the $\neg$ connective directly has to do with expressing falsity. Therefore, the trivial truth function that we will show can always be expressed in terms of the desired connectives is the one that is always false. We show this rigorously.
        
        Assume that $G$ is identically $\F$. Then, define the propositional formula $\phi : \parenth{p_1 \land \parenth{\neg p_1}}$. Even defining it as a formula on $n$ variables, it is clear to see that its truth function $F_{\phi}$ is identically $\F$. Therefore, $G = F_{\phi}$.\footnote{Admittedly, we are using the Axiom of Extensionality here to define what it means for the two functions to be equal. We will ignore this technicality for now.}
        
        \item[\textbf{\underline{Case 2.}}] The second case will be the nontrivial case of when a truth function can take on both values $\T$ and $\F$. The way we will show that $\set{\neg, \land, \lor}$ is adequate is by constructing a propositional formula in $n$ variables whose truth function is $\T$ whenever the one in question is $\T$. We will do this by isolating the inputs that yield $\T$ and manipulating propositional variables in a way that corresponds to these inputs.
        
        Assume that $G$ is not identically $\T$. Then, list all $v \in \set{\T, \F}^n$ such that $G(v) = \T$. Since $\set{\T, \F}^n$ is a finite set, this list is finite, and we can number these $v_1, \ldots, v_r$. For each $1 \leq i \leq r$, denote
        \begin{align*}
            v_i = \parenth{v_{i1}, \ldots, v_{ir}}
        \end{align*}
        where $v_{ij} \in \set{\T, \F}$ is the $j$th component of $v_i$. Let $p_1, \ldots, p_n$ be propositional variables. Define propositional formulae $\parenth{q_{ij}}_{1 \leq i \leq r, 1 \leq j \leq n}$ by
        \begin{align*}
            q_{ij} : \begin{cases}
                p_j & \text{if } v_{ij} = \T \\
                \parenth{\neg p_j} & \text{if } v_{ij} = \F
            \end{cases}
        \end{align*}
        Then, $q_{ij}$ has value $\T$ if and only if $p_j$ has value $v_{ij}$. The idea is to now construct a propositional formula that has value $\T$ if and only if $\parenth{p_1, \ldots, p_n}$ is one of the $v_i$.
        
        First, we formalise the notion of the $\parenth{p_1, \ldots, p_n}$ taking the value of one of the $v_i$. The idea is to combine them using the $\land$ connective. Define propositional formulae $\parenth{\psi_i}_{1 \leq i \leq r}$ by
        \begin{align*}
            \psi_i : \parenth{q_{i1} \land \cdots \land q_{in}}
        \end{align*}
        Then, we have that for all $1 \leq i \leq r$ and $v \in \set{\T, \F}^n$,
        \begin{align*}
            F_{\psi_i}(v) = \T
            \iff
            q_{i1}, \ldots, q_{in} \text{ all have value } \T
            \iff
            \text{Each } p_j \text{ has value } v_{ij}
            \iff
            v = v_i
        \end{align*}
        Next, we combine these $\psi_i$ so that the truth function of the resulting formula is $\T$ if and only if one of the $\psi_i$ is true, a fact that would be equivalent to the input of the truth function  being precisely one of the $v_i$. We do this using the $\lor$ connective. Define the propositional formula
        \begin{align*}
            \vartheta : \parenth{\psi_1 \lor \cdots \lor \psi_r}
        \end{align*}
    \end{description}
    Then, for all $v \in \set{\T, \F}^n$, we have that
    \begin{align*}
        F_{\vartheta}(v) = \T
        \iff
       \text{One of the } \psi_i \text{ is true}
        \iff
        v \text{ is precisely equal to one of the } v_i
    \end{align*}
    In partiular, we have that $F_{\vartheta}(v) = \T$ if and only if $G(v) = \T$ for all $v \in \set{\T, \F}^n$. Then, by \Cref{Ch1:Lemma:TruthFunEq}, we are done.
\end{proof}

Before illustrating the point of the above theorem, we make an important definition.

\begin{boxdefinition}[Disjunctive Normal Form]
    When a propositional formula is expressed only in terms of propositional variables and the set $\set{\neg, \land, \lor}$ of connectives, it is said to be in \textbf{disjunctive normal form}, which we abbreviate to \textbf{DNF}.
\end{boxdefinition}

What \Cref{Ch1:Thm:DNFConnectivesAdequate} then tells us is that every propositional formula is expressible in DNF.

\begin{boxcorollary}
    For every propositional formula in $n$ variables, there exists a logically equivalent propositional formula in $n$ variables that is in DNF.
\end{boxcorollary}
\begin{proof}
    We know that every propositional formula admits a truth function. For any propositional formula in $n$ variables, we can apply \Cref{Ch1:Thm:DNFConnectivesAdequate} to its truth function. Then, unfolding the definition of adequacy yields the desired result.
\end{proof}

\begin{boxexample}
    Let $p_1$ and $p_2$ be propositional variables. Consider the propositional formula $\chi : \parenth{\parenth{p_1 \to p_2} \to \parenth{\neg p_2}}$. We can see that $F_{\chi}(v) = \T$ only if $v = \parenth{\T, \F}$ or $v = \parenth{\F, \F}$. Therefore, the DNF of $\chi$ is
    \begin{align*}
        \parenth{\parenth{p_1 \land \parenth{\neg p_2}} \lor \parenth{\parenth{\neg p_1} \land \parenth{\neg p_2}}}
    \end{align*}
\end{boxexample}

It turns out that $\set{\neg, \land, \lor}$ is not the only adequate set of connectives.

\begin{boxexample}[Adequate Sets]
    The following sets of connectives are adequate.
    \begin{enumerate}[noitemsep, label = \normalfont(\roman*)]
        \item\label{Ch1:Lemma:AdequateSets:Case:1} $\set{\neg, \lor}$
        \item $\set{\neg, \land}$
        \item $\set{\neg, \to}$
    \end{enumerate}
    The way we can prove this is by simplifying each case using \Cref{Ch1:Thm:DNFConnectivesAdequate}. Fix propositional variables $p_1, p_2$.
    \begin{enumerate}[label = \normalfont(\roman*)]
        \item It suffices to show that $p_1 \land p_2$ can be expressed using $\neg$ and $\lor$. Indeed,
        \begin{align*}
            \parenth{p_1 \land p_2} \isleto \parenth{\neg\parenth{\parenth{\neg p_1} \lor \parenth{\neg p_2}}}
        \end{align*}
        
        \item It suffices to show that $p_1 \lor p_2$ can be expressed using $\neg$ and $\land$. Indeed,
        \begin{align*}
            \parenth{p_1 \lor p_2} \isleto \parenth{\neg\parenth{\parenth{\neg p_1} \land \parenth{\neg p_2}}}
        \end{align*}

        \item By Case~\ref{Ch1:Lemma:AdequateSets:Case:1}, it suffices to show that $p_1 \lor p_2$ can be expressed in terms of $\neg$ and $\to$. Indeed,
        \begin{align*}
            \parenth{p_1 \lor p_2} \isleto \parenth{\parenth{\neg p_1} \to p_2}
        \end{align*}
    \end{enumerate}
\end{boxexample}

There are also sets of connectives that are not adequate.

\begin{boxnexample}[Inadequate Sets]
    The following sets are not adequate.
    \begin{enumerate}[label = (\roman*)]
        \item $\set{\land, \lor}$
        \item $\set{\neg, \lr}$
    \end{enumerate}
    The way we can prove this is by constructing truth functions that cannot be realised by combining propositional variables using only the connectives in the above sets.
    \begin{enumerate}[label = (\roman*)]
        \item No truth function that is identically false can be realised. For that matter, no truth function that maps an input whose every component is $\T$ to $\F$ can be realised. Formally, consider any propositional formula $\phi$ built exclusively using a finite set of propositional variables and the connectives $\land$ and $\lor$. One can show, by induction on the number of connectives in $\phi$, that $F_{\phi}(\T, \ldots, \T) = \T$. Since this is true of any $\phi$, a truth function mapping an input of the form $(\T, \ldots, \T)$ to $\F$ is not the truth function of a propositional formula that only includes $\land$ and $\lor$.
        
        \item No truth function that is identcally true can be realised.
    \end{enumerate}
\end{boxnexample}

It turns out that there is one connective with a rather astounding adequacy property.

\begin{boxdefinition}[The NOR Connective]
    Define the \textbf{NOR connective}, denoted $\nor$, via the following truth table in propositional variables $p$ and $q$.
    \begin{table}[H]
        \centering
        \begin{tabular}{cc|c}
            $p$ & $q$ & $\parenth{p \nor q}$ \\ 
            \hline
            $\T$ & $\T$ & $\F$ \\
            $\T$ & $\F$ & $\F$ \\
            $\F$ & $\T$ & $\F$ \\
            $\F$ & $\F$ & $\T$
        \end{tabular}
    \end{table}
\end{boxdefinition}

Informally, NOR corresponds to ``neither ... nor ...''. Formally, we have the following.

\begin{boxlemma}
    For all propositional variables $p$ and $q$, the DNF of $\parenth{p \nor q}$ is given by $\parenth{\parenth{\neg p} \land \parenth{\neg q}}$. In particular, we have that $\parenth{p \nor q} \isleto \parenth{\parenth{\neg p} \land \parenth{\neg q}}$.
\end{boxlemma}
We do not write out a proof, as it merely involves comparing truth tables.

\begin{boxexample}[An Adequate Set with One Connective]
    It turns out that $\set{\nor}$ is connective. Indeed, for propositional variables $p$ and $q$, we have
    \begin{enumerate}
        \item $\parenth{p \nor p} \isleto \parenth{\neg p}$.
        \item $\parenth{\parenth{p \nor p} \nor \parenth{q \nor q}} \isleto \parenth{p \land q}$.
    \end{enumerate}
\end{boxexample}

So far, we have been studying \textit{meaning}, in the form of truth functions, but have yet to formally define \textit{what} we are allowed to express that \textit{has} meaning within the propositional paradigm. In other words, we have been studying \textbf{semantics} but have yet to define the \textbf{syntax} of propositional logic. We will do this in the next section.
