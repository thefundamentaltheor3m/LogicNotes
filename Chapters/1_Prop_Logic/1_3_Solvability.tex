\section{Solvability and Nilpotency}

We now begin discussing some nontrivial objects in the theory of Lie algebras. Throughout this section, $L$ will denote an arbitrary Lie algebra.

\subsection{Descending Series of Ideals}

\begin{boxdefinition}[Derived Series]
    The \textbf{derived series} of $L$ is the descending series of ideals
    \begin{align*}
        L = L^{(0)} \supseteq L^{(1)} \supseteq L^{(2)} \supseteq \cdots % \supseteq L^{(n - 1)} \supseteq L^{(n)}
    \end{align*}
    where $L^{(i)} := \brac{L^{(i - 1)}, L^{(i-1)}}$ for $i \geq 1$. % \leq n$.
\end{boxdefinition}

\begin{boxdefinition}[Solvability]
    $L$ is said to be \textbf{solvable} if there exists an $n \in \N$ such that $L^{(n)} = 0$.
\end{boxdefinition}

\begin{boxdefinition}[Lower Central Series]
    The \textbf{lower central series} of $L$ is the descending series of ideals
    \begin{align*}
        L = L^{0} \supseteq L^{1} \supseteq L^{2} \supseteq \cdots % \supseteq L^{n - 1} \supseteq L^{n}
    \end{align*}
    where $L^{i} := \brac{L, L^{i-1}}$ for $i \geq 1$. % \leq n$.
\end{boxdefinition}

\begin{boxconvention}
    Elements of the derived series are denoted $L^{(i)}$, with parenthesised superscript indices, whereas elements of the lower central series are denoted $L^{i}$, with no parentheses around the indices.
\end{boxconvention}

\begin{boxdefinition}[Nilpotency]
    $L$ is said to be \textbf{nilpotent} if there exists an $n \in \N$ such that $L^{n} = 0$.
\end{boxdefinition}

Indeed, there is the following relationship between solvability and nilpotency.

\begin{lemma}\label{Ch1:Lemma:DerivedSeriesContainedInLowerCentralSeries}
    For all $i \in \N$, $L^i \supseteq L^{(i)}$.
\end{lemma}
\begin{proof}
    We argue by induction on $i$. The base case is trivial, because $L^0 = L = L^{(0)}$. Now, fix $i \in \N$ and assume that $L^i \supseteq L^{(i)}$. Then,
    \begin{align*}
        L^{i + 1}
        = \brac{L, L^i} &= \brac{L, L^{(i)}} \\
        &= \Span{\setst{\brac{\ell, x}}{x \in L^{(i)}, \ell \in L}} \\
        &\supseteq \Span{\setst{\brac{\ell, x}}{x \in L^{(i)}, \ell \in L^{(i)}}} \\
        &= \brac{L^{(i)}, L^{(i)}} = L^{(i + 1)}
    \end{align*}
    where the inclusion on the third line follows from the fact that $L^{(i)} \subseteq L$. This completes the induction and proves the desired result for all $i \in \N$.
\end{proof}

\begin{boxcorollary}
    If $L$ is nilpotent, then $L$ is solvable.
\end{boxcorollary}
\begin{proof}
    \Cref{Ch1:Lemma:DerivedSeriesContainedInLowerCentralSeries} tells us that for all $n \in \N$, $L^n = 0$ implies $L^{(n)} = 0$. Thus, if such an $n$ exists that makes $L$ nilpotent, the same $n$ would also make $L$ solvable.
\end{proof}

The converse is not true. We need to develop a bit of theory to see this.

\begin{lemma}\label{Ch1:Lemma:NilpotentCentreNonzero}
    If $L$ is nilpotent, its centre is nonzero.
\end{lemma}
\begin{proof}
    Let $n \in \N$ be such that
    \begin{align*}
        L = L^0 \supseteq L^1 \supseteq \cdots \supseteq L^n \subsetneq L^{n+1} = 0
    \end{align*}
    with $L^n \neq 0$.
\end{proof}

\subsection{Ideals, Quotients and Subalgebras}

Throughout this subsection, let $I \nsg L$ and $K \leq L$. Recall that $I$ is a Lie subalgebra of $L$ (cf. \Cref{Ch1:Lemma:IdealSubalg}), meaning we can impose solvability and nilpotency conditions on $I$ as well.

\begin{definition}[Solvability of Subalgebras]
    We say a subalgebra of $L$ is \textbf{solvable} if it is solvable as a Lie algebra in its own right.
\end{definition}

\begin{boxproposition}[Solvability Conditions]\label{Ch1:Prop:SolvabilityConditions}
    \hfill
    \begin{enumerate}[label = \normalfont\arabic*., noitemsep]
        \item If $L$ is solvable, then so is $\quotient{L}{I}$.
        \item If $L$ is solvable, then so is $K$.
        \item If $I$ and $\quotient{L}{I}$ are solvable, then so is $L$.
    \end{enumerate}
\end{boxproposition}
\begin{proof}
    Let $\phi : L \surj \quotient{L}{I}$ be the quotient homomorphism.
    \begin{enumerate}
        \item Observe that it suffices to show that $\phiof{L^{(i)}} = \phiof{L}^{(i)}$ for all $i \in \N$: if this were true, then the existence of some $n \in \N$ such that $L^{(n)} = 0$ would imply that
        \begin{align*}
            \parenth{\quotient{L}{I}}^{(n)} = \phiof{L}^{(n)} = \phiof{L^{(n)}} = \phiof{0} = 0
        \end{align*}
        making $\quotient{L}{I}$ solvable whenever $L$ is.
        
        We will now prove that $\phiof{L^{(i)}} = \phiof{L}^{(i)}$ by induction on $i$. When $i = 0$, the result is trivial: it is true that $\phiof{L} = \phiof{L}$ by reflexivity. % Lean-like?
        Now, fix $i \in \N$ and assume that $\phiof{L^{(i)}} = \phiof{L}^{(i)}$. Then,
        \begin{align*}
            \phiof{L^{(i + 1)}} = \phiof{\brac{L^{(i)}, L^{(i)}}}
            &= \phiof{\Span{\setst{\brac{x,y}}{x, y \in L^{(i)}}}} \\
            &= \Span{\phiof{\setst{\brac{x,y}}{x, y \in L^{(i)}}}} \\
            &= \Span{\setst{\brac{\phiof{x}, \phiof{y}}}{x, y \in L^{(i)}}} \\
            &= \brac{\phiof{L^{(i)}}, \phiof{L^{(i)}}} \\
            &= \brac{\phiof{L}^{(i)}, \phiof{L}^{(i)}} = \phiof{L}^{(i + 1)}
            % &= \brac{\parenth{\quotient{L}{I}}^{(i)}, \parenth{\quotient{L}{I}}^{(i)}}
            % = \parenth{\quotient{L}{I}}^{(i + 1)}
        \end{align*}
        as required.

        \item It suffices to prove that for all $i \in \N$, $K^{(i)} \subseteq L^{(i)}$: if this were true, then the existence of some $n \in \N$ such that $L^{(n)} = 0$ would imply that $K^{(n)} = 0$, making $K$ solvable whenever $L$ is.
        
        We will now prove that $K^{(i)} \subseteq L^{(i)}$ by induction on $i$. The base case is trivial, because $K^{(0)} = K \subseteq L = L^{(0)}$. Now, fix $i \in \N$ and assume that $K^{(i)} \subseteq L^{(i)}$. Then,
        \begin{align*}
            K^{(i + 1)} = \brac{K^{(i)}, K^{(i)}}
            &= \Span{\setst{\brac{x,y}}{x, y \in K^{(i)}}} \\
            &\subseteq \Span{\setst{\brac{x,y}}{x, y \in L^{(i)}}} \\
            &= \brac{L^{(i)}, L^{(i)}} = L^{(i + 1)}
        \end{align*}
        as required.

        \item Let $m \in \N$ be such that $I^{(m)} = 0$ and let $n \in \N$ be such that $\parenth{\quotient{L}{I}}^{(n)} = 0$. It suffices to prove that for all $i, j \in \N$, $\parenth{L^{(i)}}^{(j)} = L^{(i + j)}$: if this were true, then the fact that
        \begin{align*}
            \phiof{L^{(n)}} = \parenth{\quotient{L}{I}}^{(n)} = 0
        \end{align*}
        would immediately imply that $L^{(n)} \subseteq \pker{\phi} = I$, from which it would follow that $\parenth{L^{(n)}}^{(m)} = 0$, and therefore, that $L^{(n + m)} = 0$, making $L$ solvable whenever $I$ and $\quotient{L}{I}$ are.

        We will now prove that $\parenth{L^{(i)}}^{(j)} = L^{(i + j)}$ by letting $i$ be arbitrary and performing induction on $j$. The base case is trivial, because $\parenth{L^{(i)}}^{(0)} = L^{(i)}$. Now, fix $j \in \N$ and assume that $\parenth{L^{(i)}}^{(j)} = L^{(i + j)}$. Then,
        \begin{align*}
            \parenth{L^{(i)}}^{(j + 1)} = \brac{\parenth{L^{(i)}}^{(j)}, \parenth{L^{(i)}}^{(j)}} = \brac{L^{(i + j)}, L^{(i + j)}} = L^{(i + j + 1)}
        \end{align*}
        as required.
    \end{enumerate}
\end{proof}

We have similar results for nilpotency.

\begin{definition}[Nilpotency of Subalgebras]
    We say a subalgebra of $L$ is \textbf{nilpotent} if it is solvable as a Lie algebra in its own right.
\end{definition}

\begin{boxproposition}[Nilpotency Conditions]
    \hfill
    \begin{enumerate}[label = \normalfont\arabic*., noitemsep]
        \item If $L$ is nilpotent, then so is $\quotient{L}{I}$.
        \item If $L$ is nilpotent, then so is $K$.
    \end{enumerate}
\end{boxproposition}

We will not prove these results here, as they are very similar to the corresponding results for solvability. We will, however, mention that the reason why we do not have a nilpotency condition for $L$ when $I$ and $\quotient{L}{I}$ are nilpotent is that it is not, in general, true that $\parenth{L^i}^j = L^{i + j}$ for $i, j \in \N$, as we can easily see from the following counterexample.

\begin{boxcexample}
    \sorry
\end{boxcexample}

We will end the discussion on solvability and nilpotency by saying a bit about the derived subalgebra. We first make a general observation.

\begin{lemma}\label{Ch1:Lemma:AllDerivsOfDerivEqTop}
    If $L = L'$, then for all $i \in \N$, $L^{(i)} = L^{(1)} = L' = L$.
\end{lemma}
\begin{proof}
    We argue by induction on $i$. The base case is trivial, because $L^{(0)} = L$. Now, fix $i \in \N$ and assume that $L^{(i)} = L$. Then,
    \begin{align*}
        L^{(i + 1)} = \brac{L^{(i)}, L^{(i)}} = \brac{L, L} = L'
    \end{align*}
    Furthermore, it is clear that $L^{(1)} = L'$, and, by assumption, $L' = L$. This completes the induction and proves the desired result for all $i \in \N$.
\end{proof}

There is an immediate consequence.

\begin{corollary}\label{Ch1:Cor:DerivedSubalgLtOfSolvable}
    If $L \neq 0$ and $L$ is solvable, then $L' < L$.
\end{corollary}
\begin{proof}
    We argue by contraposition. If $L' = L$, then we know that $L^{(i)} = L$ for all $i \in \N$. In particular, since $L$ is nonzero, none of the $L^{(i)}$ can be zero. $L$ is therefore not solvable.
\end{proof}

We will be more interested the following, somewhat less immediate consequence that comes from combining applying the Correspondence Theorem to the ideals of quotient spaces of solvable Lie algebras.

\begin{boxproposition}\label{Ch1:Prop:ExistsIdealCodim1}
If $L$ is solvable, there exists an ideal $I \nsg L$ of codimension $1$.
\end{boxproposition}
\begin{proof}
    Consider the quotient Lie algebra $K := \quotient{L}{L'}$. We know that $0 < K$, because $K = 0$ would imply that $L = L'$, which is impossible because $L$ is solvable, as shown in \Cref{Ch1:Cor:DerivedSubalgLtOfSolvable}. Therefore, $K$ contains a subspace $W$ of codimension $1$. Since $K$ is abelian, \Cref{Ch1:Prop:SubspaceIdealOfAbelian} tells us that $W$ is an ideal of $K$. \Cref{SP:Thm:Correspondence} tells us that the preimage $V$ of $W$ under the quotient epimorphism is an ideal of $L$ that contains $L'$. Simple arithmetic and dimension results from linear algebra then tell us
    \begin{align*}
        \pdim{V} = \pdim{W} + \pdim{L'} = \parenth{\pdim{L} - \pdim{L'} - 1} + \pdim{L'} = \pdim{L} - 1
    \end{align*}
\end{proof}

\subsection{The Radical Ideal}

Throughout this subsection, we will assume that $L$ is finite-dimensional.

We begin with a basic result about the sums of ideals.

\begin{lemma}\label{Ch1:Lemma:SumIdealSolvable}
    Let $I, J \nsg L$. If $I$ and $J$ are solvable, then so is $I + J \nsg L$.
\end{lemma}
\begin{proof}
    \sorry
    % Apply 2nd iso thm
\end{proof}

\begin{boxcorollary}\label{Ch1:Cor:RadExists}
    There exists a solvable ideal of $L$ that contains all other solvable ideals of $L$.
\end{boxcorollary}
\begin{proof}
    Let $R$ be a solvable ideal of $L$ of maximal dimension.\footnote{When we say maximal dimension, we mean that the dimension of $R$ is the largest possible dimension such that a solvable ideal of that dimension exists. This is well-defined because $L$ is finite-dimensional, and the dimension of any ideal of $L$ is necessarily $\leq \pdim{L}$.} Now, fix any $I \nsg L$. \Cref{Ch1:Lemma:SumIdealSolvable} tells us that $I + R$ is a solvable ideal of $L$. But, since $R$ is of maximal dimension, we know that $\pdim{I + R} \leq \pdim{R}$. Therefore, we must have that $I + R = R$. The fourth point in \Cref{Ch1:Prop:IdealBhv} then tells us that $I \subseteq R$, as required.
\end{proof}

This solvable ideal has a name.

\begin{boxdefinition}[Radical Ideal]
    The \textbf{radical ideal} of $L$ is the solvable ideal of $L$ that contains all other solvable ideals of $L$, which we know exists from \Cref{Ch1:Cor:RadExists}.
\end{boxdefinition}

We can now define what it means for a Lie algebra to be semi-simple. We will be very interested in this class of Lie algebras going forward.

\begin{boxdefinition}[Semi-Simplicity]\label{Ch1:Def:SemiSimple}
    We say that $L$ is \textbf{semi-simple} if its radical ideal is the $0$ ideal.
\end{boxdefinition}

Our aim for this module will be to classify all semi-simple Lie algebras. We will do this by first classifying all solvable Lie algebras and then using that classification to classify all semi-simple Lie algebras. We will need a \textit{lot} more machinery before we can do this, but we will get there eventually.

We also have a notion of simplicity, which is no different from what we would expect in groups.

\begin{boxdefinition}[Simplicity]
    We say that $L$ is \textbf{simple} if it has no nontrivial ideals.
\end{boxdefinition}

\subsection{Ascending Series of Ideals}

We will end by talking about ascending series of ideals and their corresponding quotients. Throughout this subsection, $L$ will denote an arbitrary Lie algebra.

\begin{boxdefinition}[Ascending Central Series]
    We say that an increasing chain of ideals
    \begin{align*}
        0 \subseteq L_1 \subseteq L_2 \subseteq L_3 \subseteq \cdots
    \end{align*}
    is an \textbf{ascending central series} of $L$ if $L_1 = \Zof{L}$ and for all $i \in \N$, we have
    \begin{enumerate}
        \item $L_i \nsg L$ with quotient map $g_i : L \surj \quotient{L}{L_i}$
        \item $L_{i + 1} = g_i\inv\!\parenth{\Zof{\quotient{L}{L_i}}}$
    \end{enumerate}
\end{boxdefinition}

\begin{boxconvention}
    We will use subscripted $L_i$s to denote elements of the ascending central series, in contrast to superscripts used for the descending central series.
\end{boxconvention}

We now have an equivalent criterion for nilpotency.

\begin{boxproposition}
    $L$ is nilpotent if and only if $L_n = L$ for some $n \in \N$.
\end{boxproposition}
\begin{proof} \hfill
    \begin{description}
        \item[$\parenth{\implies}$]
            One can show by induction on $n$ that if $L^n = 0$, then $L_n = L$. Then, if $\exists n \in \N$ such that $L^n = 0$, ie, if $L$ is nilpotent, then $L_n = 0$ as well.  \sorry % (Same n works for both)

        \item[$\parenth{\impliedby}$] 
            % Use fact that image of centre in quotient map is centre of quotient
            % The idea is to show that $L^{k + 1} = \brac{L, L^k} \subseteq L_{n + k - 1}$ and $\brac{L, L^k} \subseteq \brac{L, L_{n - l}}$. And $L_{n - k}$ is the preimage of $\Zof{\quotient{L}{L_{n - k - 1}}}$
            \sorry
    \end{description}
\end{proof}

We end with another counterexample that shows that solvable Lie algebras need not be nilpotent.

\begin{boxcexample}
    For all $n$, $\t{n}$ is solvable but not nilpotent.
    \begin{proof}[Proof that $\t{n}$ is solvable]
        First, observe that $\brac{\t{n}, \t{n}} = \t{n}' = \u{n}$. By \sorry, we know that $\u{n}$ is nilpotent. Therefore, $\u{n}$ is solvable. Furthermore, $\quotient{\t{n}}{\t{n}'}$ is abelian, making it solvable by \sorry. Therefore, by \Cref{Ch1:Prop:SolvabilityConditions}, $\t{n}$ is solvable.
    \end{proof}

    \begin{proof}[Proof that $\t{n}$ is not nilpotent]
        
    \end{proof}
\end{boxcexample}
