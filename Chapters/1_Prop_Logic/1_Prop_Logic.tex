%% WEEK 1

\chapter{Propositional Logic}\label{Ch1:CH}
\thispagestyle{empty}

Propositional logic is the logic of reasoning and proof. Before we get started with anything formal, here's a motivating example.

Consider the following statement:
\begin{quote}
    If Mr Jones is happy, then Mrs Jones is unhappy, and if Mrs Jones is unhappy, then Mr Jones is unhappy. Therefore, Mr Jones is unhappy.
\end{quote}
One can ask ourselves whether it is logically valid to conclude that Mr Jones is unhappy based on the relationship between the happiness of Mr Jones and that of Mrs Jones expressed in the sentence preceding it.

Putting this into symbols, let $P$ denote the statement that Mr Jones is happy, and let $Q$ denote the statement that Mrs Jones is unhappy. We can express the statement as follows:
\begin{align}
    \parenth{\parenth{P \implies Q} \land \parenth{Q \implies  \neg P}} \implies \parenth{\neg P}
    \label{Ch1:eq:JonesExample}
\end{align}
This disambiguation, by removing any question of marital harmony from what is otherwise a purely logical question, allows us to manually check whether~\eqref{Ch1:eq:JonesExample} is a valid statement by constructing a \textbf{truth table}.

We will begin by developing some machinery to reason about these sorts of statements more formally.

\section{Propositional Formulae}

\subsection{Propositions and Connectives}

We begin by defining the notion of a proposition.

\begin{boxdefinition}[Proposition]
    A \textbf{proposition} is a statement that is either true or false.
\end{boxdefinition}

\begin{boxconvention}
    We will denote the state of being \textbf{true} by $\T$ and that of being \textbf{false} by $\F$.
\end{boxconvention}

Propositions can be connected to each other using tools known as \textbf{connectives}. These can be thought of as \textbf{truth table rules}.

\begin{boxconvention}
    Before we define the actual connectives we shall use, we list them down, along with notation.
    \begin{enumerate}[noitemsep]
        \item Conjunction ($\land$)
        \item Disjunction ($\lor$)
        \item Negation ($\neg$)
        \item Implication ($\to$)
        \item The Biconditional ($\lr$)
    \end{enumerate}
    In particular, we will only use the $\implies$ and $\iff$ symbols when reasoning \textbf{informally}. For \textbf{formal} use, we will stick to the $\to$ and $\lr$ symbols. In more precise terms, we will use $\implies$ and $\iff$ when reasoning \textbf{about} the language we are constructing, whereas we will use $\to$ and $\lr$ when reasoning \textbf{within} the language. As we shall see, it will be of paramount importance to distinguish between these two modes of reasoning.
\end{boxconvention}

We define them exhaustively as follows.
\begin{boxdefinition}[Connectives]\label{Ch1:Def:Connective}
    Let $p$ and $q$ be true/false variables. We define each of the connectives listed above to take on truth values depending on those of $p$ and $q$ as follows.
    \begin{table}[H]
        \centering
        \begin{tabular}{cc|cccccc}
             $p$ & $q$ & $\parenth{\neg p}$ & $\parenth{\neg q}$ & $\parenth{p \land q}$ & $\parenth{p \lor q}$ & $\parenth{p \to q}$ & $\parenth{p \lr q}$ \\
             \hline
             $\T$ & $\T$ & $\F$ & $\F$ & $\T$ & $\T$ & $\T$ & $\T$ \\
             $\T$ & $\F$ & $\F$ & $\T$ & $\F$ & $\T$ & $\F$ & $\F$ \\
             $\F$ & $\T$ & $\T$ & $\F$ & $\F$ & $\T$ & $\T$ & $\F$ \\
             $\F$ & $\F$ & $\T$ & $\T$ & $\F$ & $\F$ & $\T$ & $\T$
        \end{tabular}
        \label{tab:my_label}
    \end{table}
\end{boxdefinition}

We are now ready to define the main object of study in this section: propositional formulae.

\begin{boxdefinition}[Propositional Formula]\label{Ch1:Def:PropFormula}
    A \textbf{propositional formula} is obtained from propositional variables and connectives via the following rules:
    \begin{enumerate}[label=\normalfont (\roman*), noitemsep]
        \item Any propositional variable is a propositional formula.
        \item If $\phi$ and $\psi$ are formulae, then so are $\parenth{\neg \phi}$, $\parenth{\neg \psi}$, $\parenth{\phi \land \psi}$, $\parenth{\phi \lor \psi}$, $\parenth{\phi \to \psi}$, $\parenth{\psi \to \phi}$, and $\parenth{\phi \lr \psi}$.
        \item Any formula arises in this manner after a finite number of steps.
    \end{enumerate}
\end{boxdefinition}

What this means is that a propositional formula is a string of symbols consisting of
\begin{enumerate}[noitemsep]
    \item variables that take on true/false values,
    \item connectors that express the relationship between these variables, and
    \item parentheses/brackets that separate formulae within formulae and specify the order in which they must be evaluated when the constituent variables are assigned specific values.
\end{enumerate}
In particular, every propositional formula is either a propositional variable or is built from `shorter' formulae, where by `shorter' we mean consisting of fewer symbols.

\begin{boxconvention}
    Throughout this module, we will adopt two important conventions when dealing with propositional formulae.
    \begin{enumerate}[noitemsep]
        \item All propositional formulae, barring those consisting of a single variable, shall be enclosed in parentheses.
        \item When we want to denote a propositional formula by a certain symbol, we will use the notation ``$\text{symobol} : \text{formula}$''.
    \end{enumerate}
\end{boxconvention}

As a concluding remark on the nature of propositional formulae, we will note that just as we use trees to evaluate expressions on the computer when performing arithmetic, we can use them to express and evaluate propositional formulae as well. We will not usually do this, however, as it takes up a lot of space. In any event, we would first need to make precise the notion of \textit{evaluating} a propositional formula. For this, we will turn to the concept of a truth function.

\subsection{Truth Functions}

Any assignment of truth values to the propositional variables in a formula $\phi$ determines the truth value for $\phi$ in a \textbf{unique} manner, using the exhaustive definitions of the connectives given in \Cref{Ch1:Def:Connective}. We often express all possible values of a propositional formula in a \textbf{truth table}, much like we did in \Cref{Ch1:Def:Connective} when defining the connectives.

\begin{boxexample}
    Consider the formula $\phi : \parenth{\parenth{p \to \parenth{\neg q}} \to p}$, where $p$ and $q$ are propositional variables. We construct a truth table as follows.
    \begin{table}[H]
        \centering
        \begin{tabular}{cc|cc|c}
        $p$ & q & $\parenth{\neg q}$ & $\parenth{p \to \parenth{\neg q}}$ & $\parenth{\parenth{p \to \parenth{\neg q}} \to p}$ \\ \hline
        $\T$ & $\T$ & $\F$ & $\F$ & $\T$ \\
        $\T$ & $\F$ & $\T$ & $\T$ & $\T$ \\
        $\F$ & $\T$ & $\F$ & $\T$ & $\F$ \\
        $\F$ & $\F$ & $\F$ & $\T$ & $\F$
        \end{tabular}
    \end{table}
\end{boxexample}

From this table, it is clear that the truth value of $\phi$ depends on the truth values of $p$ and $q$ in some manner (to be perfectly precise, it only depends on the truth value of $p$, and is, in fact, biconditionally equivalent to $p$). We would like to have a formal notion of navigating this dependence to `compute a value for $\phi$ given values of $p$ and $q$'.

Throughout this subsection, $n$ will denote an arbitrary natural number.

\begin{boxdefinition}[Truth Function]
    A \textbf{truth function} of $n$ variables is a function
    \begin{align*}
        f : \set{\T, \F}^n \to \set{\T, \F}
    \end{align*}
\end{boxdefinition}

Before discussing the relevance of truth functions, we will mention a very natural fact.

\begin{boxlemma}\label{Ch1:Lemma:TruthFunEq}
    To show two truth functions are equal, it suffices that they take the value $\T$ on precisely the same inputs or that they take the value $\F$ on precisely the same inputs.
\end{boxlemma}
\begin{proof}
    This is obvious, because any truth function can only take one of two values. If they take one value on precisely the same inputs, they must take the other value on the other inputs. This precisely corresponds to what it means for functions to be equal by extensionality.
\end{proof}

These are very directly related to propositional formulae.

\begin{boxdefinition}[Truth Function of a Propositional Formula]
    Let $\phi$ be a propositional formula whose variables are $p_1, \ldots, p_n$. We can associate to $\phi$ a truth function whose truth value at any $\parenth{x_1, \ldots, x_n} \in \set{\T, \F}^n$ corresponds to the truth value of $\phi$ that arises from setting $p_i$ to $x_i$ for all $1 \leq i \leq n$. We define this truth function to be the \textbf{truth function of $\phi$}, denoted $F_{\phi}$.
\end{boxdefinition}

We can now construct a truth function for the example we saw at the very beginning involving Mr and Mrs Jones (cf.~\eqref{Ch1:eq:JonesExample}).

\begin{boxexample}
    \sorry % Jones example
\end{boxexample}

We see something quite remarkable here: the truth function of the propositional formula defined in~\eqref{Ch1:eq:JonesExample} maps every possible input to $\T$! We have a special term for this.

\begin{boxdefinition}[Tautology]\label{Ch1:Def:Tautology}
    A propositional formula $\phi$ is a \textbf{tautology} if its truth function $F_{\phi}$ maps every possible input to $\T$.
\end{boxdefinition}

We can also be more precise about what the biconditional connective actually tells us.

\begin{boxdefinition}[Logical Equivalence]\label{Ch1:Def:LogicalEquivalence}
    The propositional formulae $\psi$ and $\chi$ are \textbf{logically equivalent} if the truth function $F_{\psi \lr \chi}$ of their biconditional is a tautology.
\end{boxdefinition}

We have a fairly basic result about logical equivalence.

\begin{boxlemma}
    Let $p_1, \ldots, p_n$ be propositional variables and let $\psi$ and $\chi$ be formulae in $p_1, \ldots, p_n$. Then, $\psi$ and $\chi$ are logically equivalent if and only if $F_{\psi} = F_{\chi}$.
\end{boxlemma}

We omit the proof of this result as it merely involves checking things manually. A computer should be able to do this almost instantaneously.

We can also say something about composing formulae together.

\begin{boxlemma}
    Suppose that $\phi$ is a propositional formula with variables $p_1, \ldots, p_n$. Let $\phi_1, \ldots, \phi_n$ be propositional formulae. Denote by $\vartheta$ the result of substituting each $p_i$ with $\phi_i$ in $\phi$. Then,
    \begin{enumerate}[label = \normalfont (\roman*), noitemsep]
        \item $\vartheta$ is a propositional formula.
        \item if $\phi$ is a tautology, so is $\vartheta$.
        \item the truth function of $\vartheta$ is the result of composing the truth function of $\phi$ with the Cartesian product of the truth functions of $\phi_1, \ldots, \phi_n$.
    \end{enumerate}
\end{boxlemma}

We do not prove this result either, as it merely involves manual verification.

\begin{boxexample}
    For propositional variables $p_1, p_2$, the statement $\parenth{\parenth{\parenth{\neg p_2} \to \parenth{\neg p_1}} \to \parenth{p_1 \to p_2}}$ is a tautology. Therefore, if $\phi_1$ and $\phi_2$ are propositional formulae, then $\parenth{\parenth{\parenth{\neg \phi_2} \to \parenth{\neg \phi_1}} \to \parenth{\phi_1 \to \phi_2}}$ is a tautology as well.
\end{boxexample}

We will also mention that a composition being a tautology does not mean the outermost proposition of the composition is a tautology.

\begin{boxnexample}
    Let $p$ be a propositional variable. The formula $\phi : \parenth{p \to \parenth{\neg p}}$ is not a tautology. However, we can find a propositional formula $\phi'$ such that $\parenth{\phi_1 \to \parenth{\neg \phi_1}}$ is a tautology: for example, we can define $\phi'$ to be identically $\F$.
\end{boxnexample}

There are numerous propositional formulae that we know to be logically equivalent. Here is a (non-exhaustive) list.

\begin{boxexample}[Logically Equivalent Formulae]
    Let $p_1, p_2, p_3$ be logically equivalent formulae. Then, the following equivalences hold.
    \begin{enumerate}[noitemsep]
        \item $\parenth{p_1 \land \parenth{p_2 \land p_3}}$ is logically equivalent to $\parenth{\parenth{p_1 \land p_2} \land p_3}$.
        
        \item $\parenth{p_1 \lor \parenth{p_2 \lor p_3}}$ is logically equivalent to $\parenth{\parenth{p_1 \lor p_2} \lor p_3}$.

        \item $\parenth{p_1 \lor \parenth{p_2 \land p_3}}$ is logically equivalent to $\parenth{\parenth{p_1 \lor p_2} \land \parenth{p_1 \land p_3}}$.

        \item $\parenth{\neg \parenth{\neg p_1}}$ is logically equivalent to $p_1$.

        \item $\parenth{\neg \parenth{p_1 \land p_2}}$ is logically equivalent to $\parenth{\parenth{\neg p_1} \lor \parenth{\neg p_2}}$.

        \item $\parenth{\neg \parenth{p_1 \lor p_2}}$ is logically equivalent to $\parenth{\parenth{\neg p_1} \land \parenth{\neg p_2}}$.
    \end{enumerate}
\end{boxexample}

Upon inspection, one can find algebraic patterns in the above logical equivalences. There are similarities to the axioms of a \textbf{boolean algebra}. % TODO: READ UP ON THIS
We will not explore this further in this module, but we will adopt the convention used in algebra where parentheses are dropped when dealing with associative operations.

\begin{boxconvention}
    We will denote both $\parenth{p_1 \land \parenth{p_2 \land p_3}}$ and $\parenth{\parenth{p_1 \land p_2} \land p_3}$ by $\parenth{p_1 \land p_2 \land p_3}$. Similarly, we will denote both $\parenth{p_1 \lor \parenth{p_2 \lor p_3}}$ and $\parenth{\parenth{p_1 \lor p_2} \lor p_3}$ by $\parenth{p_1 \lor p_2 \lor p_3}$.
\end{boxconvention}

We will end with a combinatorial fact about truth functions.

\begin{boxlemma}
    There are $2^{2^n}$ possible truth functions on $n$ variables.
\end{boxlemma}
\begin{proof}
    A truth function is any function from a the set $\set{\T, \F}^n$ to the set $\set{\T, \F}$, with no further restrictions. The former set has $2^n$ elements and the latter set has $2$ elements. Therefore, there are $2^{2^n}$ possible truth functions.
\end{proof}

\subsection{Adequacy}

We have defined several connectives so far, but we have yet to say anything about whether we will be defining any more connectives going forward. To begin, we will state an important definition.

\begin{boxdefinition}[Adequacy]\label{Ch1:Def:Adequacy}
    We say that a set $S$ of connectives is \textbf{adequate} if for every $n \geq 1$, every truth function on $n$ variables can be expressed as the truth function as a propositional formula which only involves connectives from $S$ (and $n$ propositional variables).
\end{boxdefinition}

The idea that this definition seeks to express is that a set is adequate if and only if for evern $n$, every propositional formula in $n$ variables is logically equivalent to a propositional formula that only contains those $n$ variables and connectives from the set in question. In other words, every propositional formula should admit an equivalent expression that does not contain any connectives apart from those in the set in question. The reason this is expressed in terms of truth functions is that that is how logical equivalence is \textit{defined} (cf. \Cref{Ch1:Def:LogicalEquivalence}).

We now have the first theorem of this module.

\begin{boxtheorem}\label{Ch1:Thm:DNFConnectivesAdequate}
    The set $\set{\neg, \land, \lor}$ is adequate.
\end{boxtheorem}
\begin{proof}
    Fix some $n \geq 1$, and let $G : \set{\T, \F}^n \to \set{\T, \F}$ be a truth function. We have two cases.

    \begin{description}
        \item[\textbf{\underline{Case 1.}}] The first case is a trivial case. There are two trivial truth functions on $n$ variables, namely, the constrant truth functions that take the values $\T$ and $\F$ for all inputs. Truth is not something encoded `naturally' into the connectives $\set{\neg, \land, \lor}$, but falsity is: the $\neg$ connective directly has to do with expressing falsity. Therefore, the trivial truth function that we will show can always be expressed in terms of the desired connectives is the one that is always false. We show this rigorously.
        
        Assume that $G$ is identically $\F$. Then, define the propositional formula $\phi : \parenth{p_1 \land \parenth{\neg p_1}}$. Even defining it as a formula on $n$ variables, it is clear to see that its truth function $F_{\phi}$ is identically $\F$. Therefore, $G = F_{\phi}$.\footnote{Admittedly, we are using the Axiom of Extensionality here to define what it means for the two functions to be equal. We will ignore this technicality for now.}
        
        \item[\textbf{\underline{Case 2.}}] The second case will be the nontrivial case of when a truth function can take on both values $\T$ and $\F$. The way we will show that $\set{\neg, \land, \lor}$ is adequate is by constructing a propositional formula in $n$ variables whose truth function is $\T$ whenever the one in question is $\T$. We will do this by isolating the inputs that yield $\T$ and manipulating propositional variables in a way that corresponds to these inputs.
        
        Assume that $G$ is not identically $\T$. Then, list all $v \in \set{\T, \F}^n$ such that $G(v) = \T$. Since $\set{\T, \F}^n$ is a finite set, this list is finite, and we can number these $v_1, \ldots, v_r$. For each $1 \leq i \leq r$, denote
        \begin{align*}
            v_i = \parenth{v_{i1}, \ldots, v_{ir}}
        \end{align*}
        where $v_{ij} \in \set{\T, \F}$ is the $j$th component of $v_i$. Let $p_1, \ldots, p_n$ be propositional variables. Define propositional formulae $\parenth{q_{ij}}_{1 \leq i \leq r, 1 \leq j \leq n}$ by
        \begin{align*}
            q_{ij} : \begin{cases}
                p_j & \text{if } v_{ij} = \T \\
                \parenth{\neg p_j} & \text{if } v_{ij} = \F
            \end{cases}
        \end{align*}
        Then, $q_{ij}$ has value $\T$ if and only if $p_j$ has value $v_{ij}$. The idea is to now construct a propositional formula that has value $\T$ if and only if $\parenth{p_1, \ldots, p_n}$ is one of the $v_i$.
        
        First, we formalise the notion of the $\parenth{p_1, \ldots, p_n}$ taking the value of one of the $v_i$. The idea is to combine them using the $\land$ connective. Define propositional formulae $\parenth{\psi_i}_{1 \leq i \leq r}$ by
        \begin{align*}
            \psi_i : \parenth{q_{i1} \land \cdots \land q_{in}}
        \end{align*}
        Then, we have that for all $1 \leq i \leq r$ and $v \in \set{\T, \F}^n$,
        \begin{align*}
            F_{\psi_i}(v) = \T
            \iff
            q_{i1}, \ldots, q_{in} \text{ all have value } \T
            \iff
            \text{Each } p_j \text{ has value } v_{ij}
            \iff
            v = v_i
        \end{align*}
        Next, we combine these $\psi_i$ so that the truth function of the resulting formula is $\T$ if and only if one of the $\psi_i$ is true, a fact that would be equivalent to the input of the truth function  being precisely one of the $v_i$. We do this using the $\lor$ connective. Define the propositional formula
        \begin{align*}
            \vartheta : \parenth{\psi_1 \lor \cdots \lor \psi_r}
        \end{align*}
    \end{description}
    Then, for all $v \in \set{\T, \F}^n$, we have that
    \begin{align*}
        F_{\vartheta}(v) = \T
        \iff
       \text{One of the } \psi_i \text{ is true}
        \iff
        v \text{ is precisely equal to one of the } v_i
    \end{align*}
    In partiular, we have that $F_{\vartheta}(v) = \T$ if and only if $G(v) = \T$ for all $v \in \set{\T, \F}^n$. Then, by \Cref{Ch1:Lemma:TruthFunEq}, we are done.
\end{proof}

Before illustrating the point of the above theorem, we make an important definition.

\begin{boxdefinition}[Disjunctive Normal Form]
    When a propositional formula is expressed only in terms of propositional variables and the set $\set{\neg, \land, \lor}$ of connectives, it is said to be in \textbf{disjunctive normal form}, which we abbreviate to \textbf{DNF}.
\end{boxdefinition}

What \Cref{Ch1:Thm:DNFConnectivesAdequate} then tells us is that every propositional formula is expressible in DNF.

\begin{boxcorollary}
    For every propositional formula in $n$ variables, there exists a logically equivalent propositional formula in $n$ variables that is in DNF.
\end{boxcorollary}
\begin{proof}
    We know that every propositional formula admits a truth function. For any propositional formula in $n$ variables, we can apply \Cref{Ch1:Thm:DNFConnectivesAdequate} to its truth function. Then, unfolding the definition of adequacy yields the desired result.
\end{proof}

\begin{boxexample}
    Let $p_1$ and $p_2$ be propositional variables. Consider the propositional formula $\chi : \parenth{\parenth{p_1 \to p_2} \to \parenth{\neg p_2}}$. We can see that $F_{\chi}(v) = \T$ only if $v = \parenth{\T, \F}$ or $v = \parenth{\F, \F}$. Therefore, the DNF of $\chi$ is
    \begin{align*}
        \parenth{\parenth{p_1 \land \parenth{\neg p_2}} \lor \parenth{\parenth{\neg p_1} \land \parenth{\neg p_2}}}
    \end{align*}
\end{boxexample}

It turns out that $\set{\neg, \land, \lor}$ is not the only adequate set of connectives.

\begin{boxexample}[Adequate Sets]
    The following sets of connectives are adequate.
    \begin{enumerate}[noitemsep, label = \normalfont(\roman*)]
        \item\label{Ch1:Lemma:AdequateSets:Case:1} $\set{\neg, \lor}$
        \item $\set{\neg, \land}$
        \item $\set{\neg, \to}$
    \end{enumerate}
    The way we can prove this is by simplifying each case using \Cref{Ch1:Thm:DNFConnectivesAdequate}. Fix propositional variables $p_1, p_2$.
    \begin{enumerate}[label = \normalfont(\roman*)]
        \item It suffices to show that $p_1 \land p_2$ can be expressed using $\neg$ and $\lor$. Indeed,
        \begin{align*}
            \parenth{p_1 \land p_2} \isleto \parenth{\neg\parenth{\parenth{\neg p_1} \lor \parenth{\neg p_2}}}
        \end{align*}
        
        \item It suffices to show that $p_1 \lor p_2$ can be expressed using $\neg$ and $\land$. Indeed,
        \begin{align*}
            \parenth{p_1 \lor p_2} \isleto \parenth{\neg\parenth{\parenth{\neg p_1} \land \parenth{\neg p_2}}}
        \end{align*}

        \item By Case~\ref{Ch1:Lemma:AdequateSets:Case:1}, it suffices to show that $p_1 \lor p_2$ can be expressed in terms of $\neg$ and $\to$. Indeed,
        \begin{align*}
            \parenth{p_1 \lor p_2} \isleto \parenth{\parenth{\neg p_1} \to p_2}
        \end{align*}
    \end{enumerate}
\end{boxexample}

There are also sets of connectives that are not adequate.

\begin{boxnexample}[Inadequate Sets]
    The following sets are not adequate.
    \begin{enumerate}[label = (\roman*)]
        \item $\set{\land, \lor}$
        \item $\set{\neg, \lr}$
    \end{enumerate}
    The way we can prove this is by constructing truth functions that cannot be realised by combining propositional variables using only the connectives in the above sets.
    \begin{enumerate}[label = (\roman*)]
        \item No truth function that is identically false can be realised. For that matter, no truth function that maps an input whose every component is $\T$ to $\F$ can be realised. Formally, consider any propositional formula $\phi$ built exclusively using a finite set of propositional variables and the connectives $\land$ and $\lor$. One can show, by induction on the number of connectives in $\phi$, that $F_{\phi}(\T, \ldots, \T) = \T$. Since this is true of any $\phi$, a truth function mapping an input of the form $(\T, \ldots, \T)$ to $\F$ is not the truth function of a propositional formula that only includes $\land$ and $\lor$.
        
        \item No truth function that is identcally true can be realised.
    \end{enumerate}
\end{boxnexample}

It turns out that there is one connective with a rather astounding adequacy property.

\begin{boxdefinition}[The NOR Connective]
    Define the \textbf{NOR connective}, denoted $\nor$, via the following truth table in propositional variables $p$ and $q$.
    \begin{table}[H]
        \centering
        \begin{tabular}{cc|c}
            $p$ & $q$ & $\parenth{p \nor q}$ \\ 
            \hline
            $\T$ & $\T$ & $\F$ \\
            $\T$ & $\F$ & $\F$ \\
            $\F$ & $\T$ & $\F$ \\
            $\F$ & $\F$ & $\T$
        \end{tabular}
    \end{table}
\end{boxdefinition}

Informally, NOR corresponds to ``neither ... nor ...''. Formally, we have the following.

\begin{boxlemma}
    For all propositional variables $p$ and $q$, the DNF of $\parenth{p \nor q}$ is given by $\parenth{\parenth{\neg p} \land \parenth{\neg q}}$. In particular, we have that $\parenth{p \nor q} \isleto \parenth{\parenth{\neg p} \land \parenth{\neg q}}$.
\end{boxlemma}
We do not write out a proof, as it merely involves comparing truth tables.

\begin{boxexample}[An Adequate Set with One Connective]
    It turns out that $\set{\nor}$ is connective. Indeed, for propositional variables $p$ and $q$, we have
    \begin{enumerate}
        \item $\parenth{p \nor p} \isleto \parenth{\neg p}$.
        \item $\parenth{\parenth{p \nor p} \nor \parenth{q \nor q}} \isleto \parenth{p \land q}$.
    \end{enumerate}
\end{boxexample}

So far, we have been studying \textit{meaning}, in the form of truth functions, but have yet to formally define \textit{what} we are allowed to express that \textit{has} meaning within the propositional paradigm. In other words, we have been studying \textbf{semantics} but have yet to define the \textbf{syntax} of propositional logic. We will do this in the next section.

% \section{Lie Algebras of Dimension $\leq 3$}

It turns out that we do not need any particularly sophisticated machinery to classify \underline{all} Lie algebras of dimension less than or equal to $3$.

\subsection{Abelian Lie Algebras and Lie Algebras of Dimension $1$}

We begin with a simple observation about abelian Lie algebras.

\begin{boxproposition}\label{Ch1:Prop:Abelian_Lie_Algebras_Iso}
    Fix $n \in \N$. Then, any abelian Lie algebra of dimension $n$ is isomorphic to $\C^n$ with the zero bracket.
\end{boxproposition}
\begin{proof}
    Let $L$ be a Lie algebra of dimension $n$. We know there exists a $\C$-linear isomorphism $\phi : L \to \C^n$. It follows immediately that for any $x, y \in L$,
    \begin{align*}
        \phiof{\brac{x, y}} = \phiof{0} = 0 = \brac{\phiof{x}, \phiof{y}}
    \end{align*}
    A similar argument will show that $\phi\inv : \C^n \to L$, viewed as a linear map, is a Lie algebra homomorphism as well, proving that $L \cong \C^n$.
\end{proof}

The classification of Lie algebras in $1$ dimension is then straightforward. We will begin by a rather strong but straightforward result on one-dimensional subspaces of Lie algebras.

\begin{proposition}\label{Ch1:Prop:1D_Lie_Subalgebras}
    Let $L$ be a Lie algebra. Any $1$-dimensional subspace of $L$ is an abelian Lie subalgebra.
\end{proposition}
\begin{proof}
    Let $K$ be a sub-vector space of dimension $1$. We know any $\C$-basis of $K$ consists of a single, nonzero element. Consider such a basis element $x$. For any $y_1, y_2 \in L$, there exist $\lambda_1, \lambda_2 \in \C$ such that $y_1 = \lambda_1 x$ and $y_2 = \lambda_2 x$. Then,
    \begin{align*}
        \brac{y_1, y_2} = \brac{\lambda_1 x_1, \lambda_1 x_2} = \lambda_1 \lambda_2 \brac{x, x} = 0
    \end{align*}
    proving that $\liebrac = 0$. Since $K$ is a subspace, $0 \in K$, proving that $K$ is a Lie subalgebra.
\end{proof}

The classification of Lie algebras of dimension $1$ is then immediate.

\begin{boxcorollary}\label{Ch1:Cor:1D_Lie_Algebra_Classification}
    Any Lie algebra of dimension $1$ is abelian, isomorphic to $\C$ equipped with the zero bracket.
\end{boxcorollary}
\begin{proof}
    Let $L$ be a Lie algebra of dimension $1$. That $L$ is abelian follows from applying \Cref{Ch1:Prop:1D_Lie_Subalgebras} to $L$ viewed as a subspace of itself. The isomorphism then follows immediately from \Cref{Ch1:Prop:Abelian_Lie_Algebras_Iso}.
\end{proof}

We can now turn our attention to the slightly more non-trivial problem of classifying non-abelian Lie algebras of dimension $2$ and $3$.

\subsection{Lie Algebras of Dimension $2$}

From \Cref{Ch1:Prop:Abelian_Lie_Algebras_Iso}, we already know that there is only one abelian Lie algebra of dimension $2$. The question remains, how many non-abelian Lie algebras of dimension $2$ are there?

We begin by giving an example.

\begin{boxexample}[A Two-Dimensional Non-Abelian Lie Algebra]\label{Ch1:Eg:2D_NonAbelian_Lie_Algebra}
    Consider the set
    \begin{align*}
        \r_2 := \setst{
            \begin{bmatrix}
                a & b \\ 0 & 0
            \end{bmatrix}
        }{a, b \in \C}
        = \Span{\begin{bmatrix} 1 & 0 \\ 0 & 0 \end{bmatrix}, \begin{bmatrix} 0 & 1 \\ 0 & 0 \end{bmatrix}}
        \subseteq \gl{2}
    \end{align*}
    Clearly, $\r_2$ is a linear subspace of $\gl{2}$. Furthermore, One can show that
    \begin{align*}
        \brac{
            \begin{bmatrix} 1 & 0 \\ 0 & 0 \end{bmatrix}, \begin{bmatrix} 0 & 1 \\ 0 & 0 \end{bmatrix}
        } = \begin{bmatrix} 0 & 1 \\ 0 & 0 \end{bmatrix}
    \end{align*}
    proving that $\r_2$ is closed under the commutator bracket. It follows that $\r_2$ is a Lie subalgebra of $\gl{2}$, and therefore, a $2$-dimensional Lie algebra in its own right.
\end{boxexample}

% The \r_2 notation comes from French literature: the \r stands for resoluble (solvable), and it is clear that the above Lie algebra is solvable.

The reason we are interested in the above example will become clear, and we will reserve the notation $\r_2$ for this particular Lie algebra. For the remainder of this section, denote by $L$ an arbitrary non-abelian Lie subalgebra of dimension $2$.

We will begin by describing the derived subalgebra $L'$ (cf. \Cref{Ch1:Def:DerivedSubalg}) of $L$.

\begin{lemma}
    For any $\C$-basis $\set{u, v}$ of $L$, we have that $L' = \Span{\brac{u, v}}$.
\end{lemma}
\begin{proof}
    Let $\set{u, v}$ be a basis of $L$. Define $x := \brac{u, v}$. Since $L$ is non-abelian, $x \neq 0$, making $X := \Span{x}$ a $1$-dimensional subspace of $L$. Seeing as $L' = \brac{L, L} = \Span{\setst{\brac{x, y}}{x, y \in L}}$, it is clear that $L' \supseteq X$. It remains to show that $L' \subseteq X$.

    It suffices to show that $\setst{\brac{x, y}}{x, y \in L} \subseteq X$. To that end, fix $a, b \in L$. We know there exist $\lambda_1, \mu_1, \lambda_2, \mu_2 \in \C$ such that $a = \lambda_1 u + \mu_1 v$ and $b = \lambda_2 u + \mu_2 v$. Then,
    \begin{align*}
        \brac{a, b}
        &= \brac{\lambda_1 u + \mu_1 v, \lambda_2 u + \mu_2 v} \\
        &= \lambda_1 \lambda_2 \underbrace{\brac{u, u}}_{= 0} + \lambda_1 \mu_2 \brac{u, v} + \mu_1 \lambda_2 \brac{v, u} + \mu_1 \mu_2 \underbrace{\brac{v, v}}_{= 0} \\
        &= \parenth{\lambda_1 \mu_2 - \mu_1 \lambda_2} \brac{u, v} \in X
    \end{align*}
    as required.
\end{proof}

This tells us, in particular, that the span of the commutator of any basis of $L$ is an ideal. We now have everything we need to describe $L$.

\begin{boxproposition}
    $L$ is isomorphic to $\r_2$.
\end{boxproposition}
\begin{proof}
    It suffices to show that $L$ admits a basis $\set{x, y}$ such that $\brac{x, y} = y$, as this will readily yield the right structure constants.\footnote{Alternatively, if we can show that $\brac{x, y} = y$, it will follow immediately that the linear isomorphism sending $x$ to $\begin{bmatrix} 1 & 0 \\ 0 & 0 \end{bmatrix}$ and $y$ to $\begin{bmatrix} 0 & 1 \\ 0 & 0 \end{bmatrix}$ is, indeed, a Lie algebra isomorphism.}
    
    Let $\set{u, v}$ be an arbitrary $\C$-basis of $L$. Let $y := \brac{u, v}$. Since $L$ is non-abelian, $y \neq 0$. Therefore, there exists some $z \in L \setminus \set{0}$ that is linearly independent of $y$. Since $\Span{y} = L' \nsg L$, we know that $\brac{z, y} \in L'$. In particular, $\exists \lambda \in \C$ such that $\brac{z, y} = \lambda y$. Furthermore, since $y$ and $z$ are linearly independent and $L$ is non-abelian, $\lambda \neq 0$. So, define $x := \lambda\inv z$. Then, $x$ is still linearly independent of $y$, making $\set{x, y}$ a basis of $L$, and $\brac{x, y} = y$, as required.
\end{proof}

Yes, it's true! Up to isomorphism, there is \underline{only one} non-abelian Lie algebra of dimension $2$. Therefore, there are \underline{only two} Lie algebras of dimension $2$: one non-abelian one and one abelian one.

We can now turn our attention to the classification of Lie algebras in dimension $3$.

\subsection{Lie Algebras of Dimension $3$}

\sorry

% \section{Solvability and Nilpotency}

We now begin discussing some nontrivial objects in the theory of Lie algebras. Throughout this section, $L$ will denote an arbitrary Lie algebra.

\subsection{Descending Series of Ideals}

\begin{boxdefinition}[Derived Series]
    The \textbf{derived series} of $L$ is the descending series of ideals
    \begin{align*}
        L = L^{(0)} \supseteq L^{(1)} \supseteq L^{(2)} \supseteq \cdots % \supseteq L^{(n - 1)} \supseteq L^{(n)}
    \end{align*}
    where $L^{(i)} := \brac{L^{(i - 1)}, L^{(i-1)}}$ for $i \geq 1$. % \leq n$.
\end{boxdefinition}

\begin{boxdefinition}[Solvability]
    $L$ is said to be \textbf{solvable} if there exists an $n \in \N$ such that $L^{(n)} = 0$.
\end{boxdefinition}

\begin{boxdefinition}[Lower Central Series]
    The \textbf{lower central series} of $L$ is the descending series of ideals
    \begin{align*}
        L = L^{0} \supseteq L^{1} \supseteq L^{2} \supseteq \cdots % \supseteq L^{n - 1} \supseteq L^{n}
    \end{align*}
    where $L^{i} := \brac{L, L^{i-1}}$ for $i \geq 1$. % \leq n$.
\end{boxdefinition}

\begin{boxconvention}
    Elements of the derived series are denoted $L^{(i)}$, with parenthesised superscript indices, whereas elements of the lower central series are denoted $L^{i}$, with no parentheses around the indices.
\end{boxconvention}

\begin{boxdefinition}[Nilpotency]
    $L$ is said to be \textbf{nilpotent} if there exists an $n \in \N$ such that $L^{n} = 0$.
\end{boxdefinition}

Indeed, there is the following relationship between solvability and nilpotency.

\begin{lemma}\label{Ch1:Lemma:DerivedSeriesContainedInLowerCentralSeries}
    For all $i \in \N$, $L^i \supseteq L^{(i)}$.
\end{lemma}
\begin{proof}
    We argue by induction on $i$. The base case is trivial, because $L^0 = L = L^{(0)}$. Now, fix $i \in \N$ and assume that $L^i \supseteq L^{(i)}$. Then,
    \begin{align*}
        L^{i + 1}
        = \brac{L, L^i} &= \brac{L, L^{(i)}} \\
        &= \Span{\setst{\brac{\ell, x}}{x \in L^{(i)}, \ell \in L}} \\
        &\supseteq \Span{\setst{\brac{\ell, x}}{x \in L^{(i)}, \ell \in L^{(i)}}} \\
        &= \brac{L^{(i)}, L^{(i)}} = L^{(i + 1)}
    \end{align*}
    where the inclusion on the third line follows from the fact that $L^{(i)} \subseteq L$. This completes the induction and proves the desired result for all $i \in \N$.
\end{proof}

\begin{boxcorollary}
    If $L$ is nilpotent, then $L$ is solvable.
\end{boxcorollary}
\begin{proof}
    \Cref{Ch1:Lemma:DerivedSeriesContainedInLowerCentralSeries} tells us that for all $n \in \N$, $L^n = 0$ implies $L^{(n)} = 0$. Thus, if such an $n$ exists that makes $L$ nilpotent, the same $n$ would also make $L$ solvable.
\end{proof}

The converse is not true. We need to develop a bit of theory to see this.

\begin{lemma}\label{Ch1:Lemma:NilpotentCentreNonzero}
    If $L$ is nilpotent, its centre is nonzero.
\end{lemma}
\begin{proof}
    Let $n \in \N$ be such that
    \begin{align*}
        L = L^0 \supseteq L^1 \supseteq \cdots \supseteq L^n \subsetneq L^{n+1} = 0
    \end{align*}
    with $L^n \neq 0$.
\end{proof}

\subsection{Ideals, Quotients and Subalgebras}

Throughout this subsection, let $I \nsg L$ and $K \leq L$. Recall that $I$ is a Lie subalgebra of $L$ (cf. \Cref{Ch1:Lemma:IdealSubalg}), meaning we can impose solvability and nilpotency conditions on $I$ as well.

\begin{definition}[Solvability of Subalgebras]
    We say a subalgebra of $L$ is \textbf{solvable} if it is solvable as a Lie algebra in its own right.
\end{definition}

\begin{boxproposition}[Solvability Conditions]\label{Ch1:Prop:SolvabilityConditions}
    \hfill
    \begin{enumerate}[label = \normalfont\arabic*., noitemsep]
        \item If $L$ is solvable, then so is $\quotient{L}{I}$.
        \item If $L$ is solvable, then so is $K$.
        \item If $I$ and $\quotient{L}{I}$ are solvable, then so is $L$.
    \end{enumerate}
\end{boxproposition}
\begin{proof}
    Let $\phi : L \surj \quotient{L}{I}$ be the quotient homomorphism.
    \begin{enumerate}
        \item Observe that it suffices to show that $\phiof{L^{(i)}} = \phiof{L}^{(i)}$ for all $i \in \N$: if this were true, then the existence of some $n \in \N$ such that $L^{(n)} = 0$ would imply that
        \begin{align*}
            \parenth{\quotient{L}{I}}^{(n)} = \phiof{L}^{(n)} = \phiof{L^{(n)}} = \phiof{0} = 0
        \end{align*}
        making $\quotient{L}{I}$ solvable whenever $L$ is.
        
        We will now prove that $\phiof{L^{(i)}} = \phiof{L}^{(i)}$ by induction on $i$. When $i = 0$, the result is trivial: it is true that $\phiof{L} = \phiof{L}$ by reflexivity. % Lean-like?
        Now, fix $i \in \N$ and assume that $\phiof{L^{(i)}} = \phiof{L}^{(i)}$. Then,
        \begin{align*}
            \phiof{L^{(i + 1)}} = \phiof{\brac{L^{(i)}, L^{(i)}}}
            &= \phiof{\Span{\setst{\brac{x,y}}{x, y \in L^{(i)}}}} \\
            &= \Span{\phiof{\setst{\brac{x,y}}{x, y \in L^{(i)}}}} \\
            &= \Span{\setst{\brac{\phiof{x}, \phiof{y}}}{x, y \in L^{(i)}}} \\
            &= \brac{\phiof{L^{(i)}}, \phiof{L^{(i)}}} \\
            &= \brac{\phiof{L}^{(i)}, \phiof{L}^{(i)}} = \phiof{L}^{(i + 1)}
            % &= \brac{\parenth{\quotient{L}{I}}^{(i)}, \parenth{\quotient{L}{I}}^{(i)}}
            % = \parenth{\quotient{L}{I}}^{(i + 1)}
        \end{align*}
        as required.

        \item It suffices to prove that for all $i \in \N$, $K^{(i)} \subseteq L^{(i)}$: if this were true, then the existence of some $n \in \N$ such that $L^{(n)} = 0$ would imply that $K^{(n)} = 0$, making $K$ solvable whenever $L$ is.
        
        We will now prove that $K^{(i)} \subseteq L^{(i)}$ by induction on $i$. The base case is trivial, because $K^{(0)} = K \subseteq L = L^{(0)}$. Now, fix $i \in \N$ and assume that $K^{(i)} \subseteq L^{(i)}$. Then,
        \begin{align*}
            K^{(i + 1)} = \brac{K^{(i)}, K^{(i)}}
            &= \Span{\setst{\brac{x,y}}{x, y \in K^{(i)}}} \\
            &\subseteq \Span{\setst{\brac{x,y}}{x, y \in L^{(i)}}} \\
            &= \brac{L^{(i)}, L^{(i)}} = L^{(i + 1)}
        \end{align*}
        as required.

        \item Let $m \in \N$ be such that $I^{(m)} = 0$ and let $n \in \N$ be such that $\parenth{\quotient{L}{I}}^{(n)} = 0$. It suffices to prove that for all $i, j \in \N$, $\parenth{L^{(i)}}^{(j)} = L^{(i + j)}$: if this were true, then the fact that
        \begin{align*}
            \phiof{L^{(n)}} = \parenth{\quotient{L}{I}}^{(n)} = 0
        \end{align*}
        would immediately imply that $L^{(n)} \subseteq \pker{\phi} = I$, from which it would follow that $\parenth{L^{(n)}}^{(m)} = 0$, and therefore, that $L^{(n + m)} = 0$, making $L$ solvable whenever $I$ and $\quotient{L}{I}$ are.

        We will now prove that $\parenth{L^{(i)}}^{(j)} = L^{(i + j)}$ by letting $i$ be arbitrary and performing induction on $j$. The base case is trivial, because $\parenth{L^{(i)}}^{(0)} = L^{(i)}$. Now, fix $j \in \N$ and assume that $\parenth{L^{(i)}}^{(j)} = L^{(i + j)}$. Then,
        \begin{align*}
            \parenth{L^{(i)}}^{(j + 1)} = \brac{\parenth{L^{(i)}}^{(j)}, \parenth{L^{(i)}}^{(j)}} = \brac{L^{(i + j)}, L^{(i + j)}} = L^{(i + j + 1)}
        \end{align*}
        as required.
    \end{enumerate}
\end{proof}

We have similar results for nilpotency.

\begin{definition}[Nilpotency of Subalgebras]
    We say a subalgebra of $L$ is \textbf{nilpotent} if it is solvable as a Lie algebra in its own right.
\end{definition}

\begin{boxproposition}[Nilpotency Conditions]
    \hfill
    \begin{enumerate}[label = \normalfont\arabic*., noitemsep]
        \item If $L$ is nilpotent, then so is $\quotient{L}{I}$.
        \item If $L$ is nilpotent, then so is $K$.
    \end{enumerate}
\end{boxproposition}

We will not prove these results here, as they are very similar to the corresponding results for solvability. We will, however, mention that the reason why we do not have a nilpotency condition for $L$ when $I$ and $\quotient{L}{I}$ are nilpotent is that it is not, in general, true that $\parenth{L^i}^j = L^{i + j}$ for $i, j \in \N$, as we can easily see from the following counterexample.

\begin{boxcexample}
    \sorry
\end{boxcexample}

We will end the discussion on solvability and nilpotency by saying a bit about the derived subalgebra. We first make a general observation.

\begin{lemma}\label{Ch1:Lemma:AllDerivsOfDerivEqTop}
    If $L = L'$, then for all $i \in \N$, $L^{(i)} = L^{(1)} = L' = L$.
\end{lemma}
\begin{proof}
    We argue by induction on $i$. The base case is trivial, because $L^{(0)} = L$. Now, fix $i \in \N$ and assume that $L^{(i)} = L$. Then,
    \begin{align*}
        L^{(i + 1)} = \brac{L^{(i)}, L^{(i)}} = \brac{L, L} = L'
    \end{align*}
    Furthermore, it is clear that $L^{(1)} = L'$, and, by assumption, $L' = L$. This completes the induction and proves the desired result for all $i \in \N$.
\end{proof}

There is an immediate consequence.

\begin{corollary}\label{Ch1:Cor:DerivedSubalgLtOfSolvable}
    If $L \neq 0$ and $L$ is solvable, then $L' < L$.
\end{corollary}
\begin{proof}
    We argue by contraposition. If $L' = L$, then we know that $L^{(i)} = L$ for all $i \in \N$. In particular, since $L$ is nonzero, none of the $L^{(i)}$ can be zero. $L$ is therefore not solvable.
\end{proof}

We will be more interested the following, somewhat less immediate consequence that comes from combining applying the Correspondence Theorem to the ideals of quotient spaces of solvable Lie algebras.

\begin{boxproposition}\label{Ch1:Prop:ExistsIdealCodim1}
If $L$ is solvable, there exists an ideal $I \nsg L$ of codimension $1$.
\end{boxproposition}
\begin{proof}
    Consider the quotient Lie algebra $K := \quotient{L}{L'}$. We know that $0 < K$, because $K = 0$ would imply that $L = L'$, which is impossible because $L$ is solvable, as shown in \Cref{Ch1:Cor:DerivedSubalgLtOfSolvable}. Therefore, $K$ contains a subspace $W$ of codimension $1$. Since $K$ is abelian, \Cref{Ch1:Prop:SubspaceIdealOfAbelian} tells us that $W$ is an ideal of $K$. \Cref{SP:Thm:Correspondence} tells us that the preimage $V$ of $W$ under the quotient epimorphism is an ideal of $L$ that contains $L'$. Simple arithmetic and dimension results from linear algebra then tell us
    \begin{align*}
        \pdim{V} = \pdim{W} + \pdim{L'} = \parenth{\pdim{L} - \pdim{L'} - 1} + \pdim{L'} = \pdim{L} - 1
    \end{align*}
\end{proof}

\subsection{The Radical Ideal}

Throughout this subsection, we will assume that $L$ is finite-dimensional.

We begin with a basic result about the sums of ideals.

\begin{lemma}\label{Ch1:Lemma:SumIdealSolvable}
    Let $I, J \nsg L$. If $I$ and $J$ are solvable, then so is $I + J \nsg L$.
\end{lemma}
\begin{proof}
    \sorry
    % Apply 2nd iso thm
\end{proof}

\begin{boxcorollary}\label{Ch1:Cor:RadExists}
    There exists a solvable ideal of $L$ that contains all other solvable ideals of $L$.
\end{boxcorollary}
\begin{proof}
    Let $R$ be a solvable ideal of $L$ of maximal dimension.\footnote{When we say maximal dimension, we mean that the dimension of $R$ is the largest possible dimension such that a solvable ideal of that dimension exists. This is well-defined because $L$ is finite-dimensional, and the dimension of any ideal of $L$ is necessarily $\leq \pdim{L}$.} Now, fix any $I \nsg L$. \Cref{Ch1:Lemma:SumIdealSolvable} tells us that $I + R$ is a solvable ideal of $L$. But, since $R$ is of maximal dimension, we know that $\pdim{I + R} \leq \pdim{R}$. Therefore, we must have that $I + R = R$. The fourth point in \Cref{Ch1:Prop:IdealBhv} then tells us that $I \subseteq R$, as required.
\end{proof}

This solvable ideal has a name.

\begin{boxdefinition}[Radical Ideal]
    The \textbf{radical ideal} of $L$ is the solvable ideal of $L$ that contains all other solvable ideals of $L$, which we know exists from \Cref{Ch1:Cor:RadExists}.
\end{boxdefinition}

We can now define what it means for a Lie algebra to be semi-simple. We will be very interested in this class of Lie algebras going forward.

\begin{boxdefinition}[Semi-Simplicity]\label{Ch1:Def:SemiSimple}
    We say that $L$ is \textbf{semi-simple} if its radical ideal is the $0$ ideal.
\end{boxdefinition}

Our aim for this module will be to classify all semi-simple Lie algebras. We will do this by first classifying all solvable Lie algebras and then using that classification to classify all semi-simple Lie algebras. We will need a \textit{lot} more machinery before we can do this, but we will get there eventually.

We also have a notion of simplicity, which is no different from what we would expect in groups.

\begin{boxdefinition}[Simplicity]
    We say that $L$ is \textbf{simple} if it has no nontrivial ideals.
\end{boxdefinition}

\subsection{Ascending Series of Ideals}

We will end by talking about ascending series of ideals and their corresponding quotients. Throughout this subsection, $L$ will denote an arbitrary Lie algebra.

\begin{boxdefinition}[Ascending Central Series]
    We say that an increasing chain of ideals
    \begin{align*}
        0 \subseteq L_1 \subseteq L_2 \subseteq L_3 \subseteq \cdots
    \end{align*}
    is an \textbf{ascending central series} of $L$ if $L_1 = \Zof{L}$ and for all $i \in \N$, we have
    \begin{enumerate}
        \item $L_i \nsg L$ with quotient map $g_i : L \surj \quotient{L}{L_i}$
        \item $L_{i + 1} = g_i\inv\!\parenth{\Zof{\quotient{L}{L_i}}}$
    \end{enumerate}
\end{boxdefinition}

\begin{boxconvention}
    We will use subscripted $L_i$s to denote elements of the ascending central series, in contrast to superscripts used for the descending central series.
\end{boxconvention}

We now have an equivalent criterion for nilpotency.

\begin{boxproposition}
    $L$ is nilpotent if and only if $L_n = L$ for some $n \in \N$.
\end{boxproposition}
\begin{proof} \hfill
    \begin{description}
        \item[$\parenth{\implies}$]
            One can show by induction on $n$ that if $L^n = 0$, then $L_n = L$. Then, if $\exists n \in \N$ such that $L^n = 0$, ie, if $L$ is nilpotent, then $L_n = 0$ as well.  \sorry % (Same n works for both)

        \item[$\parenth{\impliedby}$] 
            % Use fact that image of centre in quotient map is centre of quotient
            % The idea is to show that $L^{k + 1} = \brac{L, L^k} \subseteq L_{n + k - 1}$ and $\brac{L, L^k} \subseteq \brac{L, L_{n - l}}$. And $L_{n - k}$ is the preimage of $\Zof{\quotient{L}{L_{n - k - 1}}}$
            \sorry
    \end{description}
\end{proof}

We end with another counterexample that shows that solvable Lie algebras need not be nilpotent.

\begin{boxcexample}
    For all $n$, $\t{n}$ is solvable but not nilpotent.
    \begin{proof}[Proof that $\t{n}$ is solvable]
        First, observe that $\brac{\t{n}, \t{n}} = \t{n}' = \u{n}$. By \sorry, we know that $\u{n}$ is nilpotent. Therefore, $\u{n}$ is solvable. Furthermore, $\quotient{\t{n}}{\t{n}'}$ is abelian, making it solvable by \sorry. Therefore, by \Cref{Ch1:Prop:SolvabilityConditions}, $\t{n}$ is solvable.
    \end{proof}

    \begin{proof}[Proof that $\t{n}$ is not nilpotent]
        
    \end{proof}
\end{boxcexample}

% \section{Subalgebras of $\gl{n}$}

We now turn our attention to the structure of subalgebras of $\gl{n}$ for some fixed $n \in \N$. We will begin by developing some mroe general theory, following which we will prove important theorems about the structure of such subalgebras.

\subsection{Induced Actions on Quotients by Invariant Subspaces}
\label{Ch1:Subsec:QuotientByInvariantSubspaces}

We will begin by recalling the linear algebraic theory of invariant subspaces and adapt that theory to the context of Lie algebras. Throughout this subsection, we will fix a subset $L \subseteq \gl{n}$ and a subspace $U \leq \C^n$.

\begin{definition}[Invariance]
    We say that $U$ is \textbf{$L$-invariant} if for all $T \in L$ and $v \in U$, we have $\Tv \in U$.
\end{definition}

The action of $L$ on $\C^n$ induces a natural action on the quotient space of $\C^n$ by $U$.

\begin{definition}[Induced Action]
    To each $T \in L$, we can associate the linear map $\Tbar : \quotient{\C^n}{U} \to \quotient{\C^n}{U}$ defined by
    \begin{align}
        \Tbar(v + U) = \Tv + U
    \end{align}
    for all $v + U \in \quotient{\C^n}{U}$. We will refer to the map $T \mapsto \Tbar$ as the \textbf{induced action} of $L$ on $\quotient{\C^n}{U}$.
\end{definition}

Indeed, when $L$ is a Lie subalgebra of $\gl{n}$, we can go one step further.

\begin{boxproposition}\label{Ch1:Prop:InducedActionLieAlg}
    If $L$ is a Lie subalgebra of $\gl{n}$, the induced action map $\Phi : L \to \gl{\quotient{\C^n}{U}} : T \mapsto \Tbar$ is a Lie algebra homomorphism.
\end{boxproposition}
\begin{proof}
    \sorry
    % The idea is to show it to be a homomorphism of associative algebras instead.
\end{proof}

\subsection{Linear Algebraic and Lie Algebraic Nilpotency}

Recall the following definition from linear algebra.

\begin{boxdefinition}[Nilpotency of Elements]
    We say that $x \in \gl{n}$ is \textbf{nilpotent} if there exists an $m \in \N$ such that $x^m = 0$. 
\end{boxdefinition}

We can extend this to sub-vector spaces.

\begin{boxdefinition}[Nilpotency of Subspaces]
    We say a sub-vector space $N \leq \gl{n}$ is \textbf{nilpotent} if every element of $N$ is nilpotent.
\end{boxdefinition}

We can say something about the adjoint of a nilpotent element.

\begin{lemma}\label{Ch1:Lemma:adNilpotnetOfNilpotent}
    Let $x \in \gl{n}$ be nilpotent. Then, $\pad{x} \in \gl{\gl{n}}$ is nilpotent as well.
\end{lemma}
\begin{proof}
    We need to show that there exists an $m \in \N$ such that the map we get by successively composing the adjoint map $\pad{x}$ $m$ times is identically zero.
    
    Fix $y \in \gl{n}$. Then,
    \begin{align*}
        \pad{x}\!(y) = \brac{x, y} &= xy - yx \\
        \pad{x}^2\!(y) = \brac{x, \brac{x, y}} &= x\brac{x, y} - \brac{x, y}x = x^2y - xyx - xyx + yx^2 \\
        \pad{x}^3\!(y) = \brac{x, \brac{x, \brac{x, y}}} &= x^3y + \cdots + xyx^2 - yx^3
    \end{align*}
    More generally, one can show that
    \begin{align*}
        \pad{x}^m\!(y) = \sum_{i = 0}^{m} \lambda_{i, m} x^i y x^{m - i}
    \end{align*}
    for all $m \in \N$ and some $\lambda_{i, m} \in \Z$. In particular, since all powers of $x$ beyond some $m$ are zero, we have that $\pad{x}^m\!(y) = 0$ for all $y \in \gl{n}$.
\end{proof}

We have an important relationship between linear algebraic and lie algebraic nilpotency of a Lie subalgebra.

\begin{boxtheorem}[Engel's Theorem]\label{Ch1:Thm:Engel}
    Let $N$ be a Lie subalgebra of $\gl{n}$. If $N$ is nilpotent as a sub-vector space of $\gl{n}$, then there exists a basis of $\C^n$ with respect to which every element of $N$ is upper-triangular.
\end{boxtheorem}

Before proving Engel's Theorem, we will state and prove the following Corollary that underscores the significance of this result.

\begin{boxcorollary}\label{Ch1:Cor:EngelNilpotency}
    Any nilpotent sub-vector space of $\gl{n}$ is also nilpotent as a Lie subalgebra.
\end{boxcorollary}
\begin{proof}
    Let $N$ be a nilpotent sub-vector space of $\gl{n}$. By Engel's Theorem, there exists a basis of $\C^n$ with respect to which every element of $N$ is upper-triangular. In particular, they must all have zeros on the diagonal, because they are nilpotent: they are of the form
    \begin{align*}
        \begin{bmatrix}
            0 & & * \\
            \vdots & \ddots & \\
            0 & \cdots & 0
        \end{bmatrix}
    \end{align*}
    \sorry
\end{proof}

For the remainder of this subsection, we will focus on proving Engel's Theorem. We will fix a nilpotent subspace $N \leq \gl{n}$. The high-level idea is to perform induction on $\pdim{L}$ and draw a parallel with the proof of the Jordan Canonical Form theorem\footnote{Remember, we are working over $\C$.}. We will first show that it suffices to show that a certain distinguished vector exists, following which we will show that it does.

For the remainder of this subsection, we will denote the \textbf{simultaneous kernel} of all elements of $N$ by
\begin{align}
    U_n := \setst{v \in \C^n}{\forall T \in N, \ \Tv = 0} = \bigcap_{T \in N} \pker{T}
    \label{Ch1:Eq:Simultaneous_Kernel_Def}
\end{align}
As an intersection of sub-vector spaces, $U_n$ is a subspace of $\C^n$. Furthermore, $U_n$ (and, by extension, all of its subspaces) are $N$-invariant: for all $T \in N$ and $v \in U_n$, we have $\Tv = 0 \in U_n$.

We are now ready to reduce the proof of Engel's Theorem to showing that all the elements of $T$ have a common eigenvector with eigenvalue $0$---or, equivalently, to showing that $U_n$ is nonzero.

\begin{lemma}
    If $U_n$ contains a nonzero element, then there exists a basis of $\C^n$ with respect to which every element of $N$ is upper-triangular.
\end{lemma}
\begin{proof}
    We argue by induction on $n$. When $n = 1$, the result is trivial: every element of $N$ (and of $\gl{n} = \gl{1}$) is upper-triangular, so the fact that $U_n = 0$ is not a problem. Now, suppose that there exists a nonzero element $v \in U_n$, ie, such that $\Tv = 0$ for all $T \in N$. Let $V = \quotient{\C^n}{\Span{v}}$. Since $\Span{v} \leq U_n$ and $U_n$ is $N$-invariant, we know that $\Span{v}$ is $N$-invariant as well, allowing us to develop the machinery developed in \Cref{Ch1:Subsec:QuotientByInvariantSubspaces} that tells us about the Lie algebraic properties of $\gl{V}$.
    
    From \Cref{Ch1:Prop:InducedActionLieAlg}, we know that the map that takes $T \in N$ to its induced action on $V$ is a Lie algebra homomorphism. Therefore, by \Cref{Ch1:Lemma:im_ker_subalg}, its image is a subalgebra of $\gl{V}$. Indeed, the elements of this subalgebra consists of nilpotent elements: for any $T \in \gl{V}$, we know there exists some $m \in \N$ such that $T^m = 0$, and the same $m$ will work for the induced action $\Tbar$ of $T$ on the quotient space: for any $x + \Span{v} \in \quotient{\C^n}{\Span{v}}$,
    \begin{align*}
        \overline{T}^m(x + U) = \overline{T}^m(x) + U = 0 + U
    \end{align*}
    Therefore, by the induction hypothesis, there exists a basis
    \begin{align*}
        \overline{B} := \set{v_1 + \Span{v}, \ldots, v_{n - 1} + \Span{v}}
    \end{align*}
    of $V$ with respect to which every element of the image of $N$ is upper-triangular. We can then lift this basis to a basis and this basis
    \begin{align*}
        B := \set{v_1, \ldots, v_{n - 1}, v}
    \end{align*}
    of $\C^n$ by adding $v$ to it. $B$ has the desired property that every element of $N$ is upper-triangular with respect to it.
\end{proof}

% What exactly does this mean? Put it elsewhere perhaps...

The way we will prove Engel's Theorem is to construct a sequence of subspaces
\begin{align*}
    0 = V_0 \subsetneq V_1 \subsetneq \cdots \subsetneq V_m = \C^n
\end{align*}
such that $N(V_i) \subseteq V_{i - 1}$. We will refine this sequence so that $m = n$, ie, so that
\begin{align*}
    \pdim{\quotient{V_{i}}{V_{i - 1}}} = 1
\end{align*}
using \Cref{Ch1:Prop:ExistsIdealCodim1}. We can then take distinguished elements from each of the quotients to form a basis of $\C^n$, and this will be the basis with respect to which every element of $N$ is upper-triangular.

We will now show that $U_n$ is, indeed, nonzero.

\begin{lemma}
    There exists a nonzero vector $v \in U_n$.
\end{lemma}
\begin{proof}
    We need to show that $\Tv = 0$ for all $T \in N$. We argue by induction on $\pdim{N}$. The base case $\pdim{N} = 1$ is clear: $N$ must be the span of a single, nilpotent element, which necessarily has a (nonzero) eigenvector with eigenvalue $0$. So, assume $N$ is such that for all Lie subalgebras of $\gl{n}$ of dimension less than $\pdim{N}$, there exists a nonzero vector in the simultaneous kernel $U_n$ of all elements of $N$.
    
    Let $A \subseteq N$ be a maximal\footnote{with respect to inclusion}, proper Lie subalgebra of $N$. Consider the map $\phi : A \to \gl{\quotient{N}{A}}$ that maps any $g \in A$ to the map that sends every $T + A \in \quotient{N}{A}$ to the map $\brac{g, T} + A \in \quotient{N}{A}$, where the map $\brac{g, T}$ is the map $gT - Tg$. % Something like the induced action of the adjoint map??? That would immediately give us all the machinery defined in the section about quotienting by invariant subalgebras.

    Observe that since $\pdim{\phiof{A}} \leq \pdim{A}$ and $\pdim{A} < \pdim{L}$ by the assumption that $A$ is proper, we know that $\pdim{A} < L$. Therefore,we can apply the induction hypothesis to $A$.
    \sorry
\end{proof}

% IDEA: Begin by taking a nonzero element of the overarching Lie algebra N. We know that its bracket with itself is zero. Its span is a subalgebra because any 1D subspace is a subalgebra. We need other guys such that their brackets with the first guy are contained in their span. This is how we build the basis we need.

There is also a more general formulation of Engel's Theorem over arbitrary Lie algebras.

\begin{boxtheorem}[Engel's Theorem, Second Version]\label{Ch1:Thm:EngelOverAnyLie}
    Let $L$ be an arbitrary Lie algebra. Then, $L$ is nilpotent if and only if for all $x \in L$, the adjoint map $\pad{x}$ is nilpotent.
\end{boxtheorem}
\begin{proof}
    Assume that $L$ is nilpotent. Then, there exists some $m \in \N$ such that $L^m = 0$. Then, any composition of Lie brackets of length $m$ is zero: for all $x, y \in L$,
    \begin{align*}
        \pad{x}^n\!(y) = \brac{x, \brac{x, \cdots \brac{x, y}\cdots}} = 0
    \end{align*}
    This gives us one direction of the proof.

    For the converse direction, we apply \Cref{Ch1:Thm:Engel} (the standard formulation of Engel's Theorem) to the \sorry
\end{proof}

\subsection{Weights of Lie Algebras}

Throughout this subsection, let $V$ be a finite-dimensional $\C$-vector space, $L$ a lie subalgebra of $\gl{V}$, and $\lambda : L \to \C$ be an arbitrary function.

\begin{boxdefinition}[Weight Space]
    We say the \textbf{weight space} of $\lambda$ with respect to $L$ %?%
    is the space
    \begin{align*}
        V_{\lambda} := \setst{v \in V}{\forall T \in L, \ \Tv = \lambda(T) \cdot v}
    \end{align*}
\end{boxdefinition}

The weight space gives us useful information about $L$ and $\lambda$.

\begin{lemma}
    If $V_{\lambda}$ is nonzero, then $\lambda$ is a linear map.
\end{lemma}
\begin{proof}
    Suppose that $V_{\lambda} \neq 0$. Fix $S, T \in L$. We know there exists a nonzero vector $v \in V$ such that $\Sof{v} = \lambda(S) \cdot v$ and $\Tv = \lambda(T) \cdot v$. In particular, we have that
    \begin{align*}
        \lambda(S + T) \cdot v = (S + T)(v) = \Sof{v} + \Tof{v} = \lambda(S) \cdot v + \lambda(T) \cdot v = (\lambda(S) + \lambda(T)) \cdot v
    \end{align*}
    Given that $\lambda(S + T)$ and $\lambda(S) + \lambda(T)$ are both scalars, we must have that $\lambda(S + T) = \lambda(S) + \lambda(T)$.
\end{proof}

\begin{lemma}
    $V_\lambda$ is a sub-vector space of $V$.
\end{lemma}
\begin{proof}
    \sorry
\end{proof}

Weight spaces are rather interesting, and in the remainder of this subsection, we will prove a lemma that will help us prove the very important \Cref{Ch1:Thm:Lie}, which we shall see shortly.

\begin{boxlemma}[The Invariance Lemma]\label{Ch1:Lemma:InvarianceLemma}  % cf. Lemma 5.5 in Erdmann Wildon
    Let $A$ be an ideal of $L$ and let $\lambda : A \to \C$ be a weight on $A$. Then, the weight space
    \begin{align*}
        V_{\lambda} = \setst{v \in V}{\forall a \in A,\ a(v) = \lambda(a) \cdot v}
    \end{align*}
    of $I$ is an $L$-invariant subspace of $V$.
\end{boxlemma}
\begin{comment} % We will follow the Erdmann-Wildon proof instead.
    Fix $u \in U$ and $T \in L$. Let $\mu : I \to \C$ be a map such that $U = I_{\mu}$. We want $\Tof{u}$ to be in $U$ as well: for any $S \in I$, we would like to have that
    \begin{align*}
        \Sof{\Tof{u}} = \muof{S} \cdot \Tof{u}
    \end{align*}
    First, observe that
    \begin{align*}
        \Sof{\Tof{u}}
        &= \text{\sorry} \\
        &= \brac{S, T}\!(u) + \Tof{\Sof{u}} \\
        &= \muof{\brac{S, T}} \cdot u + \Tof{\muof{S} \cdot u} \\
        &= \muof{\brac{S, T}} \cdot u + \muof{S} \cdot \Tof{u}
    \end{align*}
    If we can show that $\muof{\brac{S, T}} = 0$, we will be done.

    Consider the following subspace of $V$:
    \begin{align}
        W := \Span{\set{u, T(u), T^2(u), \ldots}}
    \end{align}
    Since $W \leq V$, which is finite-dimensional, we know $W$ is finite-dimensional as well. Then, let $m = \pdim{W}$. We know, from linear algebra, that $\setst{T^i(u)}{0 \leq i \leq m - 1}$ is a basis of $W$.
    
    We will begin by showing that for all $0 \leq i \leq m - 1$, there exist $\alpha_{i, 1}, \ldots, \alpha_{i, i - 1}$ such that
    \begin{align}
        ST^i(u) = 
        \label{Ch1:Eq:InvarianceLemma_1}
    \end{align}
    We will prove \eqref{Ch1:Eq:InvarianceLemma_1} by induction on $i$.

    \sorry

    \eqref{Ch1:Eq:InvarianceLemma_1} tells us the following two facts.
    \begin{enumerate}
        \item $W$ is $S$-invariant.
        \sorry
        \item There exists a basis of $W$ with respect to which the matrix of $S$ is
        $\displaystyle \begin{bmatrix}
            \muof{S} & & * \\
            & \ddots & \\
            0 & & \muof{S}
        \end{bmatrix}$.
        \sorry
    \end{enumerate}

    The second point above tells us that $\Tr{S} = \pdim{W} \cdot \muof{S}$. Since the above is true of all $S \in I$, it is, in particular, true of $\brac{T, S}$. % REPHRASE SO THAT WE USE A LETTER OTHER THAN S IN THE STEPS ABOVE!!!!!!!!!!
    From the fact that the trace of a commutator is zero, we know that this quantity must equal zero. We also know that $\pdim{W} \neq 0$. We can conclude that $\muof{\brac{T, S}} = 0$, as required.
    % Rephrase!!
\end{comment}
\begin{proof}
    Fix $y \in L$ and $w \in V_{\lambda}$. We want to show that $y(w) \in V_{\lambda}$, ie, that $y(w)$ is an eigenvector of every $a \in A$, with corresponding eigenvalue $\lambda(a)$.
\end{proof}

We end by computing all the weights and weight spaces of a few subalgebras of $\gl{n}$.

\begin{boxexample}
    Let $I$ denote the $n \times n$ identity matrix. Consider the set
    \begin{align*}
        \setst{\lambda \cdot I}{\lambda \in \C}
    \end{align*}
    One can show that this is a subalgebra of $\gl{n}$. In this case, there is only one weight: this is the map that sends every element $\lambda \cdot I$ of the above subalgebra to the corresponding $\lambda$. The weight space of this weight is all of $\C^n$.
\end{boxexample}

\begin{boxexample}
    Let $n = 3$, and let $E_{ij}$ denote the $3 \times 3$ matrix whose only nonzero entry is a $1$ in the $i$th row and $j$th column. Consider the set
    \begin{align*}
        \setst{
            % \lambda E_{11} + \lambda E_{22} + \mu E_{33}
            \begin{bmatrix}
                \alpha & 0 & 0 \\
                0 & \alpha & 0 \\
                0 & 0 & \beta
            \end{bmatrix}
        }{
            \alpha, \beta \in \C
        }
        % \begin{bmatrix} \lambda & 0 & 0 \\ 0 & \lambda & 0 \\ 0 & 0 & \mu \end{bmatrix}
    \end{align*}
    One can show that the above set is a subalgebra. \\

    The key to computing the weights and weight spaces is to consider the eigenvectors of the elements of the subalgebra. Clearly, a basis of the above is $\set{E_{11} + E_{22}, E_{33}}$. The two weights are the maps that send each of these to $1$. \sorry
\end{boxexample}

We can apply the above to compute the weights of the upper-triangular Lie subalgebra.

\begin{boxexample}[$\t{n}$]
    Consider the standard basis $\set{e_1, \ldots, e_n}$ of $\C^n$. We know that $\Span{e_1}$ is an eigenvector of every element of $\t{n}$. Indeed, one can show (\sorry) that it is the \textit{only} such simultaneous eigenvector. Therefore, the only weight is the one that maps any element of $\t{n}$ to the element that lives in its first row and first column, which is precisely the eigenvalue of $e_1$ under its action. \sorry
\end{boxexample}

Finally, we underscore the importance of weight spaces by mentioning that they can be used to prove a seemingly unrelated fact about matrices.

\begin{lemma}\label{Ch1:Lemma:SimulDiagOfCommuting}
    Let $A, B \in \gl{n}$ be diagonalisable. If $A$ and $B$ commute, then there exists a basis with respect to which both $A$ and $B$ are diagonalisable.
\end{lemma}
\begin{proof}
    Consider the subspace $L := \Span{A, B} \leq \gl{n}$. Observe that since $A$ and $B$ commute, $\brac{A, B} = 0$. Therefore, $L$ is an abelian Lie subalgebra of $\gl{n}$. The idea is that the weight space of 
\end{proof}

\begin{corollary}
    Let $A, B \in \gl{n}$ be diagonalisable. If $A$ and $B$ commute, then $A + B$ is diagonalisable.
\end{corollary}
\begin{proof}
    Rewrite $A$ and $B$ in the basis given in the previous lemma. Their sum is then a sum of diagonal matrices, which is diagonal (in that same basis).
\end{proof}

\subsection{Lie's Theorem}

In this subsection, we discuss a result similar to Engel's Theorem, but for \textit{solvable} Lie algebras instead of nilpotent ones. Throughout, we fix a \textbf{solvable} subalgebra $L \leq \gl{n}$ for some $n \in \N$.

\begin{boxtheorem}[Lie's Theorem]\label{Ch1:Thm:Lie}
    There exists a basis of $\C^n$ such that every element of $L$ is upper-triangular with respect to it.
\end{boxtheorem}

Before proceeding with the proof of Lie's Theorem, we will prove a corollary that underscores the significance of this result.

\begin{boxcorollary}\label{Ch1:Cor:SolvableDerivNilpotent}
    $L'$ is solvable.
\end{boxcorollary}
\begin{proof}
    Consider the adjoint map $\ad : L \to \gl{L}$. We know that $\pad{L}$ is solvable by \sorry; therefore, by Lie's Theorem, it is contained in $\t{L}$. % an object we have not quite defined...
    \sorry
\end{proof}

Our proof strategy will be to obtain some $\lambda \in L^*$, ie, a linear function $L \to \C$, such that ${\C^n}_{\lambda}$. We will repeat a simpler version of the proof of Engel's Theorem: we will perform induction on $n$, applying the induction hypothesis to the image of $L$ under the quotient epimorphism $\Phi_\lambda : L \surj \gl{\quotient{\C^n}{\C^n_\lambda}}$. $\Phi_\lambda(L)$ is solvable, because it is a quotient of a solvable Lie algebra. We will then be able to apply the fact that
\begin{align*}
    \pdim{\Phi_\lambda(L)}
    \leq \pdim{\C^n} - \pdim{\C^n_\lambda}
    < \pdim{\C^n} = n
\end{align*}
because $\C^n_{\lambda} \neq 0$, allowing us to apply the induction hypothesis on $\Phi_\lambda(L)$.

\sorry

We end with an example that shows that Lie's Theorem is not necessarily true over fields of prime characteristic.

\begin{boxcexample}[Lie's Theorem Fails over Prime Characteristic]
    Let $p$ be a prime number and let $F$ be a field of characteristic $p$. Consider the vector space $F^p$, and let $\set{e_1, \ldots, e_p}$ denote a basis of it. We know that $\gl{F^p}$, the set of $F$-linear maps from $F^p$ to itself, is a Lie algebra over $F$ with the commutator bracket. Denote it by $L$ for the purposes of this example. \\

    Consider the following elements of $L$:
    \begin{align*}
        x &:= e_i \mapsto i \cdot e_{i} \\
        y &:=
        \begin{cases}
            e_i \mapsto e_{i+1} & \text{ if } i < p \\
            e_p \mapsto e_1
        \end{cases}
         \in L
    \end{align*}
    We will show that $x$ and $y$ have no common eigenvectors, a fact we can use to generate a basis 
\end{boxcexample}

This is more of an aside, since we are primarily interested in complex Lie algebras in this module. Nevertheless, we mention it here because it is interesting.
