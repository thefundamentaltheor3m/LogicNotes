\section{Important Definitions and First Examples}

We will begin by defining the fundamental objects of study in this course. We will then provide some examples of these objects and discuss means of constructing them.

\subsection{Algebras}

We begin by recalling the notion of a bilinear map.

\begin{definition}[Bilinear Map]
    Let $V$ and $W$ be vector spaces. We say that a map $f: V \times W \to \C$ is \textbf{bilinear} if it is linear in each argument. That is, for all $v, v' \in V$, $w, w' \in W$ and $\lambda \in \C$, we have
    \begin{align*}
        \fof{v + v', w} = \fof{v, w} + \fof{v', w} \\
        \fof{v, w + w'} = \fof{v, w} + \fof{v, w'} \\
        \fof{\lambda v, w} = \lambda \fof{v, w} = \fof{v, \lambda w}
    \end{align*}
\end{definition}

We will be particularly interested in bilinear maps from a vector space to itself.

\begin{boxdefinition}[Algebra]
    An \textbf{algebra} is a vector space $A$ equipped with a bilinear map $\cdot: A \times A \to A$.
\end{boxdefinition}

\begin{boxconvention}
    Given any algebra $A$, we will often refer to the corresponding bilinear map $\cdot$ as the \textbf{multiplication} map of $A$, and denote $\cdot(x, y)$ as simply $x \cdot y$ or even $xy$ (where the definition of $\cdot$ is clear from the context) for any $x, y \in A$.
\end{boxconvention}

There are many different kinds of algebras. We will be particularly interested in Lie algebras and associative algebras.

\begin{boxdefinition}[Associative Algebras]
    We say that an algebra $A$ is \textbf{associative} if the multiplication map $\cdot$ is associative. That is, for all $x, y, z \in A$, we have
    \begin{align*}
        (x \cdot y) \cdot z = x \cdot (y \cdot z)
    \end{align*}
\end{boxdefinition}

We have all seen associative algebras before.

\begin{boxexample}[The Matrix Algebra]\label{Ch1:Eg:MatrixAlgebra}
    The set $\MnC$ of $n \times n$ matrices over $\C$ forms an associative algebra under matrix multiplication, known as the Matrix Algebra.
\end{boxexample}

We will come back to associative algebras soon enough. We will now define the main object of study in this module.

\begin{boxdefinition}[Lie Algebras]
    A \textbf{Lie algebra} is an algebra $L$ whose bilinear map $\brac{\cdot, \cdot}: L \times L \to L$ satisfies the following properties:
    \begin{enumerate}
        \item For all $x \in L$, we have $[x, x] = 0$.
        \item For all $x, y, z \in L$, we have
        \begin{align}
            \brac{x, \brac{y, z}} + \brac{y, \brac{z, x}} + \brac{z, \brac{x, y}} = 0
            \label{SP:eq:JacobiIdentity}
        \end{align}
    \end{enumerate}
    Such a bilinear map $\liebrac$ is known as a \textbf{Lie Bracket}, and~\eqref{SP:eq:JacobiIdentity} is known as the \textbf{Jacobi Identity}.
\end{boxdefinition}

\begin{remark}
    We immediately notice that the first condition (over not just $\C$ but any field) implies the fact that
    \begin{align}
        \brac{x, y} = - \brac{y, x}
        \label{SP:Eq:LieBracSymmNeg}
    \end{align}
    One simply needs to apply bilinearity and the first condition to evaluate $\brac{x + y, x + y}$. This argument reverses nicely as well, but only over fields of characteristic $\neq 2$.
\end{remark} 

One may recall that the $\liebrac$ notation is often used in group theory to denote the \textbf{commutator} of two elements. The reason why the same notation is used for the Lie bracket is the following.

\begin{lemma}\label{Ch1:Lemma:CommBracket}
    Let $A$ be an associative algebra. Then, the commutator map $\brac{x, y} = xy - yx$ is a Lie bracket on $A$.
\end{lemma}
\begin{proof}  % Generated by Copilot
    Clearly, $\brac{x, x} = xx - xx = 0$ for all $x \in A$. We now show that $\liebrac$ satisfies~\eqref{SP:eq:JacobiIdentity}: for all $x, y, z \in A$, we have
    \begin{align*}
        \brac{x, \brac{y, z}} + \brac{y, \brac{z, x}} + \brac{z, \brac{x, y}} &= \brac{x, yz - zy} + \brac{y, zx - xz} + \brac{z, xy - yx} \\
        &= % x(yz - zy) - (yz - zy)x + y(zx - xz) - (zx - xz)y + z(xy - yx) - (xy - yx)z \\
        % &= xyz - xzy - yzx + yxz + yzx - yxz - zxy + zyx + zxy - zyx - xyz + yxz \\
        6xyz - 6xyz = 0
    \end{align*}
    where we skip over some of the intermediate computations because they are tedious and uninteresting.
\end{proof}

\Cref{Ch1:Lemma:CommBracket} gives us a large class of examples of Lie algebras. One of the most important of these is the following.

\begin{boxexample}[General Linear Lie Algebra]\label{Ch1:Eg:gl}
    For all $n \in \N$, the set of all $n \times n$ matrices forms a Lie algebra under the commutator bracket: this follows immediately from applying \Cref{Ch1:Lemma:CommBracket} to \Cref{Ch1:Eg:MatrixAlgebra}. We call this the \textbf{General Linear Lie Algebra}, denoted $\gl{n}$.
\end{boxexample}

\begin{boxconvention}
    We will denote by $\MnC$ the set of all $n \times n$ matrices, viewed (interchangeably) as a \textit{set}, a \textit{vector space} or an \textit{associative algebra}. When viewing it as a \textit{Lie algebra under the commutator bracket}, we will adopt the notation $\gl{n, \C}$, where $\C$ can be replaced by any field. We will usually abbreviate this to $\gl{n}$, because we will primarily work over $\C$.
\end{boxconvention}

Lastly, we will define the notion of an abelian Lie algebra.

\begin{definition}[Abelian Lie Algebra]
    A Lie algebra $A$ is said to be \textbf{abelian} if for all $x \in A$, we have $\brac{x, x} = 0$.
\end{definition}

The reason for this terminology is that if $A$ is an associative algebra whose multiplication map is commutative, then its commutator bracket is identically zero, making the corresponding Lie algebra abelian.

\begin{boxexample}\label{Ch1:Eg:gl1}
    Clearly, $\gl{1}$ is abelian: for all $x, y \in \gl{1} = \C$, we have $xy - yx = 0$.
\end{boxexample}

We will now define subalgebras and homomorphisms of algebras, which will allow us to construct more examples of algebras (Lie and otherwise).

\subsection{Subalgebras and Homomorphisms}

As with objects in any category, we have subobjects and morphisms. We will define these over general algebras and apply them to get more examples of Lie algebras.

\begin{boxdefinition}[Subalgebras]
    Let $A$ be a vector space. A \textbf{subalgebra} of $A$ is a subspace $B \subseteq A$ such that $B$ is closed under the multiplication map of $A$. That is, for all $x, y \in B$, we have $x \cdot y \in B$.
\end{boxdefinition}

\begin{boxconvention}
    Given an algebra $A$ and a subset $B \subseteq A$, we will denote the statement that $B$ is a subalgebra of $A$ by $B \leq A$.
\end{boxconvention}

\begin{boxdefinition}[Homomorphisms]
    Let $A$ and $B$ be algebras. A \textbf{homomorphism} $\phi: A \to B$ is a linear map that respects the multiplication maps of $A$ and $B$. That is, for all $x, y \in A$, we have
    \begin{align*}
        \phiof{x \cdot y} = \phiof{x} \cdot \phiof{y}
    \end{align*}
\end{boxdefinition}

\begin{boxconvention}
    We will define Lie subalgebras to be subalgebras with respect to the algebra structure given by the Lie bracket, and we will define Lie algebra homomorphisms to be homomorphisms that respect the Lie bracket (ie, that are algebra homomorphisms with respect to the algebra structure given by the Lie bracket).
\end{boxconvention}

We have the following unsurprising result.

\begin{lemma}\label{Ch1:Lemma:im_ker_subalg}
    Let $A$ and $B$ be algebras, and let $\phi: A \to B$ be a homomorphism. Then,
    \begin{enumerate}[label= \normalfont\arabic*., noitemsep]
        \item $\pim{\phi} \leq B$
        \item $\pker{\phi} \leq A$
    \end{enumerate}
\end{lemma}
\begin{proof}
    These are standard results, but we will prove them for completentess.
    % AI-generated proof with minor human edits
    \begin{enumerate}
        \item Fix $x, y \in \pim{\phi}$. Then, there exist $a, b \in A$ such that $\phiof{a} = x$ and $\phiof{b} = y$. Since $\phi$ is a homomorphism, we have
        \begin{align*}
            x \cdot y = \phiof{a} \cdot \phiof{b} = \phiof{a \cdot b} \in \pim{\phi}
        \end{align*}
        so $\pim{\phi}$ is closed under the multiplication map of $B$.
        \item Let $x, y \in \pker{\phi}$. Then, we have
        \begin{align*}
            \phiof{x \cdot y} = \phiof{x} \cdot \phiof{y} = 0 \cdot 0 = 0
        \end{align*}
        where the last equality follows from the fact that $\cdot$ is bilinear. Therefore, $x \cdot y \in \pker{\phi}$, and $\pker{\phi}$ is closed under the multiplication map of $A$.
    \end{enumerate}
\end{proof}

This allows us to construct another matrix Lie algebra.

\begin{boxexample}[The Special Linear Lie Algebra]\label{Ch1:Eg:sl}
    For all $n \in \N$, consider the trace map $\operatorname{Tr} : \gl{n} \to \gl{1}$. This is a (Lie) algebra homomorphism: for all $A, B \in \gl{n}$,
    \begin{align*}
        \Tr{\brac{A, B}} = \Tr{AB - BA} = \Tr{AB} - \Tr{BA} = 0 = \brac{\Tr{A}, \Tr{B}}
    \end{align*}
    because the Lie algebra $\gl{1}$ is abelian (see \Cref{Ch1:Eg:gl1}). By \Cref{Ch1:Lemma:im_ker_subalg}, its kernel, the set of all $n \times n$ matrices of trace zero, is a Lie subalgebra of $\gl{n}$. We call this the \textbf{Special Linear Lie Algebra}, denoted $\sl{n}$.
\end{boxexample}

\begin{remark}
    In \Cref{Ch1:Eg:sl}, we have indirectly shown that
    \begin{align*}
        \pim{\liebrac} = \brac{\gl{n}, \gl{n}} \subseteq \sl{n}
    \end{align*}
    because of the unique property of the trace that $\Tr{AB} = \Tr{BA}$ for any $A, B \in \gl{n}$.
\end{remark}

The very natural relationship between associative and Lie algebra structures given by \Cref{Ch1:Lemma:CommBracket} gives us an elegant criterion for proving that a subspace is a subalgebra of a Lie algebra whose Lie bracket is the commutator of an associative bilinear map.

\begin{boxproposition}\label{Ch1:Prop:Subalg_Commbracket}
    Let $\parenth{A, \cdot_A}$ be an associative algebra and let $\parenth{B, \cdot_B}$ be a subalgebra of $A$. Denoting by $\parenth{A, \liebrac_A}$ the Lie algebra whose Lie bracket is the commutator of the multiplication map of $A$ and by $\parenth{B, \liebrac_B}$ the Lie algebra whose Lie bracket is the commutator of the multiplication map of $B$, we have $B' \leq A'$. In other words, the following diagram commutes:

    \begin{cd}
        \parenth{A, \cdot_A}
        \arrow[r] &[6em]
        \parenth{A, \liebrac_A} \\[1em]
        \parenth{B, \cdot_B}
        \arrow[r]
        \arrow[u, "\text{Associative Subalgebra}", hook] &[6em]
        \parenth{B, \liebrac_B}
        \arrow[u, "\text{Lie Subalgebra}"', hook', dashed]
        \label{Ch1:cd:Subalg_Commbracket}
    \end{cd}
\end{boxproposition}
\begin{proof}
    First, observe that $\liebrac_B = \liebrac_A\vert_B$ (ie, the Lie bracket obtained from $\cdot_B$ agrees with the one obtained from $\cdot_A$ on $B$): for all $T_1, T_2 \in B$,
    \begin{align*}
        \brac{T_1, T_2}_B = T_1 \cdot_B T_2 - T_2 \cdot_B T_1 = T_1 \cdot_A T_2 - T_2 \cdot_A T_1 = \brac{T_1, T_2}_A
    \end{align*}
    % The Lean coder in me wants to put coercion arrows everywhere... :,)
    Therefore, since $B$ is closed under $\liebrac_B$ (which, by definition, is a map from $B \times B$ to $B$), $B$ must be closed under $\liebrac_A$.
\end{proof}

This allows us to construct more examples still.

\begin{boxexample}[The Upper-Triangular Lie Algebra]
    For $n \in \N$, we define the \textbf{Upper-Triangular Lie Algebra} to be the set of all $n \times n$ upper-triangular matrices (with respect to some predetermined basis), denoted $\t{n}$. Given that the product of upper-triangular matrices is upper-triangular, $\t{n}$ forms an associative subalgebra of $\MnC$, and therefore, a Lie subalgebra of $\gl{n}$.
\end{boxexample}

\subsection{Ideals}

Throughout this subsection, we will denote by $L$ an arbitrary Lie algebra.

\begin{boxdefinition}[Ideal]\label{Ch1:Def:Ideal}
    We say that $I \subseteq L$ is an \textbf{ideal} of $L$, denoted $I \nsg L$, if $I$ is a linear subspace of $L$ and $\brac{x, y} \in I$ for all $x \in L$ and $y \in I$.
\end{boxdefinition}

\begin{boxconvention}
    We will use the notation $\brac{I, L}$ to denote the subspace of $L$ spanned by all elements of the form $\brac{i, \ell}$ for $i \in I$ and $\ell \in L$.
\end{boxconvention}

\begin{remark}
    We could equivalently require that $\brac{I, L}\leq L$ in the definition of an ideal instead of requiring that $\brac{x, y} \in I$ for all $x \in L$ and $y \in I$. Similarly, we can observe that it doesn't matter whether we require $\brac{x, y} \in I$ or $\brac{y, x} \in I$ because of~\eqref{SP:Eq:LieBracSymmNeg} and bilinearity.
\end{remark}

\begin{boxexample}[Trivial Ideals]
    Given any Lie algebra $L$, both $\set{0}$ and $L$ are ideals of $L$.
\end{boxexample}

In certain respects, despite their name, ideals of Lie algebras are more like normal subgroups of a group than they are like ideals of a ring.

\begin{lemma}\label{Ch1:Lemma:IdealSubalg}
    Any ideal $I \nsg L$ is also a subalgebra of $L$.
\end{lemma}
\begin{proof}
    This is clear from \Cref{Ch1:Def:Ideal}.
\end{proof}

\begin{lemma}
    For any Lie algebra $K$ and homomorphism $\phi : L \to K$, we have $\pker{\phi} \nsg L$.
\end{lemma}
\begin{proof}
    From \Cref{Ch1:Lemma:im_ker_subalg}, we know that $\pker{\phi}$ is a linear subspace of $L$. We now need to show that $\brac{x, y} \in \pker{\phi}$ for all $x \in L$ and $y \in \pker{\phi}$. To that end, fix $x \in \pker{\phi}$ and $y \in L$. Then,
    \begin{align*}
        \phiof{\brac{x, y}} = \brac{\phiof{x}, \phiof{y}} = \brac{0, \phiof{y}} = 0
    \end{align*}
    proving that $\brac{x, y} \in \pker{\phi}$ as required.
\end{proof}

We come back to the theme of the Lie bracket being some sort of `commutator' when we define the notion of the centre of a Lie algebra: the terminology and notation match those from group theory, where the centre consists of elements that commute with every other element of the group (making its commutator with every other element the identity).

\begin{boxdefinition}[The Centre of a Lie Algebra]
    We define the \textbf{centre} of $L$ to be
    \begin{align*}
        \Zof{L} := \setst{x \in L}{\forall y \in L, \ \brac{x, y} = 0}
    \end{align*}
\end{boxdefinition}

\begin{lemma}
    $\Zof{L}$ is an ideal of $L$.
\end{lemma}
\begin{proof}
    The fact that $\Zof{L}$ is a subspace of $L$ follows from the fact that $\liebrac$ is bilinear. Now, fix $x \in \Zof{L}$ and $y \in L$. Clearly, $\brac{x, y} = 0$, and it is easily seen that $0 \in \Zof{L}$.
\end{proof}

\begin{boxexample}
    For all $n \in \N$,
    \begin{align*}
        \Zof{\gl{n}} = \setst{A \in \gl{n}}{\exists \lambda \in \C \st A = \lambda I}
    \end{align*}
    \begin{proof}
        Let $S := \setst{A \in \gl{n}}{\exists \lambda \in \C \st A = \lambda I}$. It is clear that $S \subseteq \Zof{\gl{n}}$. Now, fix $A \in \Zof{\gl{n}}$. Then, for all $B \in \gl{n}$, we have that $\brac{A, B} = AB - BA = 0$. In particular, this implies that $A$ commutes with all the elementary matrices $E_{ij}$, which are the matrices with a $1$ in the $ij$-th position and $0$ elsewhere. Therefore, $A$ must be a diagonal matrix. 
    \end{proof}
\end{boxexample}

It turns out that ideals are well-behaved under several operations.

\begin{boxproposition}[The Behaviour of Ideals]\label{Ch1:Prop:IdealBhv}
    Let $I, J \nsg L$. Then,
    \begin{enumerate}[label = \normalfont\arabic*., noitemsep]
        \item $I + J \nsg L$
        \item $I \cap J \nsg L$
        \item $\brac{I, J} := \Span{\setst{\brac{i, j}}{i \in I, j \in J}} \nsg L$
        \item If $I + J = J$, then $I \subseteq J$.
    \end{enumerate}
\end{boxproposition}
\begin{proof}
    \sorry
\end{proof}

The abelian case is particularly nice.

\begin{proposition}[Ideals of an Abelian Lie Algebra]\label{Ch1:Prop:SubspaceIdealOfAbelian}
    Let $L$ be abelian. Then, every sub-vector space of $L$ is an ideal of $L$.
\end{proposition}
\begin{proof}
    Let $I$ be a sub-vector space of $L$. Then, for all $x \in L$ and $y \in I$, we have $\brac{x, y} = 0$. Since $I$ is a subspace, we must have $0 \in I$, proving that $I$ is an ideal of $L$.
\end{proof}

We end by defining a special kind of ideal, which will become rather important.

\begin{boxdefinition}[Derived Subalgebra]\label{Ch1:Def:DerivedSubalg}
    The \textbf{derived subalgeba} of $L$, denoted $L'$, is the ideal (and subalgebra) $\brac{L, L}$.
\end{boxdefinition}

Note that $L'$ is, indeed, an ideal, by the third property proven in \Cref{Ch1:Prop:IdealBhv}.

\begin{boxconvention}
    Though $L'$ is an ideal, we will often refer to it as either the \textbf{derived subalgebra} or the \textbf{commutator subalgebra} of $L$. Indeed, \Cref{Ch1:Lemma:IdealSubalg} tells us that this is a reasonable, if not the most completely descriptive, thing to do.
\end{boxconvention}

\subsection{Quotients}

We now define the notion of a quotient (Lie) algebra. For the remainder of this subsection, let $L$ be a Lie algebra and $I$ an arbitrary ideal of $L$. Given that we already have a notion of $\quotient{L}{I}$---recall that $I$ is a subspace of $L$, meaning we can take the quotient in a linear algebraic sense---it seems only natural to attempt to define a Lie bracket on this vector space. It turns out that the definition of an ideal allows us to do this in a very natural way.

\begin{boxproposition}\label{Ch1:Prop:QuotientAlgebraLieBracket}
    Consider the vector space $\quotient{L}{I}$. The map $\liebrac : \quotient{L}{I} \times \quotient{L}{I} \to \quotient{L}{I}$ given by
    \begin{align}
        \brac{x + I, y + I} := \brac{x, y} + I
        \label{Ch1:Eq:QuotientBracket}
    \end{align}
    for all $x, y \in L$ is a Lie bracket on $\quotient{L}{I}$.
\end{boxproposition}
\begin{proof}
    We begin by showing that the Lie bracket on $\quotient{L}{I}$ is well-defined. Fix $x, x', y, y' \in L$ with $x - x' = i \in I$ and $y - y' = j \in I$, so that $x + I = x' + I$ and $y + I = y' + I$. Then,
    \begin{align*}
        \brac{x, y} - \brac{x', y'}
        &= \brac{x' + i, y' + j} - \brac{x', y'} \\
        &= \cancel{\brac{x', y'}} + \brac{i, y'} + \brac{x', j} + \brac{i, j} - \cancel{\brac{x', y'}} \\
        &= \brac{i, y'} + \brac{x', j} + \brac{i, j} \in I
    \end{align*}
    because $I$ is an ideal, proving that $\brac{x, y} + I = \brac{x', y'} + I$, making the choice of representative irrelevant and the bracket on $\quotient{L}{I}$ well-defined.

    From the definition of $\liebrac$ on $\quotient{L}{I}$, it is clear that $\brac{x + I, x + I} = 0$ for all $x \in L$. Now, for all $x, y, z \in L$, notice that
    \begin{align*}
        \brac{x + I, \brac{y + I, z + I}} = \brac{x + I, \brac{y, z} + I} = \brac{x, \brac{y, z}} + I
    \end{align*}
    The Jacobi identity follows immediately.
\end{proof}

\begin{boxdefinition}[Quotient Algebra]
    The \textbf{quotient algebra} of $L$ with respect to $I$ is the vector space $\quotient{L}{I}$ equipped with the bracket defined in \eqref{Ch1:Eq:QuotientBracket}, which we showed to be a Lie bracket in \Cref{Ch1:Prop:QuotientAlgebraLieBracket} above.
\end{boxdefinition}

\begin{boxexample}[Quotienting by the Derived Subalgebra]
    The quotient of $L$ by $L'$ is always an abelian Lie algebra.
\end{boxexample}

% WHEN YOU MOD OUT BY THE CENTRE OF A NONABELIAN, NILPOTENT LIE ALGEBRA, THE IMAGE OF THE CENTRE IN THE QUOTIENT IS ZERO, BUT THE QUOTIENT MIGHT STILL HAVE A NONZERO CENTRE.

The centre is particularly well-behaved under taking quotients, a fact we will use when studying a class of Lie algebras called \textit{nilpotent} Lie algebras.

\begin{boxproposition}
    Let $\phi : L \surj \quotient{L}{\Zof{L}}$ be the quotient epimorphism. Then, $\phiof{\Zof{L}} = \Zof{\phiof{L}} = \Zof{\quotient{L}{\Zof{L}}}$.
\end{boxproposition}

Indeed, we can show that the map $x \mapsto x + I : L \to \quotient{L}{I}$ is a Lie algebra homomorphism. More generally, we have the following results. % Correct term?

\subsection{Isomorphism Theorems}

Our favourite isomorphism theorems do, indeed, hold in the category of Lie algebras. Throughout this subsection, let $L$ be a Lie algebra.

\begin{boxtheorem}[First Isomorphism Theorem]\label{SP:Thm:FirstIso}
    Let $K$ be a Lie algebra and $\phi : L \to K$ a Lie algebra homomorphism. Then,
    \begin{align}
        \quotient{L}{\pker{\phi}} \cong \pim{\phi}
        \label{SP:Eq:FirstIsomorphism}
    \end{align}
\end{boxtheorem}

\begin{boxtheorem}[Second Isomorphism Theorem]\label{SP:Thm:SecondIso}
    Let $I, J \nsg L$. Thhen,
    \begin{align}
        \quotient{I + J}{I} \cong \quotient{J}{I \cap J}
        \label{SP:Eq:SecondIsomorphism}
    \end{align}
\end{boxtheorem}

% Keep adding to this section as needed

We also have a correspondence between ideals of $L$ and ideals of $\quotient{L}{I}$.

\begin{boxtheorem}[The Correspondence Theorme]\label{SP:Thm:Correspondence}
    Let $I \nsg L$. Then, there is a one-to-one correspondence between the ideals of $L$ containing $I$ and the ideals of $\quotient{L}{I}$. Ie, there is a bijection
    \begin{align}
        \setst{J \nsg L}{J \supseteq I}
        \longleftrightarrow
        \set{J \nsg \quotient{L}{I}}
    \end{align}
    \label{SP:Eq:Correspondence}
\end{boxtheorem}
\begin{proof}
    
\end{proof}

Note that each of the sets in \eqref{SP:Eq:Correspondence} is partially ordered by inclusion.

\subsection{Adjoints}

Throughout this subsection, $V$ will refer to a finite-dimensional vector space.

We begin with a general Lie algebra construction.

\begin{boxdefinition}[General Linear Lie Algebra over an Arbitrary Vector Space]\label{Ch1:Def:gl_V}
    We define the \textbf{General Linear Lie Algebra over $V$} to be the set of all linear maps from $V$ to $V$, viewed as a Lie algebra under the commutator bracket. We denote it $\gl{V}$.
\end{boxdefinition}

That this is, indeed, a Lie algebra should come as no surprise. Given that this construction is well-defined over \textit{any} vector space, we can, in particular, apply it to Lie algebras.

For the remainder of this subsection, let $L$ denote an arbitrary Lie algebra. It turns out that we can define a rather nice map that relates $L$ with $\gl{L}$: the adjoint.

\begin{boxdefinition}[Adjoint Map]
    To every $x \in L$, we can associate the linear map
    \begin{align*}
        \pad{x} : L \to L : y \mapsto \brac{x, y}
    \end{align*}
    We call this map the \textbf{adjoint map} associated to $x$.
\end{boxdefinition}

\begin{boxproposition}\label{Ch1:Prop:AdjointLieAlgHom}
    The adjoint map $\ad : L \to \gl{L}$ is a Lie algebra homomorphism.
\end{boxproposition}
\begin{proof}
    That $\ad$ is linear follows from the fact that $\liebrac$ is bilinear. Now, fix $x, y \in L$, and consider the map $\pad{\brac{x, y}} \in \gl{L}$. We need to show that
    \begin{align*}
        \pad{\brac{x, y}} = \pad{x}\pad{y} - \pad{y}\pad{x}
    \end{align*}
    because the Lie bracket on $\gl{L}$ is the commutator with respect to composition of linear maps. To that end, fix $z \in L$. Then,
    \begin{align*}
        \parenth{\pad{x}\pad{y} - \pad{y}\pad{x}}\!(z)
        &= \pad{x}\!\parenth{\pad{y}\!(z)} - \pad{y}\!\parenth{\pad{x}\!(z)} & \\
        &= \pad{x}\!\parenth{\brac{y, z}} - \pad{y}\!\parenth{\brac{x, z}} & \\
        &= \brac{x, \brac{y, z}} - \brac{y, \brac{x, z}} & \\
        &= \brac{x, \brac{y, z}} + \brac{y, \brac{z, x}} & \text{(by \eqref{SP:Eq:LieBracSymmNeg})} \\
        &= - \brac{z, \brac{x, y}} & \text{(by the Jacobi Identity)} \\
        &= \brac{\brac{x, y}, z} \\
        &= \pad{\brac{x, y}}\!(z)
    \end{align*}
\end{proof}

Furthermore, we make the following observation:
\begin{boxlemma}
    $\pker{\ad} = \Zof{L}$
\end{boxlemma}
\begin{proof}
    This is immediate. We only state the result to highlight it.
\end{proof}

\subsection{Derivations}

Throughout this subsection, let $A$ be an arbitrary algebra with multiplication $\cdot$.

\begin{boxdefinition}\label{Ch1:Def:Derivation}
    We say that a linear map $D : A \to A$ is a \textbf{derivation} if it satisfies the Leibniz rule, ie, if
    \begin{align}
        D(x \cdot y) = x \cdot D(y) + D(x) \cdot y
        \label{Ch1:Eq:LeibnizRule}
    \end{align}
    for all $x, y \in A$.
\end{boxdefinition}
\begin{boxconvention}
    We will denote the set of all derivations of an algebra $A$ by $\Der{A}$.
\end{boxconvention}

Recall that since $A$ is a vector space, $\gl{A}$ is a Lie algebra with respect to the commutator bracket (cf. \Cref{Ch1:Def:gl_V}). It turns out there is a relationship between $\Der{A}$ and $\gl{A}$.

\begin{boxproposition}\label{Ch1:Prop:DerLieSubalg}
    $\Der{A}$ is a Lie subalgebra of $\gl{A}$.
\end{boxproposition}
\begin{proof}
    That $\Der{A}$ is a subspace of $\gl{A}$ is not too difficult to show: it is clear that the zero map satisfies \eqref{Ch1:Eq:LeibnizRule}, and it readily follows from the bilinearity of $\cdot$ that $\Der{A}$ is closed under addition and scalar multiplication.

    We now need to show that $\Der{A}$ is closed under the commutator bracket. Fix $D, E \in \Der{A}$. We need to show that $\brac{D, E} = DE - ED$ satisfies \eqref{Ch1:Eq:LeibnizRule}. Indeed, for all $x, y \in A$,
    \begin{align*}
        \parenth{DE - ED}\!(x \cdot y)
        &= D\!\parenth{E(x \cdot y)} - E\!\parenth{D(x \cdot y)} \\
        &= D\!\parenth{x \cdot E(y) + E(x) \cdot y} - E\!\parenth{x \cdot D(y) + D(x) \cdot y} 
        % &= x \cdot D\!\parenth{E(y)} + D(x) \cdot E(y) + E(x) \cdot D(y) + D\!\parenth{E(x)} \cdot y
    \end{align*}
    which can be simplified, if tediously, to the desired form.
\end{proof}

Most readers will have encountered derivations before. We give below a classic example (over $\R$, for the first time so far) that the reader is sure to recognise.

\begin{boxexample}
    The space $C^{\infty}(\R)$ of smooth $\R \to \R$ functions is an $\R$-algebra under pointwise addition and multiplication. The differentiation map $D : C^{\infty}(\R) \to C^{\infty}(\R) : f \mapsto f'$ is easily seen to be a derivation.
\end{boxexample}

We have also encountered a slightly more sophisticated derivation. For the remainder of this subsection, let $L$ be an arbitrary Lie algebra.

\begin{boxproposition}
    For all $x \in L$, the adjoint map $\pad{x} : L \to L : y \mapsto \brac{x, y}$ associated with $x$ is a derivation.
\end{boxproposition}
\begin{proof}
    We already know that $\pad{x} \in \gl{L}$. It only remains to show that $\pad{x}$ satisfies \eqref{Ch1:Eq:LeibnizRule} with respect to $\liebrac$. To that end, fix $y, z \in L$. Then, we have that
    \begin{align*}
        \pad{x}\!\parenth{\brac{y, z}}
        &= \brac{x, \brac{y, z}} \\
        &= - \brac{y, \brac{z, x}} - \brac{z, \brac{x, y}} \\
        &= \brac{y, \brac{x, z}} + \brac{\brac{x, y}, z} \\
        &= \brac{y, \pad{x}\!(z)} + \brac{\pad{x}\!(y), z}
    \end{align*}
    as required.
\end{proof}

Abbreviating the set $\setst{\pad{x}}{x \in L}$ of all adjoint maps on $L$ to $\pad{L}$, we have the following chain of Lie subalgebras:

\begin{boxlemma}\label{Ch1:Lemma:adSubalgDer}
    $\pad{L} \leq \Der{L} \leq \gl{L}$
\end{boxlemma}
\begin{proof}
    \sorry
\end{proof}

In fact, $\pad{L}$ is more than just a subalgebra of $\Der{L}$.

\begin{boxlemma}
    $\pad{L} \nsg \Der{L}$.
\end{boxlemma}

\subsection{Structure Constants}

Fix $n \in \N$, and let $L$ be an $n$-dimensional Lie algebra. Consider the $\C$-basis $\B = \set{e_1, \ldots, e_n}$ of $L$. Given the fundamentally linear algebraic nature of Lie algebras, it is natural to study what happens when we apply the Lie bracket to elements of $\B$.

\begin{boxdefinition}[Structure Constants]
    Fix $i, j \in \set{1, \ldots, n}$. We know that there exist unique constants $s_{ij1}, s_{ij2}, \ldots, s_{ijn}$ such that
    \begin{align*}
        \brac{e_i, e_j} = \sum_{k = 1}^{n} s_{ijk} e_k
    \end{align*}
    We call the scalars $\set{s_{ijk}}_{1 \leq i, j, k \leq n}$ the \textbf{structure constants} of $L$ with respect to $\B$.
\end{boxdefinition}

%% TODO: Describe how structure constants give isomorphisms
% Also talk about how they relate to matrices and how we can describe identities in them like Jacobi and skew-symmetry

\subsection{Direct Sums}

In this subsection, we briefly describe the theory of the direct sum of two Lie algebras. Let $L_1$ and $L_2$ be arbitrary Lie algebras. Just as we did in \Cref{Ch1:Prop:QuotientAlgebraLieBracket}, we will define a Lie bracket on the vector space $L_1 \+ L_2$, and define the Lie algebra direct sum of $L_1$ and $L_2$ to be this vector space equipped with this bracket.

\begin{boxproposition}\label{Ch1:Prop:DirectSumLieBracket}
    Define the map $\liebrac : \parenth{L_1 \+ L_2} \times \parenth{L_1 \+ L_2} \to \parenth{L_1 \+ L_2}$ given by
    \begin{align}
        \brac{x_1 \+ x_2, y_1 \+ y_2} := \brac{x_1, y_1} \+ \brac{x_2, y_2}
        \label{Ch1:Eq:DirectSumBracket}
    \end{align}
    for all $x_1, y_1 \in L_1$ and $x_2, y_2 \in L_2$. Then, $\liebrac$ is a Lie bracket on $L_1 \+ L_2$.
\end{boxproposition}
\begin{proof}
    \sorry
\end{proof}

\begin{boxdefinition}[Direct Sum]\label{Ch1:Def:DirectSum}
    The \textbf{direct sum} of $L_1$ and $L_2$ is the vector space $L_1 \+ L_2$ equipped with the bracket defined in \eqref{Ch1:Eq:DirectSumBracket}, which we showed to be a Lie bracket in \Cref{Ch1:Prop:DirectSumLieBracket} above.
\end{boxdefinition}

We can repeat this definition successively to define the direct sum of any finite number of Lie algebras. We will not explore this idea in any more detail and will take it for granted.
